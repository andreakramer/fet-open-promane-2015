% the document class specification for the proposal writing process, add the 'submit' option
% for submitting (switches off various draft features); add the 'public' option to exclude
% any private parts. 
\documentclass[noworkareas,deliverables,gitinfo]{euproposal}
%\documentclass[submit,noworkareas,deliverables]{euproposal}
%\documentclass[submit,public,noworkareas,deliverables]{euproposal}
%% TODO these don't work with WA (https://trac.kwarc.info/sTeX/ticket/1697)
%\usepackage[T1]{fontenc}
%\usepackage[utf8]{inputenc}
\addbibresource{../lib/dummy}
%%% institutions
\WAinstitution[id=KUL,
        countryshort=BE,
        acronym=KUL]
        {KU Leuven}

\WAinstitution[id=PAR2,
        countryshort=BE,
        acronym=Partner2]
        {Partner Two}

\WAinstitution[id=PAR3,
        countryshort=BE,
        acronym=Partner3]
        {Partner Three}

\WAperson[id=cmaes,
           personaltitle=Prof.,
           birthdate=1 Jan. 2000,
           academictitle=Professor,
           affiliation=KUL,
           department=Instituut voor Theoretische Fysica,
           privaddress=None of your business,
           privtel=that neither,
           email=christian.maes@fys.kuleuven.be,
           workaddress={Celestijnenlaan 200D, B-3001 Leuven, Belgium},
           worktel=+32 16 32 72 33,
           workfax=N/A
           ]
           {Christian Maes}

%%% Local Variables: 
%%% mode: latex
%%% TeX-master: "proposal"
%%% End: 
% Some sections of the included files depend on this.


\begin{document}
\begin{center}\color{red}\huge
  This mock proposal is just an example for \texttt{euproposal.cls} it reflects the ICT
  template of January 2012
\end{center}
\begin{proposal}[site=jacu,jacuRM=36,
  site=efo,efoRM=36,
  site=bar,barRM=36,
  site=baz,bazRM=36,
  coordinator=miko,
  acronym={iPoWr},
  acrolong={\underline{I}ntelligent} {\underline{P}r\underline{o}sal} {\underline{Wr}iting},
  title=\pn: \protect\pnlong,
  callname = ICT Call 1,
  callid = FP7-???-200?-?,
  instrument= Small or Medium-Scale Focused Research Project (STREP), 
  challengeid = 4,
  challenge = ICT for EU Proposals,
  objectiveid={ICT-2012.4.4}, 
  objective = Technology-enhanced Documents,
  outcomeid = b1,
  outcome = {More time for Research, not Proposal writing},
  coordinator=miko,
  months=24,
  compactht]
\begin{abstract}
  Writing grant proposals is a collaborative effort that requires the integration of
  contributions from many individuals. The use of an ASCII-based format like {\LaTeX}
  allows to coordinate the process via a source code control system like
  {\textsc{Subversion}}, allowing the proposal writing team to concentrate on the contents
  rather than the mechanics of wrangling with text fragments and revisions.
\end{abstract}

\tableofcontents

\begin{todo}{from the proposal template}
  Recommended length for the whole part B: 50--60 pages (including tables, references,
  etc.)
\end{todo}
\chapter{Scientific and Technical Quality}\label{chap:quality}
\begin{todo}{from the proposal template}
  Maximum length for the whole of Section 1 –-- twenty pages, not including the tables in
  Section 1.3
\end{todo}

\section{Targeted breakthrough and long-term vision}\label{sec:objectives}
\begin{todo}{from the proposal template}
  Describe the breakthrough(s) that you are targeting to achieve. What is the long-term
  vision (scientific, technological, societal, other) that motivates this breakthrough?
  Explain how this breakthrough is an essential step towards the achievement of your
  long-term vision, in particular in terms of new forms and uses of information and
  information technologies. Describe the concrete objectives that you consider to
  constitute the proof-of-concept of such a breakthrough. The objectives should be those
  that you consider achievable within the project, in spite of the inherent risks. They
  should be stated in a verifiable form, including through the milestones that will be
  indicated under Section 1.3 below.
\end{todo}

%%% Local Variables: 
%%% mode: latex
%%% TeX-master: "propB"
%%% End: 

\section{Novelty and foundational character}\label{sec:progress}
\begin{todo}{from the proposal template}
  Describe the state-of-the-art in the area(s) concerned, and the advance that the
  proposed project would bring about. Clearly describe the novelty of your proposal. In
  what way do you challenge current thinking or assumptions? Novelty should come from new
  ideas, not from the incremental refinement of existing approaches. It can also come from
  new and unexpected combinations of insights from various disciplines. What is the
  scientific foundation that you aim to develop and what are the specific contributions to
  science and technology that your project will make (including in case of failure)?
\end{todo}

%%% Local Variables: 
%%% mode: latex
%%% TeX-master: "propB"
%%% End: 

\section[S/T Methodology]{S/T Methodology\footnote{Note that, whereas the scientific and technological methodology is evaluated
under the criteria ‘S/T quality’, the quality of the
actual workplan is evaluated under FET-Open under the criteria ‘Implementation’.}}\label{sec:methodology}
\begin{todo}{from the proposal template}
Provide a detailed description of the scientific and technological approach or methodology
by which you will attempt to reach your objectives. Demonstrate that you are aware of the
level and nature of the risks of failure, and that you have a good idea on how to address
these risks. Describe a progression of crucial milestones and decision points for your
project, and their expected timing. What would constitute success? What would you learn
from an eventual failure? Where relevant, show how your approach takes into account the
difficulties inherent to the multi-disciplinary nature of the idea or approach that you
are proposing. 
 A detailed work plan should be presented, broken down into work
packages\footnote{A work package is a major sub-division of the proposed project with a
  verifiable end-point – normally a deliverable or an important milestone in the overall
  project.} (WPs) which should follow the logical phases of the implementation of the
project, and include consortium management and assessment of progress and results. (Note
that your overall approach to management will be described later, in Section 2).

Notes: The number of work packages used must be appropriate to the complexity of the work
and the overall value of the proposed project. The planning should be sufficiently
detailed to justify the proposed effort and allow progress monitoring by the Commission.

\end{todo}
\newpage\begin{todo}{from the proposal template}
\begin{enumerate}
\item Describe the overall strategy of the work plan\ednote{Maximum length – one page}
\item Show the timing of the different WPs and their components (Gantt chart or similar).
\end{enumerate}
\end{todo}
\begin{figure}
  \caption{Work package dependencies}
  \label{fig:wp-deps}
\end{figure}

\ganttchart[draft,xscale=.45] 

%%% Local Variables: 
%%% mode: LaTeX
%%% TeX-master: "propB"
%%% End: 

% LocalWords:  workplan.tex ednote wp-deps ganttchart xscale


\newpage
\subsection{Work Package List}\label{sec:wplist}

\begin{todo}{from the proposal template}
Please indicate one activity per work package:
RTD = Research and technological development; DEM = Demonstration; MGT = Management of the consortium
\end{todo}

%\makeatletter\wp@total@RM{management}\makeatother
\wpfig[pages,type]

\newpage\subsection{List of Deliverables}\label{sec:deliverables}

\begin{todo}{from the proposal template}
\begin{compactenum}
\item Deliverable numbers in order of delivery dates. Please use the numbering convention <WP number>.<number of deliverable within
that WP>. For example, deliverable 4.2 would be the second deliverable from work package 4.
\item Please indicate the nature of the deliverable using one of the following codes:
R = Report, P = Prototype, D = Demonstrator, O = Other
\item Please indicate the dissemination level using one of the following codes:
PU = Public
PP = Restricted to other programme participants (including the Commission Services).
RE = Restricted to a group specified by the consortium (including the Commission Services).
CO = Confidential, only for members of the consortium (including the Commission Services).
\end{compactenum}
\end{todo}
We will now give an overview over the deliverables and milestones of the work
packages. Note that the times of deliverables after month 24 are estimates and may change
as the work packages progress.

In the table below, {\emph{integrating work deliverables}} (see top of
section~\ref{sec:wplist}) are printed in boldface to mark them. They integrate
contributions from multiple work packages. \ednote{CL: the rest of this paragraph does not
  comply with the EU guide for applicants, needs to be rewritten}These can have the
dissemination level ``partial'', which indicates that it contains parts of level
``project'' that are to be disseminated to the project and evaluators only. In such
reports, two versions are prepared, and disseminated accordingly.

{\footnotesize\inputdelivs{8cm}}


%%% Local Variables: 
%%% mode: latex
%%% TeX-master: "propB"
%%% End: 

\newpage\eucommentary{Milestones means control points in the project that help to chart progress. Milestones may
correspond to the completion of a key deliverable, allowing the next phase of the work to begin.
They may also be needed at intermediary points so that, if problems have arisen, corrective
measures can be taken. A milestone may be a critical decision point in the project where, for
example, the consortium must decide which of several technologies to adopt for further
development.}

\begin{milestones}
\milestone[id=start, month=1, verif={Public announcement of open positions. Website online.}]
%
{Starting up}
%
{Formal start of the project: Open the research positions, set up the website.}

\milestone[id=models,month=12, verif={N/A}]
%
{Systems setup}
%
{Set up the model systems for both experiments and simulation software.}

\milestone[id=data1,month=20, verif={Availability of data.}]
%
{Data I}
%
{The first round of experiments and simulation campaigns provides data.}

\milestone[id=framework,month=28, verif={Availability of theoretical models. Publication of
  first technological report.}]
%
{Common theoretical framework}
%
{The physical properties of all model systems, theoretical and experimental, are formulated
  in a unified manner. The role of nonequilibrium in the properties of the systems is given
  a meaning.}

\milestone[id=data2,month=42, verif={Availability of data.}]
%
{Data II}
%
{The secound round of experiments and simulation campaigns provides data.}

\milestone[id=final,month=48, verif={Publication of final report and second technological
  report. Joint theoretical/experimental publication on the project.}]%
{Final milestone}
%
{Join the theoretical model and experimental results in joint publications}

\end{milestones}

%%% Local Variables:
%%% mode: latex
%%% TeX-master: "proposal"
%%% End:



\subsection{Work Package Descriptions}\label{sec:workpackages}
\begin{workplan}
\begin{workpackage}[id=management,type=MGT,wphases=0-24!.2,
  title=Project Management,short=Management,
  jacuRM=2,barRM=2,efoRM=2,bazRM=2]
We can state the state of the art and similar things before the summary in the boxes
here. 
\wpheadertable
\begin{wpobjectives}
  \begin{itemize}
    \item To perform the administrative, scientific/technical, and financial
      management of the project
    \item To co-ordinate the contacts with the EU
    \item To control quality and timing of project results and to resolve conflicts
    \item To set up inter-project communication rules and mechanisms
  \end{itemize}
\end{wpobjectives}

\begin{wpdescription}
  Based on the Consortium Agreement, i.e. the contract with the European Commission, and
  based on the financial and administrative data agreed, the project manager will carry
  out the overall project management, including administrative management.  A project
  quality handbook will be defined, and a {\pn} help-desk for answering questions about
  the format (first project-internal, and after month 12 public) will be established. The
  project management will\ldots we can even reference deliverables:
  \delivref{management}{report2} and even the variant with a title:
  \delivtref{management}{report2}
\end{wpdescription}

\begin{wpdelivs}
  \begin{wpdeliv}[due=1,id=mailing,nature=O,dissem=PP,miles=kickoff]
    {Project-internal mailing lists}
  \end{wpdeliv}
  \begin{wpdeliv}[due=3,id=handbook,nature=R,dissem=PU,miles=consensus]
    {Project management handbook}
  \end{wpdeliv}
\begin{wpdeliv}[due={6,12,18,24,30,36,42,48},id=report2,nature=R,dissem=public,miles={consensus,final}]
  {Periodic activity report} 
  Partly compiled from activity reports of the work package
  coordinators; to be approved by the work package coordinators before delivery to the
  Commission.  Financial reporting is mainly done in months 18 and 36.\Ednote{how about
    these numbers?}
  \end{wpdeliv}
 \begin{wpdeliv}[due=6,id=helpdesk,dissem=PU,nature=O,miles=kickoff]
    {{\pn} Helpdesk}
  \end{wpdeliv}
  \begin{wpdeliv}[due=36,id=report6,nature=R,dissem=PU,miles=final]
    {Final plan for using and disseminating the knowledge}
  \end{wpdeliv}
  \begin{wpdeliv}[due=48,id=report7,nature=R,dissem=PU,miles=final]
    {Final management report}
  \end{wpdeliv}
\end{wpdelivs}
\end{workpackage}

%%% Local Variables: 
%%% mode: LaTeX
%%% TeX-master: "propB"
%%% End: 

% LocalWords:  wp-management.tex workpackage efoRM bazRM wpheadertable pn ldots
% LocalWords:  wpobjectives wpdescription delivref delivtref wpdelivs wpdeliv
% LocalWords:  dissem Ednote pdataRef deliv mansubsusintReport wphases
\newpage
\begin{workpackage}%
[id=dissem,type=RTD,lead=efo,
 wphases=10-24!1,
 title=Dissemination and Exploitation,short=Dissemination,
 efoRM=8,jacuRM=2,barRM=2,bazRM=2]
We can state the state of the art and similar things before the summary in the boxes
here. 
\wpheadertable

\begin{wpobjectives}
  Much of the activity of a project involves small groups of nodes in joint work. This
  work package is set up to ensure their best wide-scale integration, communication, and
  synergetic presentation of the results. Clearly identified means of dissemination of
  work-in-progress as well as final results will serve the effectiveness of work within
  the project and steadily improve the visibility and usage of the emerging semantic
  services.
\end{wpobjectives}

\begin{wpdescription}
  The work package members set up events for dissemination of the research and
  work-in-progress results for researchers (workshops and summer schools), and for
  industry (trade fairs). An in-depth evaluation will be undertaken of the response of
  test-users.

  Within two months of the start of the project, a project website will go live. This
  website will have two areas: a members' area and a public area.\ldots
\end{wpdescription}

\begin{wpdelivs}
  \begin{wpdeliv}[due=2,id=website,nature=O,dissem=PU,miles=kickoff]
     {Set-up of the Project web server}
   \end{wpdeliv}
   \begin{wpdeliv}[due=8,id=ws1proc,nature=R,dissem=PU,miles={kickoff}]
     {Proceedings of the first {\pn} Summer School.}
   \end{wpdeliv}
   \begin{wpdeliv}[due=9,id=dissem,nature=R,dissem=PP]
     {Dissemination Plan}
   \end{wpdeliv}
   \begin{wpdeliv}[due=9,id=exploitplan,nature=R,dissem=PP,miles=exploitation]
     {Scientific and Commercial Exploitation Plan}
   \end{wpdeliv}
   \begin{wpdeliv}[due=20,id=ws2proc,nature=R,dissem=PU,miles={exploitation}]
     {Proceedings of the second {\pn} Summer School.}
   \end{wpdeliv}
   \begin{wpdeliv}[due=32,id=ss1proc,nature=R,dissem=PU,miles={exploitation}]
     {Proceedings of the third {\pn} Summer School.}
   \end{wpdeliv}
   \begin{wpdeliv}[due=44,id=ws3proc,nature=R,dissem=PU,miles=exploitation]
     {Proceedings of the fourth {\pn} Summer School.}
   \end{wpdeliv}
 \end{wpdelivs}
\end{workpackage}

%%% Local Variables: 
%%% mode: LaTeX
%%% TeX-master: "propB"
%%% End: 

% LocalWords:  wp-dissem.tex workpackage dissem efo fromto bazRM wpheadertable
% LocalWords:  wpobjectives wpdescription ldots wpdelivs wpdeliv ws1proc pn
% LocalWords:  exploitplan ws2proc ss1proc ws3proc pdataRef deliv
% LocalWords:  mansubsusintReport
\newpage
\begin{workpackage}[id=class,type=RTD,lead=jacu,
                    wphases=3-9!1,
                    title=A {\LaTeX} class for EU Proposals,short=Class,
                    jacuRM=12,barRM=12]
We can state the state of the art and similar things before the summary in the boxes
here. 
\wpheadertable
\begin{wpobjectives}
\LaTeX is the best document markup language, it can even be used for literate
programming~\cite{DK:LP,Lamport:ladps94,Knuth:ttb84}

  To develop a {\LaTeX} class for marking up EU Proposals
\end{wpobjectives}

\begin{wpdescription}
  We will follow strict software design principles, first comes a requirements analys,
  then \ldots
\end{wpdescription}

\begin{wpdelivs}
  \begin{wpdeliv}[due=6,id=req,nature=R,dissem=PP,miles=kickoff]
     {Requirements analysis}
   \end{wpdeliv}
   \begin{wpdeliv}[due=12,id=spec,nature=R,dissem=PU,miles=consensus]
     {{\pn} Specification }
   \end{wpdeliv}
   \begin{wpdeliv}[due=18,id=demonstrator,nature=P,dissem=PU,miles={consensus,final}]
     {First demonstrator ({\tt{article.cls}} really)}
   \end{wpdeliv}
   \begin{wpdeliv}[due=24,id=proto,nature=P,dissem=PU,miles=final]
     {First prototype}
   \end{wpdeliv}
    \begin{wpdeliv}[due=36,id=release,nature=P,dissem=PU,miles=final]
      {Final {\LaTeX} class, ready for release}
    \end{wpdeliv}
  \end{wpdelivs}
Furthermore, this work package contributes to {\pdataRef{deliv}{managementreport2}{label}} and
{\pdataRef{deliv}{managementreport7}{label}}.
\end{workpackage}

%%% Local Variables: 
%%% mode: LaTeX
%%% TeX-master: "propB"
%%% End: 
\newpage
\begin{workpackage}[id=temple,type=DEM,lead=bar,
  wphases=6-12!1,
  title={\pn} Proposal Template,short=Template,barRM=6,bazRM=6]
We can state the state of the art and similar things before the summary in the boxes
here. 
\wpheadertable

\begin{wpobjectives}
  To develop a template file for {\pn} proposals
\end{wpobjectives}

\begin{wpdescription}
  We abstract an example from existing proposals
\end{wpdescription}

\begin{wpdelivs}
  \begin{wpdeliv}[due=6,id=req,nature=R,dissem=PP,miles=kickoff]
    {Requirements analysis}
  \end{wpdeliv}
  \begin{wpdeliv}[due=12,id=spec,nature=R,dissem=PU,miles=consensus]
    {{\pn} Specification }
  \end{wpdeliv}
  \begin{wpdeliv}[due=18,id=demonstrator,nature=D,dissem=PU,miles={consensus,final}]
    {First demonstrator ({\tt{article.cls}} really)}
  \end{wpdeliv}
  \begin{wpdeliv}[due=24,id=proto,nature=P,dissem=PU,miles=final]
    {First prototype}
  \end{wpdeliv}
  \begin{wpdeliv}[due=36,id=release,nature=P,dissem=PU,miles=final]
    {Final Template, ready for release}
  \end{wpdeliv}
\end{wpdelivs}
Furthermore, this work package contributes to {\pdataRef{deliv}{managementreport2}{label}} and
{\pdataRef{deliv}{managementreport7}{label}}.
\end{workpackage}

%%% Local Variables: 
%%% mode: LaTeX
%%% TeX-master: "propB"
%%% End: 

% LocalWords:  wp-temple.tex workpackage fromto pn bazRM wpheadertable wpdelivs
% LocalWords:  wpobjectives wpdescription wpdeliv req dissem tt article.cls
% LocalWords:  pdataRef deliv systemsintReport
\newpage
\end{workplan}
\newpage\subsection{Significant Risks and Associated Contingency Plans}\label{sec:risks}
\begin{todo}{from the proposal template}
  Describe any significant risks, and associated contingency plans
\end{todo}
\begin{oldpart}{need to integrate this somewhere. CL: I will check other proposals to see how they did it; the Guide does not really prescribe anything.}
\paragraph{Global Risk Management}
The crucial problem of \pn (and similar endeavors that offer a new basis for communication
and interaction) is that of community uptake: Unless we can convince scientists and
knowledge workers industry to use the new tools and interactions, we will
never be able to assemble the large repositories of flexiformal mathematical knowledge we
envision. We will consider uptake to be the main ongoing evaluation criterion for the network.
\end{oldpart}

%%% Local Variables: 
%%% mode: latex
%%% TeX-master: "propB"
%%% End: 



%%% Local Variables: 
%%% mode: latex
%%% TeX-master: "propB"
%%% End: 

% LocalWords: workplan newpage wplist makeatletter makeatother wpfig
% LocalWords:  workpackages wp-dissem wp-class wp-temple

%%% Local Variables: 
%%% mode: LaTeX
%%% TeX-master: "propB"
%%% End: 

\newpage
\chapter{Implementation}\label{chap:implementation}

\section{Management Structure and Procedures}\label{chap:management}
\begin{todo}{from the proposal template}
  Describe the organizational structure and decision-making mechanisms
  of the project. Show how they are matched to the nature, complexity
  and scale of the project.  Maximum length of this section: five pages.
\end{todo}

The Project Management of {\pn} is based on its Consortium Agreement, which will be
signed before the Contract is signed by the Commission. The Consortium Agreement will
enter into force as from the date the contract with the European Commission is signed.
\subsection{Organizational structure}\label{sec:management-structure}
\subsection{Risk Assessment and Management}
\subsection{Information Flow and Outreach}\label{sec:spread-excellence}
\subsection{Quality Procedures}\label{sec:quality-management}
\subsection{Internal Evaluation Procedures}
\newpage
\section{Individual Participants}\label{sec:partners}
\begin{todo}{from the proposal template}
For each participant in the proposed project, provide a brief description of the legal entity, the main
tasks they have been attributed, and the previous experience relevant to those tasks. Provide also a
short profile of the individuals who will be undertaking the work.\\
Maximum length for Section 2.2: one page per participant. However, where two or more departments within
an organisation have quite distinct roles within the proposal, one page per department is acceptable.\\
The maximum length applying to a legal entity composed of several members, each of which is a separate
legal entity (for example an EEIG1), is one page per member, provided that the members have quite distinct
roles within the proposal.
\end{todo}
\newpage
\begin{sitedescription}{jacu}

\paragraph{Organization} Jacobs University Bremen is a private research university patterned
after the Anglo-Saxon university system.  The university opened in
2001 and has an international student body ($1,245$ students from 102
nations as of 2011, admitted in a highly selective process).

The KWARC (KnoWledge Adaptation and Reasoning for
Content\footnote{\url{http://kwarc.info}}) Group headed by
{\emph{Prof.\ Dr.\ Michael Kohlhase}} specializes in building
knowledge management systems for e-science applications, in particular
for the natural and mathematical sciences.  Formal logic, natural
language semantics, and semantic web technology provide the
foundations for the research of the group.
  
  Since doing research and developing systems is much more fun than writing proposals,
  they try go do that as efficiently as possible, hence this meta-proposal. 

\paragraph{Main tasks}

\begin{itemize}
\item creating {\LaTeX} class files
\end{itemize}

\paragraph{Relevant previous experience}

The KWARC group is the main center and lead implementor of the OMDoc
(Open Mathematical Document) format for representing mathematical
knowledge.  The group has developed added-value services powered by such semantically rich representations, different paths to obtaining them, as well as platforms that integrate both aspects.  Services include the adaptive context-sensitive presentation framework JOMDoc and the semantic search engine MathWebSearch.  For obtaining rich mathematical content, the group has been pursuing the two alternatives of assisting manual editing (with the sTeXIDE editing environment) and automatic annotation using natural language processing techniques.  The latter is work in progress but builds on the arXMLiv system, which is currently capable of converting 70\% out of the 600,000 scientific publications in the arXiv from {\LaTeX} to XHTML+MathML without errors.  Finally, the KWARC group has been developing the Planetary integrated environment.

\paragraph{Specific expertise}

\begin{itemize}
\item writing intelligent proposals
\end{itemize}

\paragraph{Staff members involved}

\textbf{Prof.\ Dr.\ Michael Kohlhase} is head of the KWARC research
group.  He is the head developer of the OMDoc mathematical markup
language.  He was a member of the Math Working Group at W3C, which finished its work with the publication of the MathML 3 recommendation.  He is president of the OpenMath society and trustee of the MKM
interest group.

\keypubs{KohDavGin:psewads11,Kohlhase:pdpl10,Kohlhase:omdoc1.2,CarlisleEd:MathML10,StaKoh:tlcspx10}
\end{sitedescription}

%%% Local Variables: 
%%% mode: LaTeX
%%% TeX-master: "propB"
%%% End: 

% LocalWords:  site-jacu.tex sitedescription emph textbf keypubs KohDavGin
% LocalWords:  psewads11 pdpl10 StaKoh tlcspx10
\newpage
\begin{sitedescription}{efo}
\paragraph{Organization}
 The EFO is the world leader in futurology, \ldots
\paragraph{Main tasks}
\paragraph{Relevant previous experience}
\paragraph{Specific expertise}
\paragraph{Staff members undertaking the work}
\keypubs{providemore}
\end{sitedescription}

%%% Local Variables: 
%%% mode: LaTeX
%%% TeX-master: "propB"
%%% End: 
\newpage
\begin{sitedescription}{bar}

\paragraph{Organization}
  Universit\'e de BAR specializes on drinking lots of red wine. It is a partner in the
  consortium, because it has a very nice chateau on the Cote d'Azure, where it can host
  gorgeous project meetings.

\paragraph{Main tasks}
\paragraph{Relevant previous experience}
\paragraph{Specific expertise}
\paragraph{Staff members undertaking the work}
\keypubs{providemore}

\end{sitedescription}

%%% Local Variables: 
%%% mode: LaTeX
%%% TeX-master: "propB"
%%% End: 
\newpage
\begin{sitedescription}{baz}
\paragraph{Organization}
\paragraph{Main tasks}
\paragraph{Relevant previous experience}
\paragraph{Specific expertise}
\paragraph{Staff members undertaking the work}
\keypubs{providemore}
\end{sitedescription}

%%% Local Variables: 
%%% mode: LaTeX
%%% TeX-master: "propB"
%%% End: 
\newpage

\section{The {\protect\pn} consortium as a whole}
\begin{todo}{from the proposal template}
  Describe how the participants collectively constitute a consortium capable of achieving
  the project objectives, and how they are suited and are committed to the tasks assigned
  to them. Show the complementarity between participants. Explain how the composition of
  the consortium is well-balanced in relation to the objectives of the project.  

  If appropriate describe the industrial/commercial involvement to ensure exploitation of
  the results. Show how the opportunity of involving SMEs has been addressed
\end{todo}

The project partners of the \pn project have a long history of successful collaboration;
Figure~\ref{tab:collaboration} gives an overview over joint projects (including proposals) and
joint publications (only international, peer reviewed ones).

\jointorga{jacu,efo,baz}
\jointpub{efo,baz,jacu}
\jointproj{efo,bar}
\coherencetable

\subsection{Subcontracting}\label{sec:subcontracting}
\begin{todo}{from the proposal template}
  If any part of the work is to be sub-contracted by the participant responsible for it,
  describe the work involved and explain why a sub-contract approach has been chosen for
  it.
\end{todo}
\subsection{Other Countries}\label{sec:other-countries}
\begin{todo}{from the proposal template}
  If a one or more of the participants requesting EU funding is based outside of the EU
  Member states, Associated countries and the list of International Cooperation Partner
  Countries\footnote{See CORDIS web-site, and annex 1 of the work programme.}, explain in
  terms of the project’s objectives why such funding would be essential.
\end{todo}

\subsection{Additional Partners}\label{sec:assoc-partner}
\begin{todo}{from the proposal template}
  If there are as-yet-unidentified participants in the project, the expected competences,
  the role of the potential participants and their integration into the running project
  should be described
\end{todo}
\section{Resources to be Committed}\label{sec:resources}
\begin{todo}{from the proposal template}
Maximum length: two pages

Describe how the totality of the necessary resources will be mobilized, including any resources that
will complement the EC contribution. Show how the resources will be integrated in a coherent way,
and show how the overall financial plan for the project is adequate.

In addition to the costs indicated on form A3 of the proposal, and the effort shown in Section 1.3
above, please identify any other major costs (e.g. equipment). Ensure that the figures stated in Part B
are consistent with these.
\end{todo}

\subsection{Travel Costs and Consumables}\label{sec:travel-costs}
\subsection{Subcontracting Costs}
\subsection{Other Costs}

%%% Local Variables: 
%%% mode: LaTeX
%%% TeX-master: "propB"
%%% End: 

% LocalWords:  pn newpage site-jacu site-efo site-baz jointpub efo baz
% LocalWords:  jointproj coherencetable assoc-partner
\newpage
\subsection{Expected impacts}

\eucommentary{
  Please be specific, and provide only information that applies to the proposal
  and its objectives. Wherever possible, use quantified indicators and targets.
%
  \begin{compactitem}
  \item Describe how your project will contribute to the expected impacts set out
    in the work programme under the relevant topic.
  \item Describe the importance of the technological outcome with regards to its
    transformational impact on science, technology and/or society.
  \item Describe the empowerment of new and high-potential actors towards future
    technological leadership.
  \end{compactitem}
}
Statistical mechanics has by itself had an enormous impact on technological and economic developments in the last century.  Its framework has not only inspired the logical framework of modeling and analysing microscopics to reach mesoscopic and macroscopic scales, but it has very specifically also enabled ab initio schemes and algorithms to reach and predict macroscopic behavior.  The foundations of solid state physics and quantum field theory that have reshaped the modern world and brought major changes in daily life invariably are consequences of the fluctuation theoretical ideas pioneered in statistical mechanics.  Yet most, if not all, of that is restricted to equilibrium statistical mechanics with its powerful and general Gibbs formalism.  While that formalism and the consequent algorithms and implementations remain a superb tool in technological research and innovation, the present project turns to nonequilibrium theory and phenomenology.  There we reach even much richer grounds, also connecting with biological systems and hence also medical applications.  The expected impact here will therefore add to the well-proven and established record of statistical mechanics in general, while opening new avenues in the following directions:\\

- new materials and material properties: we have in mind here extending the systematics and the solid theoretical base in the search of goog and useful biomaterials and meta-materials.  Transport theory beyond linear response and the emerging theory of statistical forces beyond equilibrium will allow systematic and algorithmic searches for new properties and materials.
- Biomedical processes and transport:  the cybernetics of active media and the steering of active particles in biological (living) environments is expected to revolutionize medical interventions and pharmacy.  Transport of drugs and signals in biomaterials is essential for new treatments.  That requires the understanding and modeling of interacting particle systems in nonequilibtium environments and the competence of realizing complex algorithms for transport.\\
- Biomechanical structures and motility: Similar to the previous point, we need to understand much better the structural and architectural aspects of living materials such as the cytoskeleton of a cell or in general the mechanical behavior of tissue.  That is essential also for understanding the motility of cells and for the control of transduction of signals to cell interior.  In our vision, basic nonequilibrium physics insights are crucial here, and complement the more biochemical and experimental research there.\\
- Quantum stability and coherence control: One of the major difficulties in upscaling the wonders of quantum mechanics is the loss of coherence when in contact with large thermal environments of when involving a great number of degrees of freedom.  A new idea is to bring the system in contact, not with an equilibrium theral environment, but with a driven nonequilibrium medium with well-defined and controlled properties that in fact would control the obtained coherence of immersed (quantum) devices.\\
- Transient and fast processes: A lot of industrial production relies on process that occur far from equilibrium.  Parameters like stresses and temperatures are changing very rapidly, and the products are subject to transient or very time-dependent environments. That can only be controlled more systematically from the point of view of nonequilibrium physics where response theory also for transient and nonequilibrium processes are in full development.\\
- Economy and society:  Econophysics, financial mathemtatics and socio-phyiscs all rely on models and insights of nonequilibrium physics. We expect that better insights in nonequilibrium processes can contribute to implement beneficial algorithms in economic or societal policy.  As one example, the idea of active and interacting agents that give rise to a collective macroscopic behavior, be it economic or political, is an interesting or even crucial addition to existing models if only for grasping nonequilibirum phenomena like clustering, flocking, heavy tails or non-additivity of forces.\\
- Climate science and weather prediction: It goes almost without saying that climate science and meteorolgy are fundamentally based on nonequilibrium physics as realized more specifically in fluid mechanics and atmosphere dynamics. Statistical modeling and understanding rare events are more and more central and most relevant issues in these. The behavior of probes in turbulent flow, the modeling of a global weather process and the setting up of reliable predictive capacity  are important research themes for some of today's world problems and nonequilibrium physics will obviously be a key-ingredient in any solution or advancement.\\ 
- Eco-science and sustainable energy: New ways of protecting the planet, keeping life interesting and have sustainable resources will rely on understanding non-dissipative aspects of physical processes.  Dissipation governs most of equilibrium physics, but beyond there are possibilities of self-organization and stabilization of structures and phases that are unstable or even non-existing in equilibrium. \\ 

In all of the above one recognizes the start of a major new ingredient in dealing with certain processes and technology.  The importance of the nonequilibrium paradigm and its transformational impact on science, technology and society are enormous and largely unexploited.  The present project wants to take that serious and start the far reaching way of implementing far from equilibrium physics. 

\subsection{Measures to maximise impact}

\paragraph{a) Dissemination and exploitation of results}

The major method of dissemination of our results is first the standard scientific practice of open access publishing, and participation in international conferences and discussion fora.  
The universities have important research and development centres where starting spin-offs find resources and support.
There will be important contacts with these centres. \\


Dissemination of the results is done by the free web archive, in publications in the standard specialized scientific journals (mostly non-commercial), by talks, by schooling and in contacts in conferences. As members of the Editorial Boards of JPhysA (IOP), of JMathPhys (APS), of the Journal of Statistical Physics (Springer), of Annales Henri Poincar\'e and of Fundamental Theories of Physics (Springer), we can organize special issues and create special volumes dedicated to aspects of the project. Indeed, as the project deals with some very novel aspect that has not appeared in standard or even not so standard treatments of irreversible thermodynamics, an important effort will be necessary and will be made to reach also less specialized researchers as well as scientists involved in possible applications.
Obviously an important cost then also goes to travelling and participation in conferences and international discussions. Again, since the project is not entirely mainstream and some concepts/questions are quite new, the project needs to invest in visibility and in numerous exchanges on international platforms. Standard recurring international conferences such as the Rutgers Statistical Mechanics Meeting, les Journ\'ees de Physique Statistique (Paris), MECO (Mid-European), the Statphys IUPAP meeting and the IAMathPhys-meeting will certainly be attended.  Key-world leading teams for the present project, certainly in Europe but also outside Europe including those in Japan and India will be frequented. We also hope to extend our network to groups in Beijing (e.g. Beijing computational center and the Biodynamic Optical Imaging Center of Peking University) and Singapore (IMS).  All team members will be expected to show great mobility.  Similar considerations apply for inviting scientists where costs include travelling, housing and subsistence. World-leading scientists will be invited to give intensive courses.  Those and other visiting experts will enable excellent collaborations.  We also foresee 20 seminars per year in Leuven directly related to the project, mostly from European scientists. They will of course be invited to stay a few days to make time for informal discussions. The project will also invest (still under other direct costs) in bringing together various points of view on the role of dynamical activity and the construction of nonequilibrium statistical mechanics. For that reason we foresee to organize two major conferences, one at around month 20, the other towards the end of the project.  In between, mini-workshops will be organized to discuss progress and challenges in smaller groups of experts: one at the beginning of the project, a second one when the project has gone through initial stages, and a final one towards the end to discuss its future. The costs should cover the travel of the expert participants and their presence at the conference site.


\eucommentary{
  \begin{compactitem}
  \item Provide a plan for disseminating and exploiting the project results. The
    plan, which should be proportionate to the scale of the project, should
    contain measures to be implemented both during and after the project.
  \item Explain how the proposed measures will help to achieve the expected
    impact of the project.
  \item Where relevant, include information on how the participants will manage
    the research data generated and/or collected during the project, in
    particular addressing the following issues\footnote{For further guidance on
      research data management, please refer to the H2020 Online Manual on the
      Participant Portal.}:
    \begin{compactitem}
    \item What types of data will the project generate/collect?  o What
      standards will be used?
    \item How will this data be exploited and/or shared/made accessible for
      verification and re-use?  If data cannot be made available, explain why.
    \item How will this data be curated and preserved?
    \end{compactitem}
 %      
    You will need an appropriate consortium agreement to manage (amongst other
    things) the ownership and access to key knowledge (IPR, data etc.). Where
    relevant, these will allow you, collectively and individually, to pursue
    market opportunities arising from the project's results.\\
%
    The appropriate structure of the consortium to support exploitation is
    addressed in section~3.3.
%
  \item Outline the strategy for knowledge management and protection. Include
    measures to provide open access (free on-line access, such as the 'green' or
    'gold' model) to peer- reviewed scientific publications which might result
    from the project.%
    \footnote{Open access must be granted to all scientific publications
      resulting from Horizon 2020 actions. Further guidance on open access is
      available in the H2020 Online Manual on the Participant Portal.}\\
%
    Open access publishing (also called 'gold' open access) means that an
    article is immediately provided in open access mode by the scientific
    publisher. The associated costs are usually shifted away from readers, and
    instead (for example) to the university or research institute to which the
    researcher is affiliated, or to the funding agency supporting the research.\\
%
    Self-archiving (also called 'green' open access) means that the published
    article or the final peer-reviewed manuscript is archived by the researcher
    - or a representative - in an online repository before, after or alongside
    its publication.  Access to this article is often - but not necessarily -
    delayed ('embargo period'), as some scientific publishers may wish to recoup
    their investment by selling subscriptions and charging pay-per-download/view
    fees during an exclusivity period.
%
  \end{compactitem}
}

\paragraph{b) Communication activities}

\eucommentary{
  \begin{compactitem}
  \item Describe the proposed communication measures for promoting the project
    and its findings during the period of the grant. Measures should be
    proportionate to the scale of the project, with clear objectives. They
    should be tailored to the needs of various audiences, including groups
    beyond the project's own community. Where relevant, include measures for
    public/societal engagement on issues related to the project.
  \end{compactitem}
}

Various members of the team have larger networks of communication both towards the general public (public outreach) as towards industrial players and groups of interest.  The first aim there is to transfer knowledge and tools, and to help in optimizing public interest. An important initiative will be the organization of ``physics meets industry days''
where each time during a week, specific problems of industry or economic activity will be presented.  They will be treated by students and experts towards helping to solve these problems, with direct feedback toward the industry of company.  Such initiatives exist already in some countries but will be started up in other European countries, and with an additional selection and expertise platform related to complex and nonequilibrium phenomena.



%%% Local Variables:
%%% mode: latex
%%% TeX-master: "proposal"
%%% End:
\newpage
\chapter{Ethical Issues}\label{chap:ethical}
\begin{todo}{from the proposal template}
  Describe any ethical issues that may arise in the project. In particular, you should
  explain the benefit and burden of the experiments and the effects it may have on the
  research subject. Identify the countries where research will be undertaken and which
  ethical committees and regulatory organisations will need to be approached during the
  life of the project.

  Include the Ethical issues table below.  If you indicate YES to any issue, please
  identify the pages in the proposal where this ethical issue is described. Answering
  'YES' to some of these boxes does not automatically lead to an ethical review1.  It
  enables the independent experts to decide if an ethical review is required. If you are
  sure that none of the issues apply to your proposal, simply tick the YES box in the last
  row.
\end{todo}

\begin{small}
\begin{tabular}{|p{1em}p{11cm}|l|l|}\hline
  \multicolumn{2}{|l|}{\cellcolor{lightgray}{\strut}} & 
  \cellcolor{lightgray}{YES} & 
  \cellcolor{lightgray}{PAGE}\\\hline 
  \multicolumn{2}{|l|}{\bf{Informed Consent}} & & \\\hline
  & Does the proposal involve children?  & & \\\hline
  & Does the proposal involve patients or persons not able to give consent? & & \\\hline
  & Does the proposal involve adult healthy volunteers? & & \\\hline
  & Does the proposal involve Human Genetic Material? & & \\\hline
  & Does the proposal involve Human biological samples? & & \\\hline
  & Does the proposal involve Human data collection? & & \\\hline
  \multicolumn{2}{|l|}{\bf{Research on Human embryo/foetus}}  & & \\\hline
  & Does the proposal involve Human Embryos? & & \\\hline
  & Does the proposal involve Human Foetal Tissue / Cells? & & \\\hline
  & Does the proposal involve Human Embryonic Stem Cells? & & \\\hline
  \multicolumn{2}{|l|}{\bf{Privacy}} & & \\\hline
  & Does the proposal involve processing of genetic information 
         or personal data (eg. health, sexual lifestyle, ethnicity, 
         political opinion, religious or philosophical conviction)  & & \\\hline 
  & Does the proposal involve tracking the location or observation 
         of people? & & \\\hline 
  \multicolumn{2}{|l|}{\bf{Research on Animals}} & & \\\hline 
  & Does the proposal involve research on animals? & & \\\hline 
  & Are those animals transgenic small laboratory animals? & & \\\hline 
  & Are those animals transgenic farm animals? & & \\\hline 
  & Are those animals cloned farm animals? & & \\\hline 
  & Are those animals non-human primates?  & & \\\hline 
  \multicolumn{2}{|l|}{\bf{Research Involving Developing Countries}} & & \\\hline 
  & Use of local resources (genetic, animal, plant etc) & & \\\hline 
  & Benefit to local community (capacity building 
         i.e. access to healthcare, education etc) & & \\\hline 
  \multicolumn{2}{|l|}{\bf{Dual Use}} & & \\\hline 
  & Research having direct military application  & & \\\hline 
  & Research having the potential for terrorist abuse & & \\\hline 
  \multicolumn{2}{|l|}{\bf{ICT Implants}} & & \\\hline 
  & Does the proposal involve clinical trials of ICT implants?  & & \\\hline 
  \multicolumn{2}{|l|}{\bf\footnotesize{I CONFIRM THAT NONE OF THE ABOVE ISSUES APPLY TO MY PROPOSAL}} 
      & &\cellcolor{lightgray}{} \\\hline 
\end{tabular}
\end{small}

\section{Personal Data}

\end{proposal}
\end{document}

%%% Local Variables: 
%%% mode: LaTeX
%%% TeX-master: t
%%% End: 

% LocalWords:  efo efoRM baz bazRM miko acrolong ntelligent iting pn pnlong
% LocalWords:  textsc newpage compactht texttt euproposal.cls callname callid
% LocalWords:  challengeid objectiveid outcomeid tableofcontents
