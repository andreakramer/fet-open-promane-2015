
\subsection{Targeted breakthrough, Long term vision and Objectives}\label{sec:objectives}

\eucommentary{
  \begin{compactitem}
  \item Describe the targeted scientific breakthrough of the project.
  \item Describe how the targeted breakthrough of the project contributes to a
    long-term vision for new technologies.
  \item Describe the specific objectives for the project, which should be clear,
    measurable, realistic and achievable within the duration of the project.
  \end{compactitem}
}

The project is aimed at exploiting and controlling the motion of probes and devices in
and through contact with nonequilibrium media.  The latter refers to an environment that is active or driven away from its
thermodynamic equilibrium.
%
This goal involves the transition to a third major stage of
nonequilibrium research. Until now, the focus has been on studying the approach to equilibrium and on the
derivation and description of dissipative transport of mass, energy and momentum between
various equilibrium reservoirs.
%
By deriving and using the precise relation between systematic forces, friction
and noise on probes, we open  a new and realistic class of dynamical systems for which the control lies in the nonequilibrium environment.
Complementary, the understanding of the fluctuation and response behavior of
nonequilibrium media allows us to extend the standard irreversible thermodynamics as it was conceived in 1930-1970.  Both, new control on dynamical systems and kinetic to thermodynamic behavior  are essential ingredients
in future technology and industrial applications.  To cite a few general far-reaching examples:  the controlled motion or transport of material through living tissue or in interaction with life processes, requires fundamental departures
from traditional transport theory; the stabilization of quantum coherence as needed for future quantum technology cannot be obtained within thermal environments but could be enhanced via nonequilibirum contacts; mobilities and conductivities close-to-equilibirum are subject to the fluctuation-dissipation relation but nonequilibrium driving can produce unseen transport and conductivity behavior for the realization of new materials.
%
The long term vision is thus that a systematic understanding of nonequilibrium response and
fluctuation behavior, beyond the traditional linear response regime, will uncover new
parameters to control technological processes, to steer motion and to stabilize matter.
%
One specific challenge is that such nonequilibrium parameters contain kinetic components (in addition to the usual
thermodynamic ones: chemical potential, pressure and temperature) and involve therefore the
operational understanding of what today is called dynamical activity, or to so called frenetic contribution.
%
That leads to two major objectives of this research proposal:
\begin{inparaenum}[A.]
\item the statistical mechanical study of probes under nonequilibrium conditions, including the statistical thermodynamics of active media;
\item the control theory for nonequilibrium media, including kinetic parameters as they contribute to the dynamical activity.
\end{inparaenum}

These objectives split up in further research goals as we will indicate below, but they are
also the pillars for revolutionizing ideas in technological processes. Clearly,
nonequilibria and contact with nonequilibria is everywhere in modern engineering and medical
applications.
%
But so far it has not been possible to rely on a solid scientific basis as is done for other
technologies since the industrial revolution, where equilibrium or near-equilibrium theories
are used.
%
Similarly, time-dependent phenomena at very long and even at very fast time scales, such as in glasses
or electronic devices, respectively, elude the usual equilibrium thermodynamic
approach. Besides the theoretical and experimental underpinning of the recent insights and progress in nonequilibrium science, part of the project will also realize numerical and algorithmic methods for efficient
simulation of such active processes, and theoretical advances will replace much of the
current trial and error methods. Specific objectives for the project then include:
\begin{inparaenum}[A.]
	\item Providing a scientific basis and new avenues for the realization of new properties of matter (in response and transport), and the
	enhanced stability of thermodynamic phases or behavior.  See further also in Work Package 2 for the conception of meta-materials. 
	\item The control of colloids in nonequilibrium baths and in nonlinear media.  See Work Package 3 for the theoretical and experimental study of probes in visco-elastic solvents.
	Most fluids found in industry and biology are visco-elastic and/or out of equilibrium.
	\item Kinetic control, see e.g. Work Package 4 below.  Many challenges are kinetic rather than thermodynamic, especially for processes and manipulations in nonequilibirum envrionments or for time-dependent parameters.  Irreversible thermodynamics vcannot deal with the overwhelming nature of kinetic control, but nonequilibrium statistical insights point to the control of dynamical activity; what some of us have called the frenetic contribution.  One objective is to make a frenometer for operational control and measurement of dynamical activity.
	That is essential for the control and steering of dynamics out-of equilibrium.  For example, polymer physics outside equilibrium will supply a necessary and complementary component to the vast domain of polymer research with thermodynamic control.
	\item Develop the theory of statistical forces on probes in contact with macroscopic nonequilibria, essential for controlling motion in turbulent media (atmosphere dynamics) or under biological flow (blood streams).
	Work Packages 5 and 6 will work with experimental and observational predictions of new emergent behavior for probes in nonequilibria, as resulting from the essentially nongradient, nonadditive and nonlocal features of these nonequilibrium statistical forces.
For example, we envision new separation techniques and methods to either improve or to avoid the clustering of particles, which is part of many technological challenges.
\end{inparaenum}
These objectives will find realizations in our existing contacts with colloidal and polymer scientists,
microbio-engineers and atmosphere researchers. Related topics are for example found in the
physics of polymer folding and the phase diagram of bio-polymers, the mechanical properties
of the cytoskeleton and fluctuations of bio-membranes, and the statistical analysis and
development of climate and weather models.

\begin{draft}
  From Klaus: Doesn't this sound a bit incoherent or eclectic? Could it be formulated in a
  way that makes the width of possible applications palpable but also makes the
  methodological coherence of our approach more apparent?
\end{draft}

\subsection{Relation to the work programme}\label{sec:relation-wp}

{\bf Section in extensive reformulation to fit Work Programme document}

We review here its appropriate character of \TheProject in light of the FET-Open objectives:
\begin{compactdesc}
\item[Long-term vision] Current technologies rely on an understanding of thermodynamics that
dates from half a century ago. The proposed update of the set of tools available to
scientists and to the industry, on the basis of nonequilibrium thermodynamics, is the first
step in a change of paradigm across many technological fields.
\item[Breakthrough S\&T target] The developments within \TheProject rely directly on
technological applications, either via relevant physical models or via the experimental
partners (\site{USTUTT} and \site{ULEI}). Nonequilibrium theories are currently not used in
the industry and we will provide the building blocks to enable it. We expect that the
developement of nonequilibrium theories will deliver a competitive advantage, similar in
principle to the one delivered by James Watt developments for the steam engine.
%
A critical part of this advantage arises from the development of control strategies that
will deliver concrete guidance XXX.
\item[Foundational] TODO
\item[Novelty] The changes that we are proposing do not aim at a minor modification of the
existing processes but at replacing them with a better understanding and, more specifically,
new strategies for diagnosis and control.
\item[High risk] TODO
\item[Interdisciplinarity] Since its invention, the field of thermodynamics has extended its
scope from the study of steam engines to being used in most of the physical sciences and
engineering disciplines: chemical reactions, materials sciences, biological systems
(e.g. heat regulation or food chain), information theory, etc. The progresses we envision in
\TheProject are firmly based on physical models and experiments but domains of go far beyond.
\end{compactdesc}

For this FET-open research programme we set as task to provide theoretical and experimental ground for dealing with and steering probes in active, driven and nonlinear media.
Many aspects of technology, whether in production, optimization or automation, require understanding and controlling nonequilibrium processes.  That includes both the traditional engineering processes, but also and more than ever, food processing, pharmaceutical, biochemical and biomedical applications.  Scientific knowledge is the only and proven road to success in these applications and ranks far above, high through-put or trial and error based methods.  Moreover, having today a choice of so many materials or components, and having also available huge data sets can leave the applied scientist paralysed in front of the many of today's technological and societal challenges; tradition and experience built over the years alone do not suffice.  Solid scientific and high-level understanding are the keys to further important developments for sustaining or improving technological levels.  The present project emphasizes the nonequilibrium aspects, complementing and moving beyond the great success of equilibrium statistical mechanics and irreversible thermodynamics in obtaining and revolutionizing technology for the last 100 years.
The present proposal has as major challenge to control and steer devices or materials in contact with and through nonequilibrium and nonlinear environments.  That includes a great number of possible scenarios, including,\\
a) new materials and material properties:  can one obtain interesting transport coefficients and use them in new situations by contact with rightly monitored nonequilibria.  What are the thermal conductivities or mobilities for nonequilibrium materials?  What are the differential equations of relaxation to nonequilibrium stationary conditions?\\
b) Biomedical processes and transport: how to control transport of particles in nonequilibrium flows or through living tissues?  How are chemical nonequilibria transduced as signals for mechanical changes? Are self-organized patterns and morphognesis obtained from micriscopic nonequilibrium descriptions?\\
c) Biomechanical structures and motility: how do the mechanical properties change of materials undergoing active processes?  What are the shapes and stresses of bio-materials?\\
d) Quantum stability and coherence control: can one obtain enhanced stability of quantum structures and coherence by providing nonequilibrium control?\\
e) Transient and fast processes: Is there a nonequilibrium thermodynamic treatment of quenched dynamics and of ultraslow and ultrafast processes?  What is the validity there of the concept of effective temperature and can one avoid kinetic constraints (with associated many-body localization, jamming, clustering...) via appropriate nonequilibrium control?\\
f) Economy and society: are there analogues to the nonequilibrium notions of entropy production and dynamical activity in at least theoretical models of economic markets or society changes?  Is the volatility predictable ?  How can nonequilibrium statistical physics really contribute in the climate science debate, or in the development of technologies with  sustainable energy.\\

The work programme will be systematized through six work packages, each time with one node being the main responsible, leader and spokes person for one of them.\\
We list them already here with some key-words referring to the scientific ambitions:\\
WP1 Eindhoven (Prague, Leuven): Nonequilibrium thermodynamics.  Applied mathematics and the derivation of analytical tools for relaxational behavior.\\
WP2 Prague (Leuven): Quantum transport processes and restoration of coherence in the presence of nonequilibrium environments.\\
WP3 Padova (Leipzig): Polymer processes in nonequilibrium media.\\
WP4 Leipzig, (Stuttgart):  Active media, non-additive and non-gradient forces\\
WP5 Stuttgart (Padova): Micro-rheology, fluid mechanics, active swimmers\\
WP6 Leuven (Prague): control theory, nonequilibrium statistical forces.\\

\subsection{Novelty, level of ambition and foundational character}\label{sec:progress}

\eucommentary{
  \begin{compactitem}
  \item Describe the advance your proposal would provide beyond the
    state-of-the-art, and to what extent the proposed work is ambitious, novel
    and of a foundational nature. Your answer could refer to the ground-breaking
    nature of the objectives, concepts involved, issues and problems to be
    addressed, and approaches and methods to be used.
  \end{compactitem}
}

There are three major aspects of novelty and ambition. Our work will be foundational and ground-breaking in 1) bringing forward the role of non-dissipative aspects in nonequilibrium physics, and 2) its study of the dynamics of probes in nonequilibrium environments: 
%
{\bf ad 1)} Now is the time to launch a decisive study on whether indeed the dynamical
activity complements in essential ways entropic considerations in the construction of
nonequilibrium statistical mechanics. We propose therefore to devote a major and markedly
innovative effort to unravel the conceptual, computational and operational meaning of the
time-symmetric fluctuation sector, and how it enters observationally in the response of open
systems driven away from equilibrium.  The main objective
then is to further develop fluctuation-response theory to systematize investigations on the
operational-thermodynamic role and the statistical-mathematical nature of nonequilibrium
kinetics for the elucidation of steady states beyond close-to-equilibrium regimes. The
impact of a valid fluctuation-response theory away from equilibrium will be important in the
more general context of complex systems and dissipative dynamics.
%
A major part of theoretical and mathematical nonequilibrium studies of the last 150 years
have been devoted to the relaxation to and the response in equilibrium. Where steady regimes
further away from equilibrium are considered, the emphasis has remained on entropic
considerations, either in the study of entropy production (as dissipated heat) or from the
point of view of information-theoretic aspects.  Since some two decades nonequilibrium
studies have revisited steady regimes further away from equilibrium, and a
fluctuation-response theory is starting to emerge. Yet most results concern specific (often
toy-)models, be it stochastic or highly chaotic. In general the broader theoretical ideas
concentrate on entropy production as inherited from irreversible thermodynamics. Where
time-symmetric quantities have entered, the understanding remains rather formal.  The proposal chooses specific questions, some of which are mathematically very challenging, in fluctuation–response theory to open the meaning of dynamical activity, and to make it operational. The latter means to search for specific observational consequences and experimental control.\\
%
{\bf ad 2)} We want to set up a fluctuation and response theory for nonequilibrium media
from their influence on probe dynamics, and we want to be able to control and monitor a
system’s behavior from its contact with a nonequilibrium medium. The point is to explore the
possible very new behaviour and properties of systems in contact with nonequilibrium
media. A first question one can ask about a probe in active contact with a nonequilibrium
sea is about the induced systematic statistical forces of the sea on the probe.  One starts
from the more microscopic mechanical force and, assuming that the probe changes its state
(e.g. position) on a much slower timescale than the medium, one integrates that mechanical
force over the degrees of freedom of the stationary medium.  In equilibrium (that is, when
the probe is in contact with an equilibrium reservoir) the resulting (statistical) forces
are of gradient type.  The standard thermodynamic potentials are (indeed) the potentials
from which the force can be derived as in Newtonian mechanics. The result there is that it
allows us to draw free energy landscapes to understand the changes in the system, as if it
concerned a conservative mechanical system for which we give the potential energy.  A
special interesting example are the entropic forces, which work by the power of large
numbers; then, the free energy has negligible energetic or enthalpic contributions.  These
are known to be important in elasticity and polymer physics. Moreover, always for
equilibrium, the statistical forces are additive for example in the sense that bulk
contributions can easily be separated from boundary contributions.  The latter is crucial
for thermodynamic behaviour.  For example, the very possibility of thermodynamic behaviour,
the presence of an equation of state etc depend on that distinction.  All of that need not
and in general, will not,
be true for probes (systems, walls, collective coordinates) in contact with (genuine)
nonequilibrium media or reservoirs.  We have added the word “genuine” to indicate that we
leave open the possibility that there are exceptional situations, for example in terms of
specific models or special regimes where additive behaviour of the statistical forces would
be restored.  We know that the close-to-equilibrium static large deviation functionals
remain local and additive, so that it is fair to expect there the more usual thermodynamics
compatible with e.g. 0th law behaviour or with the Clausius heat theorem that yields a
calorimetric meaning to the fluctuation functionals close-to-equilibrium.  Yet, in general,
and as generic as there are long range effects in nonequilibria we expect a breaking of
additivity and of the gradient nature of statistical forces. Small perturbations in the
sense of changes in boundary conditions or in changes of external potential can drastically
change the stationary distribution also in the bulk of the system.  That has possibly
interesting consequences in the manipulation of such forces, adding oscillatory components,
going from attractive to repulsive and obtaining nonvanishing resulting forces that
otherwise, in equilibrium, would be zero from mutually canceling contributions.
%
It also means that the stationary positions of the probe (quasi-static) would be at
different locations compared with equilibrium, allowing possibly increased stability of
phases that would otherwise (in equilibrium again) be unstable.  The paradigmatic example is
here the Kapitza oscillator (pendulum) which remains stable upright when being shaken.  That
example is however likely to be systematized also outside the theory of dynamical systems,
reaching then in this project the stability of macroscopic behaviour of the probe
(collective coordinate) in for equilibrium very unlikely phases and values for order
parameters.  Needless to add here that this may have dramatic consequences on the phase
diagram for systems in contact with nonequilibrium media, not only breaking the Gibbs phase
rule but also introducing new phases of matter.  That is certainly one of the most exciting
possibilities of explorations in the present project. As another theme of stability we want
to understand some mechanism of homeostasis. In our set-up we treat spatially extended
systems with time-dependent boundary driving. We will see under what conditions the bulk of
the system reaches a steady (time-independent) regime. A related consequence of contact with
nonequilibrium reservoirs is the possibility of population inversion in the system, which
will mean that the effective temperature of the probe could be much larger for certain
purposes.  There will for sure not be the usual Einstein or second fluctuation-dissipation
relation between noise and friction on the probe which also entails that the friction
coefficient can show further nontrivial behaviour.  The phenomena of shear thinning and
possibly shear thickening could very well be related to these.
%


\subsection{Research methods}\label{sec:methods}
\eucommentary{
  \begin{compactitem}
  \item Describe the overall research approach, the methodology and explain its
    relevance to the objectives.
  \item Where relevant, describe how sex and/or gender analysis is taken into
    account in the project's content.
  \end{compactitem}
}

The overall research approach is that of modern physics, with its traditional ways of open exchange of ideas and results.  There are fundamentally two components, which are deeply connected: the theoretical and experimental side.
We certainly must work to keep up and even improve the exchanges between these efforts and to bring about and foster multiple exchanges of ideas and questions.  The way from theory to experiment is also the method to reach more applied science and eventually to influence industrial and economic research centers and technology.  We have been asked before in consulting with industrial projects (e.g. water transfer in porous media, new challenges for construction physics, crack evolution and creep in glassy materials etc) but the present project will allow systematic developments in contacting and helping industrial players.
For the more daily research methods, the basic tools are stochastic processes, analysis and computer programming.  For the latter molecular dynamics simulations will play a very big role, also because they reach more realistic scenario's that can be implemented and further verified in experimental work and tests.  The experimental side is much based on methods in fluid mechanics and optical control.\\

1. Mathematical methods: Here we mostly follow stochastic calculus and the specialized probabilistic techniques of dealing with spatially extended stochastic dynamics of interacting particle systems. Limit theorems and large deviation theory are central tools.\\
2. Theoretical framework:  Much emphasis remains on path-integral techniques in dynamical ensembles.  Such a Lagrangian statistical mechanics takes up the entropy fluxes and the changes in DA to build the action functional for dynamical ensembles.  The nonequilibrium techniques will be essentially non-perturbative in the driving, and keeping a distance from the Keldysh formalisms.\\
3. Computational support: For various model systems that are not covered by theorems and analytic results, we use numerical methods and large scale simulation.\\
4. Experimental work: Here we mostly undertake experiments on active materials and particles in the appropriate soft condensed matter labs.  Other questions relate more to biophysics and still others more to gas reactivity experiments.\\

Overall, this project will also deliver methodology and development tools, such as
\begin{inparaenum}[A.]
	\item the making available strong molecular dynamics simulation codes for the behavior of
	probes in nonequilibrium media and for the dynamics of active particles;
	\item the development of operational tools for measuring and quantifying kinetic parameters
	as used in the theory of dynamical ensembles, including the measurement of excess
	thermodynamic quantities and dynamical activities;
	\item providing a toolkit for {\it ab initio} calculations for macroscopic nonequilibrium
	behavior, possibly allowing new stable phases, including a quantitative description of
	(also) not-dissipative relaxation to stationary and steady conditions.
\end{inparaenum}


\subsection{Interdisciplinary nature}\label{sec:interdisc}

\eucommentary{
  \begin{compactitem}
  \item Describe the research disciplines involved and the added value of the inter-disciplinarity.
  \end{compactitem}
}
Statistical and mathematical physics are by their nature rather inter-disciplinary.  Not only is there very often a large intersection with mathematics, statistics and computer science, but also the subjects are often very diverse and reach out in many various directions.  For the present project where we emphasize the nonequilibrium nature of phenomena and processes, there is the immediate contact with condensed matter physics, with fluid mechanics and with biological and medical/engineering sciences.  That is quite obvious as nonequilibria are ubiquitous.  Discovering common grounds of study and constructing for them a nonequilibrium statistical mechanics is exactly the heart of the present programme.\\
Research disciplines that are involved are very diverse. The most amazing examples of nonequilibrium are probably found in life processes, or perhaps in the origin of life itself.  There we see an open system full of transport processes, with little engines, pumps and cycles and the emergence of order on diverse spatio-temporal scales out of molecular complexity. The variety of phenomena where nonequilibrium considerations are essential is however much larger.  We find them at cosmological scales of the observable universe  and beyond the smallest sizes of nanotechnology.  Subatomic processes, the creation and annihilation of particles and fluctuations at the smallest dimensions produce sources of noise and matter, from which our world is finally made.  The second law of thermodynamics, the increase of entropy for closed systems, puts fundamental limitations on all models of the universe.  Various scenario's for the early universe  mimic transitions of metastable states to more   stable states, via
 nucleation processes as we see them also  in condensed matter.  On more earthly scales, evolution and the fight for low entropy create cycles of life and destruction that we recognize in all of nature.    Turbulent
  flow and nonlinear systems give other macroscopic realizations of nonequilibrium effects.   Climatology and ecology ask questions about the atmosphere and ocean dynamics and about food web chains as  nonequilibrium systems.
  In biology, molecular motors   give mesoscopic realizations of chemical engines and of transport on the molecular scales.   The cell itself, how it moves and how its membrane fluctuates on $\mu m$-scales, brings nonequilibrium statistical mechanics to life.   Chemical reactions are traditionally also sources of nonequilibrium changes.
 Further down, nanophysics studies processes of dissipation and transport at still smaller scales.




%%% Local Variables:
%%% mode: latex
%%% TeX-master: "proposal"
%%% End:
