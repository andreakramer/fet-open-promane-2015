\subsection{Targeted breakthrough, Long term vision and Objectives}\label{sec:objectives}

\eucommentary{
  \begin{compactitem}
  \item Describe the targeted scientific breakthrough of the project.
  \item Describe how the targeted breakthrough of the project contributes to a
    long-term vision for new technologies.
  \item Describe the specific objectives for the project, which should be clear,
    measurable, realistic and achievable within the duration of the project.
  \end{compactitem}
}

\paragraph{Scientific and technological goals}

The project is aimed at exploiting and controlling the motion of probes and devices in and
through contact with nonequilibrium media. ``Nonequilibrium media'' refers to an environment
(solvent, bulk of a material, etc) that is active or that is driven away from its
thermodynamic equilibrium.
%
The {\bf scientific
  breakthrough} involves an operational formulation of nonequilibrium statistical mechanics.
That goal implies the transition to a third major stage of {\bf fundamental nonequilibrium
  research}. Most often in the past, the focus has been either on studying the approach to equilibrium or on
the derivation and description of dissipative transport of mass, energy and momentum between
various equilibrium reservoirs.  By deriving and using the precise relation between
systematic forces, friction, and noise on probes, we open a new and realistic class of
dynamical systems for which the control lies in the nonequilibrium environment.
Complementarily, the understanding of the fluctuation and response behaviour of nonequilibrium
media allows us to extend the standard irreversible thermodynamic theory as it was conceived in
1930--1970.

Both, new control of emerging dynamical systems as probes connected to nonequilibria and the manipulation of kinetic to thermodynamic behaviour of media are essential
ingredients in future technology and industrial applications.  To cite a few general
far-reaching objectives: the controlled motion or transport of material through living
tissue or in interaction with life processes requires fundamental departures from
traditional transport theory; the stabilization of coherence as needed for future quantum
technology cannot be obtained within thermal environments but could be enhanced via
nonequilibrium contacts; mobilities and conductivities close-to-equilibrium are subject to
the fluctuation-dissipation relation but nonequilibrium driving can produce unseen transport
and conductivity behavior for the realization of new materials. 


 As is central to the goals
of the FET-open initiative, theoretical and experimental foundational work will lead to
create abilities and conditions under which technological innovation becomes possible.  For
example, serious conceptual problems need to be solved before active-particle suspension
will become a common basis for enhanced targeted catalysis or for fluid mixing and viscosity
reduction in industrial processing, or for noninvasive surgeries and drug delivery in
medicine, or for the design of smart materials with built-in actuation mechanisms. These
open conceptual issues are addressed via several pathways in the theoretical parts of the
proposal.  Leaping into macroscopic nonequilibria moreover implies that classical
measurement techniques, developed for equilibrium systems, will generally fail to work as
usual and will have to be replaced by new designs. Brownian thermometry provides an example
that is relevant to a variety of applications \cite{kroy:2014}, and which will be
generalized to conditions far from equilibrium, in the project.  On the practical and
applied side, radically new manipulation techniques are required to fully exploit the
innovative potential provided by active particles. As pertinent examples, we mention the
thermophoretic trap and the photon-nudging method pioneered by the experimental group in
Leipzig \cite{Qian2013,Braun:NanoLetters:2015}.  Both techniques steer particles (passive
and active particles in the trap and hot active particles in the nudging technique,
respectively) by exploiting far-from equilibrium conditions in the solvent. Notably, they
work without imposing external forces, via directly addressing the nonequilibrium solvent
conditions and thereby the (self-)thermophoretic propulsion (or ''swimming'') of the
particles. Their usefulness is enhanced by employing Maxwell-demon type control
strategies. Prototypes of these genuinely nonequilibrium techniques exist and will be
employed and refined (e.g. by optimizing dedicated control strategies) within the project.
One should realize that the assumption of a thermally equilibrated bath is not fulfilled in
most industrially (oil, polymer melts, dense colloidal or micellar suspensions) or
biologically relevant (blood, mucus, DNA-solutions) liquids which display visco-elastic
properties, i.e. elastic behaviour like a solid and viscous behavior like a fluid. As a
consequence, such fluids exhibit large relaxation times (up to seconds), and can thus easily
be driven out of thermal equilibrium by a forced colloidal probe. Within this proposal, we
want to study how the local perturbation of visco-elastic systems affects the dynamics and
effective interactions of colloidal particles in such media with particular view on
e.g. memory effects, many body interactions and local non-linear rheological properties.
Similarly, time-dependent phenomena at very long and even at very fast time scales, such as
in glasses or electronic devices, elude the usual equilibrium thermodynamic
approach.  Such studies will in general reveal the properties of nonequilibrium baths which
are of manifold importance.

More generally and as part of the {\bf long term vision} we believe that a systematic
understanding of nonequilibrium response and fluctuation behavior, beyond the traditional
linear response regime, will uncover new parameters to control technological processes, to
steer motion, and to explore and stabilize new interesting material properties.

\paragraph{Objectives}

can be divided into two major classes:\newline
\begin{inparaenum}[A.]
\item identifying the key-control and manipulation parameters for probes and devices in contact with nonequilibrium media.  That means to understand the emerging dynamical system immersed in nonequilibria.  It will lead to improved transport quality in environments such as the turbulent atmosphere, in active media or in nonlinear fluids.  Besides transport, we expect also the stabilization and the emergence of interesting device properties with long time coherence and classical dia- and paramagnetic phases as a couple of important examples;\newline
\item shaping nonequilibrium media for their response behavior.  The addition of kinetic and nonequilibrium parameters in material design can lead to absolutely new effects in response or susceptibility. Negative compressibilities, thermal or variable conductivities introduce new possibilities and applications for flexible and smart materials.  We provide a new computational framework for {\it ab initio} calculations of nonequilibrium phase diagrams, opening an important tool in material and device production.
\end{inparaenum}


We will provide modelling and simulation techniques for the simulation of probes and
devices in contact with nonequilibria.  These are high level molecular dynamics codes for
simulating and calculating the emergent behavior of contacts and probes.  These techniques
will be made available through all  work packages. It is an important objective
to make that tool also ready to future industrial partners  who are challenged by the complexity of
dealing with nonequilibria.


We will offer a scientific basis and new avenues for the realization of new properties of
matter (in response and transport), and the enhanced stability of thermodynamic phases or
behaviour.  See further also in \WPref{WPcompress} for the conception of
meta-materials. Specifically we have good understanding for workable implementations of
negative compressibilities.


For the control of colloidal motion in nonequilibrium baths and in nonlinear media,  \WPref{WPbrownian}, we use theoretical and experimental studies of probes in visco-elastic
solvents.  Most fluids found in industry and biology are visco-elastic and/or out of
equilibrium.  Control of shear, the development of new separation techniques based on
dynamical activities and steering of clustering properties are achievable in the current
project.


Many challenges are kinetic rather
than thermodynamic, especially for processes and manipulations in nonequilibrium
environments or for time-dependent parameters, \WPref{WPactive}.  Irreversible thermodynamics cannot deal with
the overwhelming nature of kinetic control, but nonequilibrium statistical insights point to
the control of dynamical activity; what some of us have called the frenetic contribution.
One objective is to make a frenometer for operational control and measurement of dynamical
activity.  That is essential for the control and steering of dynamics out-of equilibrium.
For example, polymer physics outside equilibrium will supply a necessary and complementary
component to the vast domain of polymer research with thermodynamic control.


We develop the theory of statistical forces on probes in contact with macroscopic
nonequilibria, essential for controlling motion in turbulent media (atmosphere dynamics) or
under biological flow (blood streams).  \WPref{WPdissipation} and \WPref{WPcore} will work towards experimental
and observational predictions of new emergent behavior for probes in nonequilibria, as
resulting from the essentially nongradient, nonadditive and nonlocal features of these
nonequilibrium statistical forces.


These objectives find further realizations in our existing contacts with colloidal and
polymer scientists, micro-electronic and microbio-engineers and atmosphere researchers. {\bf The overarching goal
remains to make available the science of nonequilibria to technological and societal
developments.}



\subsection{Relation to the work programme}\label{sec:relation-wp}

\begin{compactdesc}
\item[Long-term vision] \TheProject enables a change of paradigm across many technological
fields on the basis of nonequilibrium thermodynamics. Current technologies rely on an
understanding of thermodynamics that dates from half a century ago and we will deliver the
set of tools to scientists and to the industry that is missing.
\item[Breakthrough S\&T target] The developments within \TheProject rely directly on
technological applications, either via relevant physical models or via the experimental
partners (\site{USTUTT} and \site{ULEI}). Advanced or novel nonequilibrium theories are
currently not used in the industry and we will provide the building blocks to enable them
for the development of new material (\WPref{WPcompress}), ultra fast processes (see XXX)
and control strategies (\WPref{WPdissipation}) and viscoelastic media (\WPref{WPbrown}).
\item[Novelty] The changes that we are proposing do not aim at a minor modification of the
existing processes but at replacing them with a better understanding and, more specifically,
new strategies for diagnosis and control.
\item[Foundational] Our work will have a decisive impact on other fields in which the
control of devices lacks a comprehensive thermodynamical understanding: medicine and
pharmacology (control of nanoparticles in living systems and targeted delivery mechanisms),
engineering (fast industrial processes, see XXX).
\item[High risk] Scientific research carries risks by its very nature, as we are pushing the
frontiers of knowledge. It may be the case that our ideas stumble on mathematical or
experimental challenges that would require efforts beyond the 4 yeards of the
project. Still, the combined expertise in \TheProject offers strong chances of success and
even before completion many exciting outputs are foreseen (see XXX).
\item[Interdisciplinarity] \TheProject consists in two experimental physics groups
(\site{ULEI}, \site{USTUTT}), one mathematical group (\site{TUE}), two theoretical physics
groups (\site{KUL}, \site{UNIPD}), one of which specialized in simulation methods
(\site{UNIPD}). This combination allows a consistent organization of the work across work
packages to deliver results ranging from the conceptual to the proof of principle.
\end{compactdesc}

\subsection{Novelty, level of ambition and foundational character}\label{sec:progress}

\eucommentary{
  \begin{compactitem}
  \item Describe the advance your proposal would provide beyond the
    state-of-the-art, and to what extent the proposed work is ambitious, novel
    and of a foundational nature. Your answer could refer to the ground-breaking
    nature of the objectives, concepts involved, issues and problems to be
    addressed, and approaches and methods to be used.
  \end{compactitem}
}

The first stage of modern thermodynamics consisted in microscopic studies of equilibrium systems, giving rise to equilibrium statistical mechanics. The current body of knowledge concerning nonequilibrium
physics represents, as we have outlined in Sec.~\ref{sec:objectives}, the {\em second stage}
of the domain: the study of transport and response theory close to
equilibrium.

Since about two decades, nonequilibrium studies have revisited steady regimes further away
from equilibrium, and a fluctuation-response theory is  emerging.
The recent idea of dynamical activity starts a new line of conceptual
understanding for today's nonequilibrium thermodynamics. The focus on
entropy production has brought many insights in the last 150 years (in physics and other
fields as well, such as information theory) but is however restrictive.  We emphasize the role of excess thermodynamic quantities in relaxation to {\it non}equilibria, and we find that a crucial role is also being played by time-symmetric fluctuations.
Volatilities or time-symmetric traffic, frenesy and other names have been given to this dynamical activity which is all important  for a valid fluctuation-response theory away from equilibrium.
 Our current
understanding of time-symmetric quantities remains rather formal and most results concern
specific systems (often ``toy models'') , be it of stochastic or highly chaotic nature.
%
\TheProject chooses specific questions, some mathematically very challenging and some very
pragmatic, in fluctuation-response theory to open the meaning of dynamical activity, and to
make it operational.
%
Explicitly, we focus on consequences that can be observed and measured, and on the
experimental control that this theory brings.

The operational character of our work depends on the numerical implementation of physical
models and the experimental realization. In both cases, we implement the control and
monitoring of a system via a nonequilibrium medium, thereby replacing direct feedback
systems for the control by a monitoring of the kinetic and thermodynamic aspects of
nonequilibrium baths.
%
A first question is about the behaviour of probes in active contact with a nonequilibrium
sea; what are the induced systematic statistical forces?
%
In equilibrium (that is, when the probe is in contact with an equilibrium reservoir) the
resulting statistical forces can be derived from a gradient, reminding what is done in
Newtonian mechanics.
%
The result there is that it allows us to draw free energy landscapes to understand the
changes in the system, as if it concerned a conservative mechanical system for which we give
the potential energy.
%
Moreover, always for equilibrium, the statistical forces are additive. For instance, bulk
contributions can easily be separated from boundary contributions. The very possibility of
thermodynamic behaviour, like the presence of an equation of state, depends on that
distinction.
%
All of that need not
and in general, will not, be true for probes (systems, walls, collective coordinates) in
contact with (genuine) nonequilibrium media or reservoirs. That has interesting
consequences in the manipulation of such forces, adding oscillatory components, going from
attractive to repulsive and obtaining nonvanishing resulting forces that otherwise, in
equilibrium, would be zero from mutually canceling contributions.  
%
It also means that the stationary positions of the probe can be at different locations
compared with equilibrium, possibly increasing the stability of phases that would otherwise
(in equilibrium again) be unstable.

The Kapitza oscillator is a pendulum that remains stable upright when being shaken. This
example is a rather simple dynamical system and we aim in \TheProject at a systematic
extension of this idea for the stability of macroscopic phases that are very unlikely in
equilibrium.
%
Needless to add here that this may have dramatic consequences on the
phase diagram for systems in contact with nonequilibrium media, not only breaking the Gibbs
phase rule but also introducing new phases of matter.  That is certainly one of the most
exciting possibilities of explorations in the present project. How are for example effects
like homeostasis and adaptation linked to the nonequilibrium features of the medium and its
contact with the device?  A related consequence of contact with nonequilibrium reservoirs is
the possibility of population inversion, which can imply that the effective temperature of
the probe for example is much larger.


\subsection{Research methods}\label{sec:methods}
\eucommentary{
  \begin{compactitem}
  \item Describe the overall research approach, the methodology and explain its
    relevance to the objectives.
  \item Where relevant, describe how sex and/or gender analysis is taken into
    account in the project's content.
  \end{compactitem}
}

The fundamentally connected components in this \TheProject have mathematical, conceptual,
computational and experimental sides. The dependent character of these parts is outlined in
the chronology of the WPs in Sec.~\ref{sec:wp}.

\begin{asparaenum}
\item {\bf Mathematical methods:} We use stochastic calculus for describing the model
systems, including their spatially extended and many-body characters.
%
\site{TUE} and \site{KUL} are recognized experts on applied mathematics and
mathematical physics.
\item {\bf Theoretical and conceptual framework:} The proper understanding of thermodynamics
relies in a consistent definition of the entropy fluxes and of the dynamical activity.
%
We will develop an operationally oriented approach to nonequilibrium problems.
%
All nodes will participate in this effort.
\item {\bf Computational support:} Numerical simulations will serve as practical
implementation of the theoretical model systems and as a supporting tool for the
experiments.
%
They allow us to obtain the most complete description of the physical processes and test our ideas beyond what is analytically possible.
%
The computational developments are led by \site{KUL} and \site{UNIPD}.
%
For the more realistic scenaries (i.e. for modeling the experiments), Molecular Dynamics
simulations will be used.
\item {\bf Experimental work:} The soft condensed and liquid matter laboratories
(\site{ULEI}, \site{USTUTT}) are familiar with the techniques needed for active materials
and particles.
%
They are experts on the optical control and analysis of colloidal system and on the physical interpretations of the outcomes.
\end{asparaenum}

\subsection{Interdisciplinary nature}\label{sec:interdisc}

\eucommentary{
  \begin{compactitem}
  \item Describe the research disciplines involved and the added value of the inter-disciplinarity.
  \end{compactitem}
}

The consortium joins experts in applied mathematics, mathematical physics, statistical mechanics, polymer science, computational physics, liquid and soft condensed matter physics and fluid mechanics.
The general theme of nonequilibrium is indeed very  large, and amazing examples of nonequilibria span a great variety of phenomena ranging from nano-technology, over biology to atmosphere physics.
The researchers involved in the present project have in their own work contributed to the understanding of these various nonequilibrium sides.  Leipzig and Padova nodes have many contributions in biophysics, while Stuttgart is famous for colloidal physics, liquid matter experiments and the study of Casimir forces.  Leuven and Eindhoven nodes represent the more mathematical corners, with emphasis on analytical and stochastic methods.  Still other expertise (like in Padova and Leuven) is present in solving computational and simulation problems.   At the same time, the various labs have closer contacts, both scientific and more industrial, in a variety of disciplines, like sensor-technology, water transfer in porous media, polymer chemistry, financial mathematics, weather prediction and food science to mention a few.  Also the relation with students and young researchers spans different faculties, with science projects in engineering faculties, in institutes for cultural studies, and of course in computer science, chemistry and physics departments.



%%% Local Variables:
%%% mode: latex
%%% TeX-master: "proposal"
%%% End:
