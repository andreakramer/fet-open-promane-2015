
Body of section ``Excellence''.

\subsection{Targeted breakthrough, Long term vision and Objectives}\label{sec:objectives}

\eucommentary{
  \begin{compactitem}
  \item Describe the targeted scientific breakthrough of the project.
  \item Describe how the targeted breakthrough of the project contributes to a
    long-term vision for new technologies.
  \item Describe the specific objectives for the project, which should be clear,
    measurable, realistic and achievable within the duration of the project.
  \end{compactitem}
}

The project is aimed at studying the statistical dynamics induced by contact with nonequilibrium media thus entering a major third stage of nonequilibrium research, earlier ones being the problem of approach to equilibrium and the dissipative transport of mass, energy and momentum between various equilibrium reservoirs.  The goal is to derive the precise connection between systematic force, friction and noise on probes and the fluctuation and response behavior of nonequilibrium baths.  The newly emerging physics for systems in active contact with nonequilibria is both important for theoretical advances in steady state thermodynamics, such as for discovering the nature and role of nonequilibrium free energies, for understanding the behaviour of particles in nonequilibrium environments such as in life and atmospheric processes, as possibly for interesting implementations in materials science and for obtaining stability of otherwise (under equilibrium conditions) unstable macroscopic conditions and phases. 



\subsection{Relation to the work programme}\label{sec:relation-wp}

\eucommentary{
  \begin{compactitem}
  \item Indicate the work programme topic to which your proposal relates, and
    explain how your proposal addresses the specific challenge and scope of that
    topic, as set out in the work programme.
  \end{compactitem}
}

Many aspects of technology, whether in production, optimization of automation, require understanding and controlling nonequilibrium processes.  That includes both the traditional engineering processes, but also and more than ever, food processing, biochemical and biomedical applications.  Scientific knowledge is the only and proven road to success in these applications and ranks far above, high throug-put or trial and error based methods.  Moreover, having today a choice of so many materials, so big data  and unseen challenges, tradition and experience alone do not suffice.  Solid science and high-level understanding are the keys to further important developments for sustaining or improving technological levels.  The present project emphasizes the nonequilibrium aspects, complementing and moving beyond the great success of equilibrium statistical mechanics for obtaining and revolutionizing technology for the last 100 years.
The present proposal has as major challenge to control and steer devices or materials in contact with and through nonequilibrium environments.  That includes a great number of possible scenarios, including:
\begin{inparaenum}[(a)]
\item new materials and material properties
\item Biomedical processes and transport
\item Biomechanical structures and motility
\item Quantum stability and coherence control
\item Transient and fast processes
\item Economy and society
\item Climate science and weather prediction
\item Eco-science and sustainable energy.
\end{inparaenum}


\subsection{Novelty, level of ambition and foundational character}\label{sec:progress}

\eucommentary{
  \begin{compactitem}
  \item Describe the advance your proposal would provide beyond the
    state-of-the-art, and to what extent the proposed work is ambitious, novel
    and of a foundational nature. Your answer could refer to the ground-breaking
    nature of the objectives, concepts involved, issues and problems to be
    addressed, and approaches and methods to be used.
  \end{compactitem}
}

There are three major aspects of novelty and ambition.  Our work will be foundational and ground-breaking in 1) bringing forward the role of non-dissipative aspects in nonequilibrium physics, 2) its study of the dynamics of probes in nonequilibrium environments, 3) enabling systematic studies of certain technological questions and future activities:
ad 1) Now is the time to launch a decisive study on whether indeed the dynamical activity complements in essential ways entropic considerations in the construction of nonequilibrium statistical  mechanics. We propose therefore to devote a major and markedly innovative effort to unravel the conceptual, computational and operational meaning of the time-symmetric fluctuation sector, and how it enters observationally in the response of open systems driven away from equilibrium. At the same time, we expect such studies to unify various existing approaches and earlier suggestions such as in the escape rate formalism, or from the blowtorch theorem and in nonequilibrium reaction rate theory. The main objective then is to further develop fluctuation-response theory to systematize investigations on the operational-thermodynamic role and the statistical-mathematical nature of nonequilibrium kinetics for the elucidation of steady states beyond close-to-equilibrium regimes. The impact of a valid fluctuation-response theory away from equilibrium will be important in the more general context of complex systems and dissipative dynamics
A major part of theoretical and mathematical nonequilibrium studies of the last 150 years have been devoted to the relaxation to and the response in equilibrium. Where steady regimes further away from equilibrium are considered, the emphasis has remained on entropic considerations, either in the study of entropy production (as dissipated heat) or from the point of view of information-theoretic aspects.
Since some two decades nonequilibrium studies have revisited steady regimes further away from equilibrium, and a fluctuation-response theory is starting to emerge. Yet most results concern specific (often toy-)models, be it stochastic or highly chaotic. In general the broader theoretical ideas concentrate on entropy production as inherited from irreversible thermodynamics. Where time-symmetric quantities have entered, the understanding remains rather formal.
The usual formulation of statistical physics in its derivation of thermodynamics emphasizes the great disparity in phase space volumes corresponding to different macroscopic behavior. Away from thermodynamic equilibrium, phase space volume relations play a lesser role and moreover, as volumes get smaller, the surface area (in terms of exit and entrance rates) of phase space regions becomes more important.
Dynamical activity can in general be described as a measure of the system’s reactivity or of its escape rates, which can significantly change under driving conditions. Very recently we have learned that dynamical activity matters in nonequilibrium fluctuation-response theory, but no major observational consequences have been found yet.
Even though the notion of dynamical activity has been around in various guises in theoretical work in nonequilibrium physics, no operational or thermodynamic meaning was clearly formulated. To say it simply: we do not know how to measure it from some macroscopic quantity nor do we know how to represent it more statistical mechanically for a general system. No thermodynamic principles involve the dynamical activity.  The proposal chooses specific questions, some of which are mathematically very challenging, in fluctuation–response theory to open the meaning of dynamical activity, and to make it operational. The latter means to search for specific observational consequences and experimental control.
ad 2) We want to set up a fluctuation and response theory for nonequilibrium media from their influence on probe  dynamics, and we want to be able to control and monitor a system’s behavior from its contact with a nonequilibrium medium. The point is to explore the possible very new behaviour and properties of systems in contact with nonequilibrium media. A first question one can ask about a probe in active contact with a nonequilibrium sea is about the induced systematic statistical forces of the sea on the probe.  One starts from the more microscopic mechanical force and, assuming that the probe changes its state (e.g. position) on a much slower timescale than the medium, one integrates that mechanical force over the degrees of freedom of the stationary medium.  In equilibrium (that is, when the probe is in contact with an equilibrium reservoir) the resulting (statistical) forces are of gradient type.  The standard thermodynamic potentials are (indeed) the potentials from which the force can be derived as in Newtonian mechanics. The result there is that it allows us to draw free energy landscapes to understand the changes in the system, as if it concerned a conservative mechanical system for which we give the potential energy.  A special interesting example are the entropic forces, which work by the power of large numbers; then, the free energy has negligible energetic or enthalpic contributions.  These are known to be important in elasticity and polymer physics. Moreover, always for equilibrium, the statistical forces are additive for example in the sense that bulk contributions can easily be separated from boundary contributions.  The latter is crucial for thermodynamic behaviour.  For example, the very possibility of thermodynamic behaviour, the presence of an equation of state etc depend on that distinction.  All of that need not be true for probes (systems, walls, collective coordinates) in contact with (genuine) nonequilibrium media or reservoirs.  We have added the word “genuine” to indicate that we leave open the possibility that there are exceptional situations, for example in terms of specific models or special regimes where additive behaviour of the statistical forces would be restored.  We know that the close-to-equilibrium static large deviation functionals remain local and additive, so that it is fair to expect there the more usual thermodynamics compatible with e.g. 0th law behaviour or with the Clausius heat theorem that yields a calorimetric meaning to the fluctuation functionals close-to-equilibrium.   Yet, in general, and as generic as there are long range effects in nonequilibria we expect a breaking of additivity and of the gradient nature of statistical forces. Small perturbations in the sense of changes in boundary conditions or in changes of external potential can drastically change the stationary distribution also in the bulk of the system.  That has possibly interesting consequences in the manipulation of such forces, adding oscillatory components, going from attractive to repulsive and obtaining nonvanishing resulting forces that otherwise, in equilibrium, would be zero from mutually canceling contributions. 
It also means that the stationary positions of the probe (quasi-static) would be at different locations compared with equilibrium, allowing possibly increased stability of phases that would otherwise (in equilibrium again) be unstable.  The paradigmatic example is here the Kapitza oscillator (pendulum) which remains stable upright when being shaken.  That example is however likely to be systematized also outside the theory of dynamical systems, reaching then in this project the stability of macroscopic behaviour of the probe (collective coordinate) in for equilibrium very unlikely phases and values for order parameters.  Needless to  add here that this may have dramatic consequences on the phase diagram for systems in contact with nonequilibrium media, not only breaking the Gibbs phase rule but also introducing new phases of matter.  That is certainly one of the most exciting possibilities of explorations in the present project. As another theme of stability we want to understand some mechanism of homeostasis. In our set-up we treat spatially extended systems with time-dependent boundary driving. We will see under what conditions the bulk of the system reaches a steady (time-independent) regime. A related consequence of contact with nonequilibrium reservoirs is the possibility of population inversion in the system, which will mean that the effective temperature of the probe could be much larger for certain purposes.  There will for sure not be the usual Einstein or second fluctuation-dissipation relation between noise and friction on the probe which also entails that the friction coefficient can show further nontrivial behaviour.  The phenomena of shear thinning and possibly shear thickening could very well be related to these.
ad 3) 










\subsection{Research methods}\label{sec:methods}
\eucommentary{
  \begin{compactitem}
  \item Describe the overall research approach, the methodology and explain its
    relevance to the objectives.
  \item Where relevant, describe how sex and/or gender analysis is taken into
    account in the project's content.
  \end{compactitem}
}

The overall research approach is that of modern physics, with its traditional ways of open exchange of ideas and results.  There are fundamentally two components, which are deeply connected: the theoretical and experimental side.
We certainly must work to keep up and even improve the exchanges between these efforts and to bring about and foster multiple exchanges of ideas and questions.  The way from theory to experiment is also the method to reach more applied science and eventually to influence industrial and economic research centers and technology.  We have been asked before to hep with industrial projects (e.g. water transfer in porous media, new challenges for construction physics, crack evolution and creep in glassy materials etc) but the present project will allow systematic developments in contacting and helping industrial players.
For the more daily research methods, the basic tools are stochastic processes analysis and computer programming.  For the latter molecular dynamics simulations will play a very big role, also because they reach more realistic scenario's that can be implemented and further verified in experimental work and tests.  The experimental side is much based on methods in fluid mechanics and optical control.

\begin{asparaenum}
\item Mathematical methods: Here we mostly follow stochastic calculus and the
  specialized probabilistic techniques of dealing with spatially extended
  stochastic dynamics of interacting particle systems. Limit theorems and large
  deviation theory are central tools.
\item Theoretical framework: Much emphasis remains on path-integral techniques
  in dynamical ensembles.  Such a Lagrangian statistical mechanics takes up the
  entropy fluxes and the changes in DA to build the action functional for
  dynamical ensembles.  The nonequilibrium techniques will be essentially
  non-perturbative in the driving, and keeping a distance from the Keldysh
  formalisms.
\item Computational support: For various model systems that are not covered by
  theorems and analytic results, we use numerical methods and large scale
  simulation.
\item Experimental collaborations: Here we mostly seek collaborations with soft
  condensed matter labs.  Other questions relate more to biophysics and still
  others more to gas reactivity experiments.
\end{asparaenum}





\subsection{Interdisciplinary nature}\label{sec:interdisc}

\eucommentary{
  \begin{compactitem}
  \item Describe the research disciplines involved and the added value of the inter-disciplinarity.
  \end{compactitem}
}
Statistical and mathematical physics are by their nature rather inter-disciplinary.  Not only is their often a large intersection with mathematics, statistics and computer science, but also the subjects are often very diverse and reach in many various directions.  For the present project where we emphasize the nonequilibrium nature of phenomena and processes, there is the immediate contact with condensed matter physics , with fluid mechanics and with biological and medical/engineering sciences.  That is quite obvious as nonequilibria are ubiquitous.  Discovering common grounds of study and constructing for them a nonequilibrium statistical mechanics is exactly the heart of the present programme.
Research disciplines that are involved are very diverse. The most amazing examples of nonequilibrium are probably found in life processes, or perhaps in the origin of life itself.  There we see an open system full of transport processes, with little engines, pumps and cycles and the emergence of order on diverse spatio-temporal scales out of molecular complexity. The variety of phenomena where nonequilibrium considerations are essential is however much larger.  We find them at cosmological scales of the observable universe  and beyond the smallest sizes of nanotechnology.  Subatomic processes, the creation and annihilation of particles and fluctuations at the smallest dimensions produce sources of noise and matter, from which our world is finally made.  The second law of thermodynamics, the increase of entropy for closed systems, puts fundamental limitations on all models of the universe.  Various scenario's for the early universe  mimic transitions of metastable states to more   stable states, via
 nucleation processes as we see them also  in condensed matter.  On more earthly scales, evolution and the fight for low entropy create cycles of life and destruction that we recognize in all of nature.    Turbulent
  flow and nonlinear systems give other macroscopic realizations of nonequilibrium effects.   Climatology and ecology ask questions about the atmosphere and ocean dynamics and about food web chains as  nonequilibrium systems.
  In biology, molecular motors   give mesoscopic realizations of chemical engines and of transport on the molecular scales.   The cell itself, how it moves and how its membrane fluctuates on $\mu m$-scales, brings nonequilibrium statistical mechanics to life.   Chemical reactions are traditionally also sources of nonequilibrium changes.
 Further down, nanophysics studies processes of dissipation and transport at still smaller scales.




%%% Local Variables:
%%% mode: latex
%%% TeX-master: "proposal"
%%% End:
