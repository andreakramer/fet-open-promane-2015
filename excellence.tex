
\subsection{Targeted breakthrough, Long term vision and Objectives}\label{sec:objectives}

\eucommentary{
  \begin{compactitem}
  \item Describe the targeted scientific breakthrough of the project.
  \item Describe how the targeted breakthrough of the project contributes to a
    long-term vision for new technologies.
  \item Describe the specific objectives for the project, which should be clear,
    measurable, realistic and achievable within the duration of the project.
  \end{compactitem}
}

\paragraph{Scientific and technological goals}

The project is aimed at exploiting and controlling the motion of probes and devices in and
through contact with nonequilibrium media. ``Nonequilibrium media'' refers to an environment
(solvent, bulk of a material, etc) that is active or that is driven away from its
thermodynamic equilibrium.
%
The {\bf scientific
  breakthrough} involves an operational formulation of nonequilibrium statistical mechanics.
That goal implies the transition to a third major stage of {\bf fundamental nonequilibrium
  research}. Until now, the focus has been on studying the approach to equilibrium and on
the derivation and description of dissipative transport of mass, energy and momentum between
various equilibrium reservoirs.  By deriving and using the precise relation between
systematic forces, friction and noise on probes, we open a new and realistic class of
dynamical systems for which the control lies in the nonequilibrium environment.
Complementary, the understanding of the fluctuation and response behaviour of nonequilibrium
media allows us to extend the standard irreversible thermodynamics as it was conceived in
1930-1970.

Both, new control on dynamical systems and kinetic to thermodynamic behaviour are essential
ingredients in future technology and industrial applications.  To cite a few general
far-reaching objectives: the controlled motion or transport of material through living
tissue or in interaction with life processes, requires fundamental departures from
traditional transport theory; the stabilization of coherence as needed for future quantum
technology cannot be obtained within thermal environments but could be enhanced via
nonequilibrium contacts; mobilities and conductivities close-to-equilibrium are subject to
the fluctuation-dissipation relation but nonequilibrium driving can produce unseen transport
and conductivity behavior for the realization of new materials.  As is central to the goals
of the FET-open initiative, theoretical and experimental foundational work will lead to
create abilities and conditions under which technological innovation becomes possible.  For
example, serious conceptual problems need to be solved before active-particle suspension
will become a common basis for enhanced targeted catalysis or for fluid mixing and viscosity
reduction in industrial processing, or for noninvasive surgeries and drug delivery in
medicine, or for the design of smart materials with built-in actuation mechanisms. These
open conceptual issues are addressed via several pathways in the theoretical parts of the
proposal.  Leaping into macroscopic nonequilibria moreover implies that classical
measurement techniques, developed for equilibrium systems, will generally fail to work as
usual and will have to be replaced by new designs. Brownian thermometry provides an example
that is relevant to a variety of applications \cite{kroy:2014}, and which will be
generalized to conditions far from equilibrium, in the project.  On the practical and
applied side, radically new manipulation techniques are required to fully exploit the
innovative potential provided by active particles. As pertinent examples, we mention the
thermophoretic trap and the photon-nudging method pioneered by the experimental group in
Leipzig \cite{Qian2013,Braun:NanoLetters:2015}.  Both techniques steer particles (passive
and active particles in the trap and hot active particles in the nudging technique,
respectively) by exploiting far-from equilibrium conditions in the solvent. Notably, they
work without imposing external forces, via directly addressing the nonequilibrium solvent
conditions and thereby the (self-)thermophoretic propulsion (or ''swimming'') of the
particles. Their usefulness is enhanced by employing Maxwell-demon type control
strategies. Prototypes of these genuinely nonequilibrium techniques exist and will be
employed and refined (e.g. by optimizing dedicated control strategies) within the project.
One should realize that the assumption of a thermally equilibrated bath is not fulfilled in
most industrially (oil, polymer melts, dense colloidal or micellar suspensions) or
biologically relevant (blood, mucus, DNA-solutions) liquids which display visco-elastic
properties, i.e. elastic behaviour like a solid and viscous behavior like a fluid. As a
consequence, such fluids exhibit large relaxation times (up to seconds), and can thus easily
be driven out of thermal equilibrium by a forced colloidal probe. Within this proposal, we
want to study how the local perturbation of visco-elastic systems affects the dynamics and
effective interactions of colloidal particles in such media with particular view on
e.g. memory effects, many body interactions and local non-linear rheological properties.
Similarly, time-dependent phenomena at very long and even at very fast time scales, such as
in glasses or electronic devices, respectively, elude the usual equilibrium thermodynamic
approach.  Such studies will in general reveal the properties of nonequilibrium baths which
are of manifold importance.

More generally and as part of the {\bf long term vision} we believe that a systematic
understanding of nonequilibrium response and fluctuation behavior, beyond the traditional
linear response regime, will uncover new parameters to control technological processes, to
steer motion and to explore and stabilize new interesting material properties.

\paragraph{Objectives}

That leads to two major sides in this research proposal:
\begin{inparaenum}[A.]
\item the statistical mechanical study of probes under nonequilibrium conditions, including
the statistical thermodynamics of active media;
\item the control theory for nonequilibrium media, including kinetic parameters as they
contribute to the dynamical activity.
\end{inparaenum}
Besides the theoretical and experimental underpinning of the recent insights and progress in
nonequilibrium science, replacing much of the current trial and error methods as mentioned
above, part of the general objective also includes to realize numerical and algorithmic
methods for efficient simulation of nonequilibrium environments. Specific {\bf objectives}
for the project then include:
\begin{inparaenum}[A.]
\item To provide modelling and simulation techniques under the paradigm of probes and
devices in contact with nonequilibria.  These are high level molecular dynamics codes for
simulating and calculating the emergent behavior of contacts and probes.  These techniques
will be made available through all and for all work packages. It is an important objective
to make that tool also available to industrial partners, challenged by the complexity of
dealing with nonequilibria.
\item Offering a scientific basis and new avenues for the realization of new properties of
matter (in response and transport), and the enhanced stability of thermodynamic phases or
behaviour.  See further also in Work Package 2 for the conception of
meta-materials. Specifically we have good understanding for workable implementations of
negative compressibilities.
\item The control of colloidal motion in nonequilibrium baths and in nonlinear media.  See
Work Package 3 for the theoretical and experimental study of probes in visco-elastic
solvents.  Most fluids found in industry and biology are visco-elastic and/or out of
equilibrium.  Control of shear, the development of new separation techniques based on
dynamical activities and steering of clustering properties are achievable in the current
project.
\item Kinetic control, see e.g. Work Package 4 below.  Many challenges are kinetic rather
than thermodynamic, especially for processes and manipulations in nonequilibirum
environments or for time-dependent parameters.  Irreversible thermodynamics cannot deal with
the overwhelming nature of kinetic control, but nonequilibrium statistical insights point to
the control of dynamical activity; what some of us have called the frenetic contribution.
One objective is to make a frenometer for operational control and measurement of dynamical
activity.  That is essential for the control and steering of dynamics out-of equilibrium.
For example, polymer physics outside equilibrium will supply a necessary and complementary
component to the vast domain of polymer research with thermodynamic control.
\item Develop the theory of statistical forces on probes in contact with macroscopic
nonequilibria, essential for controlling motion in turbulent media (atmosphere dynamics) or
under biological flow (blood streams).  Work Packages 5 and 6 will work with experimental
and observational predictions of new emergent behavior for probes in nonequilibria, as
resulting from the essentially nongradient, nonadditive and nonlocal features of these
nonequilibrium statistical forces.
\end{inparaenum}
These objectives will find further realizations in our existing contacts with colloidal and
polymer scientists, microbio-engineers and atmosphere researchers. Related topics are for
example found in the physics of polymer folding and the phase diagram of bio-polymers, the
mechanical properties of the cytoskeleton and fluctuations of bio-membranes, and the
statistical analysis and development of climate and weather models. The overarching goal
remains to make available the science of nonequilibria to technological and societal
developments.



\subsection{Relation to the work programme}\label{sec:relation-wp}

We review here th character of our project \TheProject as appropriate in the light of the
FET-Open objectives:
\begin{compactdesc}
\item[Long-term vision] Current technologies rely on an understanding of thermodynamics that
dates from half a century ago. The proposed update of the set of tools available to
scientists and to the industry, on the basis of nonequilibrium thermodynamics, is the first
step in a change of paradigm across many technological fields.
\item[Breakthrough S\&T target] The developments within \TheProject rely directly on
technological applications, either via relevant physical models or via the experimental
partners (\site{USTUTT} and \site{ULEI}). Advanced or novel nonequilibrium theories are currently not used in
the industry and we will provide the building blocks to enable it. One can expect that the
development of nonequilibrium theories will deliver a competitive advantage, similar in
principle to the one delivered by pioneers as Carnot, Clausius and Joule in the 19th century that made possible the industrial revolution.
%
Concretely, an advantage arises from the development of control strategies (see
\WPref{WPdissipation}) that are unavailable currently or by the proposed development of new types of
materials or a better understanding of solvent properties (e.g. viscoelastic media in
\WPref{WPbrown}).
\item[Foundational] The project is foundational in the construction of a workable
nonequilibrium statistical mechanics. A completely new development and emphasis is on the
study of time-symmetric fluctuations and how the activity of nonequilibria influences the
dynamics of probes immersed in them. There are strong mathematical and theoretical
components in the project, especially within the nodes of Leuven, Eindhoven and Prague.  The
elucidation of statistical forces and colligative properties from nonequilibrium media opens
many new opportunities for applications of chemical thermodynamics.
\item[Novelty] The changes that we are proposing do not aim at a minor modification of the
existing processes but at replacing them with a better understanding and, more specifically,
new strategies for diagnosis and control.  We aim at new unseen material properties, at
gaining stability of macroscopic phases which are otherwise unstable and at control of
mesoscopic transport in active media.
\item[High risk] The gain is high, so is the risk for missing certain opportunities.  The
main risk is in the incomplete transfer of tools and knowledge to the technological
developers.  That is why we will pay also special attention to increased contacts with
private and industrial partners.  We will organize Industry Days on a European level to
enhance the exchange of nonequilibrium problems and their solutions.  It will also be
important to have follow-up projects in direct contact and collaboration with companies and
economic actors.
\item[Interdisciplinarity] Since its invention, the field of thermodynamics has extended its
scope from the study of steam engines to being used in most of the physical sciences and
engineering disciplines: chemical reactions, materials sciences, biological systems
(e.g. heat regulation or food chain), information theory, etc. The progresses we envision in
\TheProject are firmly based on physical models and experiments but domains of application
go far beyond.
\end{compactdesc}



For this FET-open research programme we set as task to provide theoretical and experimental
ground for dealing with and steering probes in active, driven and nonlinear media.  Many
aspects of technology, whether in production, optimization or automation, require
understanding and controlling nonequilibrium processes.  That includes both the traditional
engineering processes, but also and more than ever, food processing, pharmaceutical,
biochemical and biomedical applications.  Scientific knowledge is the only and proven road
to success in these applications and ranks far above high through-put or trial and error
based methods.  Moreover, having today a choice of so many materials or components, and
having also available huge data sets can leave the applied scientist paralysed in front of
the many of today's technological and societal challenges; tradition and experience built
over the years alone do not suffice.  Solid scientific and high-level understanding are the
keys to further important developments for sustaining or improving technological levels.
The present project emphasizes the nonequilibrium aspects, complementing and moving beyond
the great success of equilibrium statistical mechanics and irreversible thermodynamics in
obtaining and revolutionizing technology for the last 100 years.
The present proposal has as major challenge to control and steer devices or materials in contact with and through nonequilibrium and nonlinear environments.  That includes a great number of possible scenarios, including,\\
a) new materials and material properties:  can one obtain interesting transport coefficients and use them in new situations by contact with rightly monitored nonequilibria.  What are the thermal conductivities or mobilities for nonequilibrium materials?  What are the differential equations of relaxation to nonequilibrium stationary conditions?\\
b) Biomedical processes and transport: how to control transport of particles in nonequilibrium flows or through living tissues?  How are chemical nonequilibria transduced as signals for mechanical changes? Are self-organized patterns and morphognesis obtained from micriscopic nonequilibrium descriptions?\\
c) Biomechanical structures and motility: how do the mechanical properties change of materials undergoing active processes?  What are the shapes and stresses of bio-materials?\\
d) Quantum stability and coherence control: can one obtain enhanced stability of quantum structures and coherence by providing nonequilibrium control?\\
e) Transient and fast processes: Is there a nonequilibrium thermodynamic treatment of quenched dynamics and of ultraslow and ultrafast processes?  What is the validity there of the concept of effective temperature and can one avoid kinetic constraints (with associated many-body localization, jamming, clustering...) via appropriate nonequilibrium control?\\
f) Economy and society: are there analogues to the nonequilibrium notions of entropy production and dynamical activity in at least theoretical models of economic markets or society changes?  Is the volatility predictable ?  How can nonequilibrium statistical physics really contribute in the climate science debate, or in the development of technologies with  sustainable energy.


\subsection{Novelty, level of ambition and foundational character}\label{sec:progress}

\eucommentary{
  \begin{compactitem}
  \item Describe the advance your proposal would provide beyond the
    state-of-the-art, and to what extent the proposed work is ambitious, novel
    and of a foundational nature. Your answer could refer to the ground-breaking
    nature of the objectives, concepts involved, issues and problems to be
    addressed, and approaches and methods to be used.
  \end{compactitem}
}

The first stage of modern thermodynamics consisted in defining equilibrium systems, a field
that gave rise to equilibrium statistical mechanics. The current knowledge of nonequilibrium
physics represents, as we have outlined in Sec.~\ref{sec:objectives}, the {\em second stage}
of the domain: the study of relaxation to equilibrium and response theory close to
equilibrium.

\paragraph{Non dissipative nonequilibrium physics}

The recently developed idea of dynamical activity starts a new line of conceptual
understanding that is missing from today's nonequilibrium thermodynamics. The focus on
entropy production has brought many insights in the last 150 years (in physics and other
fields as well, such as information theory) but is however restrictive.
%
The impact of a valid fluctuation-response theory away from equilibrium will be important in
a more general context where one is interested in complex systems and in dissipative
dynamics.

Since about two decades, nonequilibrium studies have revisited steady regimes further away
from equilibrium, and a fluctuation-response theory is starting to emerge. Our current
understanding of time-symmetric quantities remains rather formal and most results concern
specific systems (often ``toy models'') , be it of stochastic or highly chaotic nature.
%
\TheProject chooses specific questions, some mathematically very challenging and some very
pragmatic, in fluctuation-response theory to open the meaning of dynamical activity, and to
make it operational.
%
Explicitly, we will focus on consequences that can be observed and measured, and on the
experimental control that this theory brings.

\paragraph{Probing nonequilibrium}

We want to set up a fluctuation and response theory for nonequilibrium media
from their influence on probe dynamics, and we want to be able to control and monitor a
system’s behavior from its contact with a nonequilibrium medium. The point is to explore the
possible very new behaviour and properties of systems in contact with nonequilibrium
media. A first question one can ask about a probe in active contact with a nonequilibrium
sea is about the induced systematic statistical forces of the sea on the probe.  One starts
from the more microscopic mechanical force and, assuming that the probe changes its state
(e.g. position) on a much slower timescale than the medium, one integrates that mechanical
force over the degrees of freedom of the stationary medium.  In equilibrium (that is, when
the probe is in contact with an equilibrium reservoir) the resulting (statistical) forces
are of gradient type.  The standard thermodynamic potentials are (indeed) the potentials
from which the force can be derived as in Newtonian mechanics. The result there is that it
allows us to draw free energy landscapes to understand the changes in the system, as if it
concerned a conservative mechanical system for which we give the potential energy.  A
special interesting example are the entropic forces, which work by the power of large
numbers; then, the free energy has negligible energetic or enthalpic contributions.  These
are known to be important in elasticity and polymer physics. Moreover, always for
equilibrium, the statistical forces are additive for example in the sense that bulk
contributions can easily be separated from boundary contributions.  The latter is crucial
for thermodynamic behaviour.  For example, the very possibility of thermodynamic behaviour,
the presence of an equation of state etc depend on that distinction.  All of that need not
and in general, will not,
be true for probes (systems, walls, collective coordinates) in contact with (genuine)
nonequilibrium media or reservoirs. That has possibly
interesting consequences in the manipulation of such forces, adding oscillatory components,
going from attractive to repulsive and obtaining nonvanishing resulting forces that
otherwise, in equilibrium, would be zero from mutually canceling contributions.
%
It also means that the stationary positions of the probe (quasi-static) would be at
different locations compared with equilibrium, allowing possibly increased stability of
phases that would otherwise (in equilibrium again) be unstable.  The paradigmatic example is
here the Kapitza oscillator (pendulum) which remains stable upright when being shaken.  That
example is however likely to be systematized also outside the theory of dynamical systems,
reaching then in this project the stability of macroscopic behaviour of the probe
(collective coordinate) in for equilibrium very unlikely phases and values for order
parameters.  Needless to add here that this may have dramatic consequences on the phase
diagram for systems in contact with nonequilibrium media, not only breaking the Gibbs phase
rule but also introducing new phases of matter.  That is certainly one of the most exciting
possibilities of explorations in the present project. As another theme of stability we want
to understand some mechanism of homeostasis. In our set-up we treat spatially extended
systems with time-dependent boundary driving. We will see under what conditions the bulk of
the system reaches a steady (time-independent) regime. A related consequence of contact with
nonequilibrium reservoirs is the possibility of population inversion in the system, which
will mean that the effective temperature of the probe could be much larger for certain
purposes.  There will for sure not be the usual Einstein or second fluctuation-dissipation
relation between noise and friction on the probe which also entails that the friction
coefficient can show further nontrivial behaviour.  The phenomena of shear thinning and
possibly shear thickening could very well be related to these.
%


\subsection{Research methods}\label{sec:methods}
\eucommentary{
  \begin{compactitem}
  \item Describe the overall research approach, the methodology and explain its
    relevance to the objectives.
  \item Where relevant, describe how sex and/or gender analysis is taken into
    account in the project's content.
  \end{compactitem}
}

The overall research approach is that of modern physics, with its traditional ways of open
exchange of ideas and results.  As generic and general as that may sound, the history of the
last 200 years proves that these make the best and most direct road to innovative technology
research.  The fundamentally connected components in this research have mathematical,
conceptual, computational and experimental sides.
%
We certainly must work to keep up and even improve the exchanges between these efforts and
to bring about and foster multiple exchanges of ideas and questions.  The way from theory to
experiment is also the method to reach more applied science and eventually to influence
industrial and economic research centers and technology.  About all of the project members
have been asked before in consulting with industrial projects (e.g. water transfer in porous
media, new challenges for construction physics, crack evolution and creep in glassy
materials etc) but the present project will allow systematic developments in contacting and
helping industrial players.


For the more daily research methods, the basic tools are stochastic processes, analysis and computer programming.  For the latter molecular dynamics simulations will play a very big role, also because they reach more realistic scenario's that can be implemented and further verified in experimental work and tests.  The experimental side is much based on methods in fluid mechanics and optical control.\\

1. Mathematical methods: Here we mostly follow stochastic calculus and the specialized
probabilistic techniques of dealing with spatially extended stochastic dynamics of
interacting particle systems. Limit theorems and large deviation theory are central tools.
The Eindhoven node in in an Applied Mathematics department, and Leuven and Prague memmbers
are part of the researach group in mathematical physics.

2. Theoretical framework: Much emphasis remains on path-integral techniques in dynamical
ensembles.  Such a Lagrangian statistical mechanics takes up the entropy fluxes and the
changes in dynamical activity to build the action functional for dynamical ensembles.  The
nonequilibrium techniques will be essentially non-perturbative in the driving, and keeping a
distance from the Keldysh formalisms. Strong theoretical efforts are availabel in the
Stuttgart Max Planck Institute for Intelligent Systems, and in the Theoretical physics
Institutes of Prague, Leuven, Padova and Leipzig.

3. Computational support: For various model systems that are not covered by theorems and
analytic results, we use numerical methods and large scale simulation.  Main experts are
here found in leuven and in Padova, with various schooling and outreach initiatives to
promote the computational component.

4. Experimental work: Here we mostly undertake experiments on active materials and particles
in the appropriate soft condensed and liquid matter labs.  Other questions relate more to
biophysics and polymer behavior. The experimental component is present in the Stuttgart and
Leipzig labs.

Overall, this project will also deliver methodology and development tools, such as
\begin{inparaenum}[A.]
	\item the making available of strong molecular dynamics simulation codes for the behavior of
	probes in nonequilibrium media and for the dynamics of active particles;
	\item the development of operational tools for measuring and quantifying kinetic parameters
	as used in the theory of dynamical ensembles, including the measurement of excess
	thermodynamic quantities and dynamical activities;
	\item providing a toolkit for {\it ab initio} calculations for macroscopic nonequilibrium
	behavior, possibly allowing new stable phases, including a quantitative description of
	(also) not-dissipative relaxation to stationary and steady conditions.
\end{inparaenum}


\subsection{Interdisciplinary nature}\label{sec:interdisc}

\eucommentary{
  \begin{compactitem}
  \item Describe the research disciplines involved and the added value of the inter-disciplinarity.
  \end{compactitem}
}

Statistical and mathematical physics are by their nature rather inter-disciplinary.  Not
only is there very often a large intersection with mathematics, statistics and computer
science, but also the subjects are often very diverse and reach out in many various
directions.  For the present project where we emphasize the nonequilibrium nature of
phenomena and processes, there is the immediate contact with condensed matter physics, with
fluid mechanics and with biological and medical/engineering sciences.  That is quite obvious
as nonequilibria are ubiquitous.  Discovering common grounds of study and constructing for
them a nonequilibrium statistical mechanics is exactly the heart of the present programme.

Research disciplines that are involved are very diverse. The most amazing examples of
nonequilibrium are probably found in life processes, or perhaps in the origin of life
itself.  There we see an open system full of transport processes, with little engines, pumps
and cycles and the emergence of order on diverse spatio-temporal scales out of molecular
complexity. The variety of phenomena where nonequilibrium considerations are essential is
however much larger.  We find them at the smallest sizes of nanotechnology. Fluctuations at
the smallest dimensions produce sources of noise and matter, from which our world is finally
made.  Turbulent flow and nonlinear systems give macroscopic realizations of nonequilibrium
effects.  Climatology and ecology ask questions about the atmosphere and ocean dynamics and
about food web chains as nonequilibrium systems.  In biology, molecular motors give
mesoscopic realizations of chemical engines and of transport on the molecular scales.  The
cell itself, how it moves and how its membrane fluctuates on $\mu m$-scales, brings
nonequilibrium statistical mechanics to life.  Chemical reactions are traditionally also
sources of nonequilibrium changes.





%%% Local Variables:
%%% mode: latex
%%% TeX-master: "proposal"
%%% End:
