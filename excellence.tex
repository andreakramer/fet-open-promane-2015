
Body of section ``Excellence''.

\subsection{Targeted breakthrough, Long term vision and Objectives}\label{sec:objectives}

\eucommentary{
  \begin{compactitem}
  \item Describe the targeted scientific breakthrough of the project.
  \item Describe how the targeted breakthrough of the project contributes to a
    long-term vision for new technologies.
  \item Describe the specific objectives for the project, which should be clear,
    measurable, realistic and achievable within the duration of the project.
  \end{compactitem}
}

\subsection{Relation to the work programme}\label{sec:relation-wp}

\eucommentary{
  \begin{compactitem}
  \item Indicate the work programme topic to which your proposal relates, and
    explain how your proposal addresses the specific challenge and scope of that
    topic, as set out in the work programme.
  \end{compactitem}
}

\subsection{Novelty, level of ambition and foundational character}\label{sec:progress}

\eucommentary{
  \begin{compactitem}
  \item Describe the advance your proposal would provide beyond the
    state-of-the-art, and to what extent the proposed work is ambitious, novel
    and of a foundational nature. Your answer could refer to the ground-breaking
    nature of the objectives, concepts involved, issues and problems to be
    addressed, and approaches and methods to be used.
  \end{compactitem}
}

\subsection{Research methods}\label{sec:methods}
\eucommentary{
  \begin{compactitem}
  \item Describe the overall research approach, the methodology and explain its
    relevance to the objectives.
  \item Where relevant, describe how sex and/or gender analysis is taken into
    account in the project's content.
  \end{compactitem}
}

The overall research approach is that of modern physics, with its traditional ways of open exchange of ideas and results.  There are fundamentally two components, which are deeply connected: the theoretical and experimental side.
We certainly must work to keep up and even improve the exchanges between these efforts and to bring about and foster multiple exchanges of ideas and questions.  The way from theory to experiment is also the method to reach more applied science and eventually to influence industrial and economic research centers and technology.  We have been asked before to hep with industrial projects (e.g. water transfer in porous media, new challenges for construction physics, crack evolution and creep in glassy materials etc) but the present project will allow systematic developments in contacting and helping industrial players.
For the more daily research methods, the basic tools are stochastic processesm analysis and computer programming.  For the latter molecular dynamics simulations will paly a very big role, also because they reach more realistic scenario's that can be implemented and further verified in experimental work and tests.  The experimental side is much based on methods in fluid mechanics and optical control.



\subsection{Interdisciplinary nature}\label{sec:interdisc}

\eucommentary{
  \begin{compactitem}
  \item Describe the research disciplines involved and the added value of the inter-disciplinarity.
  \end{compactitem}
}
Statistical and mathematical physics are by their nature rather inter-disciplinary.  Not only is their often a large intersection with mathematics, statistics and computer science, but also the subjects are often very diverse and reach in many various directions.  For the present project where we emphasize the nonequilibrium nature of phenomena and processes, there is the immediate contact with condensed matter physics , with fluid mechanics and with biological and medical/engineering sciences.  That is quite obvious as nonequilibria are ubiquitous.  Discovering common grounds of study and constructing for them a nonequilibrium statistical mechanics is exactly the heart of the present programme.
Research disciplines that are involved are very diverse. The most amazing examples of nonequilibrium are probably found in life processes, or perhaps in the origin of life itself.  There we see an open system full of transport processes, with little engines, pumps and cycles and the emergence of order on diverse spatio-temporal scales out of molecular complexity. The variety of phenomena where nonequilibrium considerations are essential is however much larger.  We find them at cosmological scales of the observable universe  and beyond the smallest sizes of nanotechnology.  Subatomic processes, the creation and annihilation of particles and fluctuations at the smallest dimensions produce sources of noise and matter, from which our world is finally made.  The second law of thermodynamics, the increase of entropy for closed systems, puts fundamental limitations on all models of the universe.  Various scenario's for the early universe  mimic transitions of metastable states to more   stable states, via
 nucleation processes as we see them also  in condensed matter.  On more earthly scales, evolution and the fight for low entropy create cycles of life and destruction that we recognize in all of nature.    Turbulent
  flow and nonlinear systems give other macroscopic realizations of nonequilibrium effects.   Climatology and ecology ask questions about the atmosphere and ocean dynamics and about food web chains as  nonequilibrium systems.
  In biology, molecular motors   give mesoscopic realizations of chemical engines and of transport on the molecular scales.   The cell itself, how it moves and how its membrane fluctuates on $\mu m$-scales, brings nonequilibrium statistical mechanics to life.   Chemical reactions are traditionally also sources of nonequilibrium changes.
 Further down, nanophysics studies processes of dissipation and transport at still smaller scales.




%%% Local Variables:
%%% mode: latex
%%% TeX-master: "proposal"
%%% End:
