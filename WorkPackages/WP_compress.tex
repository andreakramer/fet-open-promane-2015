\begin{workpackage}[id=WPcompress,wphases=0-36,
  short=Nonequilibrium compressibility, %XXX act. WP,% for Figure 5.
  title=Nonequilibrium compressibility, % XXX actual Work Package,
  lead=Padova,
  PadovaRM=36]




\begin{wpobjectives}
  The objectives of this WP are:
  \begin{compactitem}
  \item Understanding the response to compression of systems holding a heat flow
  \item Finding if an averaged description (no microscopic details) may lead to effective fluctuation-response relations
  \item Understanding if we can realize bodies with negative differential compressibilities if subject to thermal gradients 
  \end{compactitem}
\end{wpobjectives}

\begin{wpdescription}

Monitoring the volume of a system in equilibrium represents a standard procedure for assessing its state variables.
We know for example that for a gas in equilibrium there are state equations relating pressure and temperature to
the volume. Also the response to a compression is predictable by knowing the equilibrium correlations between the
compressing force and the volume, thanks to the fluctuation-dissipation theorem. The situation is less clear far
from equilibrium: how should we characterize the response to compression of a gas maintained between two temperatures?

It is not trivial to predict by common sense the behavior of systems under nonequilibrium conditions. 
One would naively expect a reduction of the volume by increasing the external compression, but there are already 
several examples of anomalous, or negative, response in nonequilibrium systems.
We have in mind instances of negative differential mobility, occurring when a particle starts to drift 
slower by increasing the external drive.

Of course an important class of systems for which we need to understand the nonequilibrium response is that of solids.
There are important technological applications that involve keeping a metal in a temperature gradient, a condition
that generates heat flows and constant entropy production. To name a couple, strongly
heated micro-cantilevers and high-quality table-top oscillators of gravitational wave detectors are definitely
systems for which the study of nonequilibrium fluctuations and response is timely and relevant.
%Cantilevers of the size up to $\approx 1 mm$ are used as devices to measure microscopic forces. To operate, a laser
%may be pointed at their tip, so that a heat flux is installed from there to the base attached to a frame acting
%as a heat sink. Recently it was found that vibrational modes of this system hold uneven amounts of energy.
%Similar procedure and results, but on a larger scale, pertain the dynamics of table-top gravitational wave detectors.
%Given the uneven distribution of energy among modes and the constant flow of energy between them, it is not
%possible to use notions of equilibrium statistical mechanics to guess how the system would respond to a compressive 
%periodic forcing. To this end, one needs to develop response theory of nonequilibrium systems.

These examples stimulate the characterization of nonequilibrium systems with heat flows.
We focus on the particular response coefficient of compressibility.
Thus, for nonequilibrium systems we aim at 
understanding how the response to compression takes place and how it
may be predicted by just knowing suitable (unperturbed) steady-state correlations.
This understanding is important for a second, main stage of this WP, as explained next.

The theoretical tools we will use in the study of nonequilibrium compressibility are based on a recent line of research
that aims at studying nonequilibrium linear response for stochastic systems. From such approach there emerges an
important point: to understand the nonequilibrium response we need two families of system descriptors. In the first
kind we have entropy production, a familiar concept from the study of standard equilibrium systems. There is
 nevertheless the need to introduce a second class of indicators that measures the internal activity of the system.
During the last decade it became clear that such estimate of how the system is frenzy, or ``frenetic'', is necessary 
to understand nonequilibrium environments in general. In the context of response theory one thus needs to know
how a system, upon perturbation, becomes more or less active in its dynamics. This might involve needing
to measure delicate microscopic details, which could not be easily accessible experimentally. Hence, on the practical
side we have the important task to adapt the theoretical framework to the available experimental capabilities.

The possibility to extract energy from an installed heat channel is the mechanism that we want to
analyze, to see if it can lead to the behavior of negative differential compressibility,
i.e.~the system expands when compressed. 
Nonequilibrium response may be written as an entropic term minus a frenetic one.
In our case the former is the correlation between the size of the system and the entropy produced by the compression.
The frenetic term is another correlation between size and dynamical aspects that can be monitored numerically.
This scheme thus furnishes us with a guideline to distinguish if simulations of a specific material is yielding data 
going in the direction of negative compressibility. In particular, since typically the entropic term is positive, we
are encouraged to privilege the development of models that show a large frenetic term, so that its subtraction from
the entropic one gives a total negative response.
We have in this way delineated a guided trial-and-error strategy, in which the notions of 
nonequilibrium statistical mechanics help us in faster developments of numerical studies. This is expected to
speed up the research if compared to a blind trial-and-error strategy.

A successful achievement in finding models of materials that have negative compressibility,
if subject to heat flows, may trigger technological applications with immense
relevance. Metamaterials are indeed nowadays much studied because, for example, micro-actuators would be
more easily realized if materials with a negative compressibility were introduced. We will investigate
if steady nonequilibrium conditions could produce this effect, as opposed to the recently considered 
mechanism of rearrangements from metastable states or of enhanced expansions in orthogonal directions
due to a wine-rack structure of the materials. 
Our study as a byproduct would yield in any case a deeper understanding on the compressibility
and on the vibrational modes of solids kept in temperature gradient and experiencing thus a heat flux.

In conclusion it is important to note that (a) the anomalous response we seek is in the direction 
of the compression, (b) that it is due to compression-enhanced extraction of energy from a heat channel,
which is prompt, excites the system expanding it, 
and does not require going through an hysteresis of the material, and (c) that, even if
the outcome of our investigation would be that negative compressibility is not emerging, we would still 
have a timely comprehensive study of compressibility of solids in a temperature gradient. In this sense,
we can consider this WP as a {\em low} risk high reward research plan!



\end{wpdescription}
\begin{tasklist}

  %\begin{task}[title=TASK1,id=task1,PM=12,lead=Padova,wphases=0-24!0.5]
  \begin{task}[title=TASK1,id=task1,PM=3,lead=Padova,wphases={0-6!1,6-12!0.5}]

    First task of the project: 
    we plan to characterize the response to compression
    for models of solids experiencing a heat flow due to different boundary temperatures.
    by applying available theoretical formulations to the study of numerical simulations,
    the main aim is to make a setup for the other tasks.
    
  \end{task}

  %\begin{task}[title=TASK2,id=task2,PM=12,lead=Padova,wphases=0-24!0.5]
  \begin{task}[title=TASK2,id=task2,PM=3,lead=Padova,wphases={6-18!0.5}]

    Second task of the project: 
    we are interested in finding if microscopic details, available of course in simulations, 
    are needed or if a more experimental-like coarse grained description is sufficient or even convenient for
    predicting the response of the system's length to an increased compression. In practice, for example,
    by running a simulation of coupled oscillators we assume that only ``macroscopic'' quantities as the system length
    are measurable and from there we try to see if a coarse-grained fluctuation-response approach is able to
    describe the behavior of the system when a compression is added.
    
  \end{task}

  \begin{task}[title=TASK3,id=task3,PM=6,lead=Padova,wphases={12-18!0.5,18-36!1}]
  %\begin{task}[title=TASK3,id=task3,PM=24,lead=Padova,wphases=24-48!1.0]

    Third task of the project: 
    as a main task of this WP,
    we aim at finding peculiar inter-particle potentials that may lead to negative 
    compressibility in solids experiencing heat flows. 
    We plan to first operate at the conceptual level of toy models, and then to move
    at a more detailed level of realistic models. We use an effective guided
    trial-and-error strategy to promote simulations of models with large
    frenetic term in the response.
    
  \end{task}



\end{tasklist}

\begin{wpdelivs}
  \begin{wpdeliv}[due=12,id=mydeliv1,dissem=PU,nature=DEM,lead=Padova]
      {First deliverable, after 1 year: characterization of compressibility for solids between two temperatures.}
  \end{wpdeliv}
  \begin{wpdeliv}[due=18,id=mydeliv2,dissem=PU,nature=DEM,lead=Padova]
      {Second deliverable, after 1.5 years: clear picture on how microscopic details are needed to predict compressibility from experiments}
\end{wpdeliv}
  \begin{wpdeliv}[due=36,id=mydeliv3,dissem=PU,nature=DEM,lead=Padova]
      {Third deliverable, after 3 years: characterization of the compressibility for a wide class of heat-conduction models, understanding of whether negative compressibility is achievable in real systems.}
\end{wpdeliv}
\end{wpdelivs}




\end{workpackage}
