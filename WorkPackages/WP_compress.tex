\begin{workpackage}[id=WP2ID,wphases=0-48,
  short=Nonequilibrium compressibility, %XXX act. WP,% for Figure 5.
  title=Nonequilibrium compressibility, % XXX actual Work Package,
  lead=Padova,
  PadovaRM=12,
  PAR2RM=6,
  PAR3RM=24]

\begin{wpobjectives}
  The objectives of this WP are:
  \begin{compactitem}
  \item Find the response to compression for bodies holding a heat flow
  \item Find if an averaged description (no microscopic details) may lead to effective fluctuation-response relations
  \item Understand if negative differential compressibilities can be realized
  \item Understand how active systems do respond to compression
  \end{compactitem}
\end{wpobjectives}

\begin{wpdescription}
Monitoring the volume of a system in equilibrium represents a standard procedure for assessing its state variables.
We know for example that for a gas in equilibrium there are state equations relating pressure and temperature to
the volume. Also the response to a compression is predictable by knowing the equilibrium correlations between the
compressing force and the volume, thanks to the fluctuation-dissipation theorem. The situation is less clear far
from equilibrium: how should we characterize the response to compression of a gas maintained between two temperatures?
Another example of unknown response would be that of ``active gases'', represented for example by bacteria within
a permeable box. In this case a compression would likely expel the water from the pores and expose our sensors
to a different interaction with the compressed bacteria. It is not trivial to predict by common sense the behavior 
of the system under these conditions. We would naively expect a reduction of the volume by increasing the external
compression, but there are already several examples of anomalous, or negative, response in nonequilibrium systems.
We have in mind the example of negative differential mobility, where a particle starts to drift slower by increasing
the external drive.

Of course another class of systems for which we need to understand the nonequilibrium response is that of solids.
There are important technological applications that involve keeping a metal in a temperature gradient, a condition
that generates heat flows and constant entropy production. To name a couple, strongly
heated micro-cantilevers and high-quality table-top oscillators of gravitational wave detectors are definitely
systems for which the study of nonequilibrium fluctuations and response is timely and relevant.
Cantilevers of the size up to $\approx 1 mm$ are used as devices to measure microscopic forces. To operate, a laser
may be pointed at their tip, so that a heat flux is installed from there to the base attached to a frame acting
as a heat sink. Recently it was found that vibrational modes of this system hold uneven amounts of energy.
Similar procedure and results, but on a larger scale, pertain the dynamics of table-top gravitational wave detectors.

All these examples stimulate the characterization of nonequilibrium systems with entropic flows, either macroscopically
 evident as in temperature gradients or microscopically ``hidden'' in the chemical reactions that drive bacteria. 
We focus on a particular feature, namely compressibility, of such nonequilibrium systems and we aim at understanding
how the response to compression takes place and, more importantly,
may be predicted by just knowing steady-state correlations
between observable and suitable descriptors of the system. The queen observable in this case would be the size of the
system. For this, the approach could be rephrased as a study of a nonequilibrium compressibility. The theoretical 
approaches from which to start, however, would allow a broader treatment and a focus on more general observables 
(e.g. internal energy, variances, flows of energy, etc.).

The theoretical tools we will use in the study of nonequilibrium compressibility are based on a recent line of research
that aims at studying nonequilibrium linear response for stochastic systems. From such approach there emerges an
important point: to understand the nonequilibrium response we need two families of system descriptors. In the first
kind we have entropy production, a familiar concept from the study of standard equilibrium systems. There is
 nevertheless the need to introduce a second class of indicators that measures the internal activity of the system.
During the last decade it became clear that such estimate of how the system is frenzy, or ``frenetic'', is necessary 
to understand nonequilibrium environments in general. In the context of response theory one thus needs to know
how a system, upon perturbation, becomes more or less active in its dynamics. This might involve needing
to measure delicate microscopic details, which could not be easily accessible experimentally. Hence, on the practical
side we have the important task to adapt the theoretical framework to the available experimental capabilities.





\end{wpdescription}

\begin{tasklist}

  \begin{task}[title=TASK1,id=task1,PM=15,lead=Padova,wphases=0-30!0.5]

    First task of the project: 
    for models of solids between two plates at different temperatures, 
    thus experiencing a heat flow, we plan to characterize the response to compression, mostly by applying 
    available formulations to the study of numerical simulations. The main aim is to make a setup for the
    other tasks.
    
  \end{task}

  \begin{task}[title=TASK2,id=task2,PM=15,lead=PAR2,wphases=12-42!0.5]

    Second task of the project: 
    we are interested in finding if microscopic details, available of course in simulations, 
    are needed or if a more experimental-like coarse grained description is sufficient or even convenient for
    predicting the response of the system's length to an increased compression. In practice, for example,
    by running a simulation of coupled oscillators we assume that only ``macroscopic'' quantities as the system length
    are measurable and from there we try to see if a coarse-grained fluctuation-response approach is able to
    describe the behavior of the system when a compression is added.
    
  \end{task}

  \begin{task}[title=TASK3,id=task3,PM=15,lead=Padova,wphases=0-30!0.5]

    Third task of the project: 
    we aim at finding peculiar inter-particle potentials that may lead to unusual
    compressibility in systems experiencing heat flows. The possibility to extract energy
    from an installed heat channel might in fact lead to the behavior of negative differential compressibility,
    i.e.~the system expands when compressed. 
    Though we plan to mainly operate at the conceptual level of toy
    models, a successful achievement on this direction might trigger technological applications with immense
    relevance. Metamaterials are indeed nowadays much studied because, for example, micro-actuators would be
    more easily realized if materials with a negative compressibility were introduced. We will investigate
    if steady nonequilibrium conditions could produce this effect, as opposed to the recently considered 
    mechanism of rearrangements from metastable states. It is not easy to predict the success of this line,
    which is thus a high-risk high reward task of this WP.
    
  \end{task}

  \begin{task}[title=TASK4,id=task4,PM=15,lead=Padova,wphases=0-30!0.5]

    {\bf [THIS MIGHT BE SHIFTED TO OTHER COLLABORATIONS]}
    Fourth task of the project: 
    understanding how active systems do respond to compression is pretty much
    synonymous of assessing whether state equations relating pressure to temperature, volume, and internal
    interaction parameters, are available for active matter. We aim at studying the compressibility of
    models of bacterial colonies, both from the theoretical side and by analyzing simulations of active matter.
    
  \end{task}


\end{tasklist}

\begin{wpdelivs}
  \begin{wpdeliv}[due=12,id=mydeliv1,dissem=PU,nature=DEM,lead=Padova]
      {First deliverable, after 1 year.}
  \end{wpdeliv}
  \begin{wpdeliv}[due=24,id=mydeliv2,dissem=PU,nature=DEM,lead=PAR2]
      {Second deliverable, after 2 years.}
\end{wpdeliv}
\end{wpdelivs}

\end{workpackage}
