\begin{workpackage}[id=WPactive,wphases=0-48,
  short=Active Particle Suspensions,% for Figure 5.
  title=Probing Active Particle Suspensions with Colloids and Polymers,
  lead=Leipzig,
  partners={Padova,USTUTT},
  LeipzigRM=96,PadovaRM=6,USTUTT=2]

\begin{wpobjectives}
The objectives of this WP are:
  \begin{compactitem}
  \item \textbf{Non-Equilibrium Equations of State (NEOS):} explore the existence of
  macroscopic non-equilibrium equations of state for active Brownian particle suspensions
  with microscopic many-particle models
  \item \textbf{Active Crowding:} elucidate by analytical model calculations how mechanical
  and chemical activity (such as by molecular motors in the cytoplasm of animal cells)
  affects the statistical mechanics of embedded biopolymers (such as protein fibers or DNA)
  \item \textbf{Noise Temperature:} validate the concept of a frequency-dependent noise
  temperature in non-equilibrium fluctuation-dissipation relations and the failure of
  Onsager's regression hypothesis for colloidal probes in non-isothermal solvents by
  experiments and computer simulations
  \item \textbf{Active-Particle Suspensions:} scrutinize tentative non-equilibrium equations
  of state by a combination of experiments and all-atom non-equilibrium molecular-dynamics
  simulations
  \end{compactitem}
\end{wpobjectives}

\begin{wpdescription}
In equilibrium, information about the macroscopic state of a medium can be deduced via
rather simple measurement procedures using simple probes and then be summarized in an
equation of state.
%
In the future, something similar may become possible even far from equilibrium, and
revolutionize our technological abilities, but this will require much more sophisticated
probing and analyzing procedures that are not yet generally understood.
%
Far from equilibrium, fluctuations are non-universal and standard thermodynamic notions,
such as temperature and pressure, become ambiguous.
%
This has many unintuitive practical consequences, which defy conventional thermodynamic
wisdom and potentially invalidate and doom to failure well-established standard
technological procedures.

For example, a colloidal particle in a non-isothermal solvent cannot be used for Brownian thermometry 
in the conventional way \cite{rings-etal:2010,kroy:2014}, and also the proper theoretical 
treatment of its systematic, thermophoretic ``swimming'' is the subject of ongoing debates.
%
We have recently shown how to overcome some of these limitations in terms of a frequency-dependent noise 
temperature \cite{falasco-etal:2014}.
%
We want to validate this result experimentally and to see to what extent something similar be achieved for other 
non-equilibria, particularly in active-particle suspensions.

Further, far from equilibrium, the mechanical pressure on a mesoscopic probe will generally hinge on the range and 
symmetry of the medium-probe interactions: if an active-particle suspension is compressed by a piston, the
amount of work needed to reach a prescribed amount of compression depends on the type of
piston used (e.g.\ how hard or soft its surface is), and the pressure can be macroscopically
anisotropic and heterogeneous in absence of macroscopic stationary fluxes. 
%
We want to elucidate this within a many-body theory of ``dry active matter'' \cite{marchetti-etal:2013}, 
to which we successively add elements of "wet reality".

Similarly, the myriads of motile proteins and motors in a living cell cause deviations of the function
of its constituents with respect to conventional equilibrium predictions.  We call this
effect ``active crowding'', in analogy with the much studied ``molecular crowding''. 
%
Any biopolymer in a living cell actually represents a "smart" probe that scans its environment with its many 
internal degrees of freedom \cite{otto-etal:2013}.
%
We could recently demonstrate by theory and simulations that the effect of a quenched static crowding onto an embedded
semiflexible polymer can effectively be captured by a relatively simple parameter renormalization \cite{schoebl-etal:2014}.  
%
Here we ask: can something similar be achieved for active crowding?

The above open issues severely limit a first-principles approach to living systems and
innovative technological applications involving active matter.
%
We address them as systematically as we can (though not comprehensively) with the above
objectives 1 \& 2 (theoretically) and 3 \& 4 (experimentally and by massively parallel
computer simulations).

\end{wpdescription}

\begin{tasklist}

\begin{task}[title=Non-Equilibrium Equations of State (NEOS),id=task1,PM=8,lead=Leipzig,partners={Padova,USTUTT},wphases=0-48!0.3]
As a paradigm for a non-equilibrium medium we consider the (experimentally vast) class of
active-particle suspensions, in collaboration with the group in Padova providing expert knowledge on 
non-equilibrium response theory.
%
Our starting point is a many-body Langevin equation with a Markovian Gaussian noise
(describing the coupling to an equilibrium heat bath) and individual propulsion forces for
particles representing, on an all-atom-level, the "dry" active fluid far from equilibrium \cite{solon-etal.2015}. 
%
We let the fluid interact perturbatively with one large colloidal probe particle to
deduce its effective friction kernel, and with two or more probe particles in order to
detect their induced mutual interactions, non-gradient forces, and a possible breaking of
the action-reaction principle on the coarse-grained probe level.  
%
Non-equilibrium response theory lets us, in particular, include frenetic contributions in the response functions.
%
Of special interest to us is the mechanical pressure exerted by an active fluid on a probe, 
in particular if we relax the very special geometric symmetries tacitly imposed in previous approaches.

To add elements of reality, we then introduce more realistic particle-probe interactions, as appropriate e.g.\ for the hot 
Janus particles that we sudy in our experiments and simulations.
%
This should in particular help to clarify, whether and how active-particle suspensions are fundamentally different
from externally driven systems, a question we want to elucidate in collaboration our colleagues from Stuttgart 
by searching the connection between our particle-based approach with their field theoretical approach.

\end{task}

\begin{task}[title=Active Crowding,id=task2,lead=Leipzig,wphases=0-48!0.2]
Pioneering attempts to estimate how the activity in a living cell affects the physical behavior
and function of its constituents predict that activity will, via hydrodynamic interactions, enhance the diffusion of
passive globular probes or cell organelles \cite{mikhailov-kapral:2015}.
%
Yet, it remains open, how such partial results get modified if one considers, for example, that motors or
other sources of non-equilibrium do often act directly on cytoskeletal polymers and their
weak transient networks that may get disrupted.
%
In particular, we want to consider how three different types of active
crowding affect the fluctuation spectrum and dynamics of a semiflexible test polymer: a
laser-heated metal nanoparticle tracer attached to a micron-sized semiflexible polymer such
as an actin microfilament or a microtubulus; power strokes exerted by attached molecular
motors and inducing backbone tension spikes along the polymer contour; ``inelastic''
unbinding of weak bonds along the test polymer, which are the ubiquitous cement of
cytoskeletal structures, under static or dynamic loads.
%
This may have interesting conceptual consequences for mechanical homeostasis, i.e., how cells and tissues bear up against 
their own activity.   
%
On the technological side, it would be most useful to find conditions for formal mapping to the simpler situation in 
non-isothermal simple media, via frequency-dependent noise temperatures and effective friction coefficients.
%
For these tasks we can rely on our strong track record of first-principle analytical studies
of the non-linear non-equilibrium dynamics of biopolymers \cite{otto-etal:2013}, non-quasistatic bond kinetics
under strong external forcing \cite{bullerjahn-sturm-kroy:2014}, and hot Brownian motion \cite{rings-etal:2010}.

\end{task}


\begin{task}[title=Noise Temperature,id=task3,lead=Leipzig,wphases=0-24!0.5]
We have recently established a tractable ``drosophila'' for the central theme of the
proposal, i.e.\ probing macroscopic non-equilibria, in terms so-called non-isothermal (or
``hot'') Brownian motion \cite{rings-etal:2010,chakraborty-etal:2011}.
%
Its key quantity to account for the effect of the non-equilibrium medium on the probe is a frequency-dependent
temperature for the Brownian noise strength (``noise temperature'') \cite{falasco-etal:2014}.
%
Corner stones of statistical mechanics, such as the fluctuation-dissipation theorem, can
thereby be generalized to conditions far from equilibrium.  

To test these predictions, which are relevant to metal-nanoparticle tracking and trapping technologies and emerging
nanoscopic Brownian heat engines, beyond the easily accessible Markov limit, requires to follow the fluctuations of 
a heated Brownian particle on a nanosecond timescale with a positional accuracy of picometers and below, a task 
that has, even under conventional \emph{equilibrium} conditions, only very recently been achieved \cite{kheifets-etal:2014}.
%
A dedicated optical trap with dual beam heating and a detection bandwidth of about 50 MHz
will therefore be newly constructed.

To test the theory also by massively parallel numerical simulations \cite{chakraborty-etal:2011}, 
we will extend the theory for incompressible fluids 
to Lennard--Jones solvents, which exhibit a substantial compressibility.
\end{task}

\begin{task}[title=Active-Particle Suspensions,id=task4,lead=Leipzig,wphases=24-48!0.5]
Technological applications of active-particle suspensions and systematic tests of
uncontrolled approximations that are currently unavoidable in all theoretical studies are
currently severely hampered by a lack of control over these systems.
%
Therefore, dedicated experimental investigations are still very rare and some of their
results might be specific to the experimental realization of the self-propulsion mechanism.
%
We have recently established innovative force-free control systems, so-called photon nudging
and thermophoretic trapping technologies \cite{Qian2013,Braun:NanoLetters:2015}, which deliberately exploit 
the non-equilibrium transport caused by non-isothermal conditions.
%
Thereby we can e.g.\ construct soft walls of passive or active particles that act as
semi-permeable membranes.
%
Our experiments and accompanying "all-atom" Lennard--Jones molecular-dynamics simulations \cite{chakraborty-etal:2011} 
will employ hot Janus particles, which allow for a continuous tuning of the strength of 
the activity in space and time, to target 
possible failures of the description by a thermodynamic bulk-pressure equation of
state \cite{ginot-etal:2015} due to wall interactions and ensuing "ratcheting" effects.
\end{task}


\end{tasklist}

\begin{wpdelivs}
\begin{wpdeliv}[due=12,id=D2.3,dissem=PU,nature=DEM,lead=Leipzig]
      {Experimental set-up for frequency-dependent temperature measurement}
  \end{wpdeliv}
  \begin{wpdeliv}[due=24,id=D2.1,dissem=PU,nature=DEM,lead=Leipzig]
      {Theoretical set-up for molecular relaxation dynamics in active crowding}
  \end{wpdeliv}
  \begin{wpdeliv}[due=24,id=D2.2,dissem=PU,nature=DEM,lead=Leipzig]
      {Theoretical and numerical models for NEOS of active matter}
\end{wpdeliv}
  \begin{wpdeliv}[due=36,id=D2.4,dissem=PU,nature=DEM,lead=Leipzig]
      {Experimental setup for NEOS-test in active particle suspensions}
\end{wpdeliv}
\end{wpdelivs}

\end{workpackage}
