\begin{workpackage}[id=WPactive,wphases=0-48,
  short=Active Particle Suspensions,% for Figure 5.
  title=Probing Active Particle Suspensions with Colloids and Polymers,
  lead=Leipzig,
  LeipzigRM=12]

\begin{wpobjectives}
The objectives of this WP are:
  \begin{compactitem}
  \item \textbf{Non-Equilibrium Equations of State (NEOS):} explore the existence of
  macroscopic non-equilibrium equations of state for active Brownian particle suspensions
  with microscopic many-particle models
  \item \textbf{Active Crowding:} elucidate by analytical model calculations how mechanical
  and chemical activity (such as by molecular motors in the cytoplasm of animal cells)
  affects the statistical mechanics of embedded biopolymers (such as protein fibers or DNA)
  \item \textbf{Noise Temperature:} validate the concept of a frequency-dependent noise
  temperature in non-equilibrium fluctuation-dissipation relations and the failure of
  Onsager's regression hypothesis for colloidal probes in non-isothermal solvents by
  experiments and computer simulations
  \item \textbf{Active-Particle Suspensions:} scrutinize tentative non-equilibrium equations
  of state by a combination of experiments and all-atom non-equilibrium molecular-dynamics
  simulations
  \end{compactitem}
\end{wpobjectives}

\begin{wpdescription}
In equilibrium, information about the macroscopic state of a medium can be deduced via
rather simple measurement procedures using simple probes and then be summarized in an
equation of state.
%
In the future, something similar may become possible even far from equilibrium, and
revolutionize our technological abilities, but this will require much more sophisticated
probing and analyzing procedures that are not yet generally understood.
%
Far from equilibrium, fluctuations are non-universal and standard thermodynamic notions,
such as temperature and pressure, become ambiguous.
%
This has many unintuitive practical consequences, which defy conventional thermodynamic
wisdom and potentially invalidate and doom to failure well-established standard
technological procedures.

For example, a colloidal particle in a non-isothermal solvent cannot be used for Brownian thermometry in the conventional way, and also the proper theoretical treatment of its systematic, thermophoretic ``swimming'' is the subject of ongoing debates.
%
Also, the mechanical pressure will generally hinge on the range and symmetry of the
particle-probe interactions: if an active-particle suspension is compressed by a piston, the
amount of work needed to reach a prescribed amount of compression depends on the type of
piston used (e.g.\ how hard or soft its surface is), and the pressure can be macroscopically
anisotropic and heterogeneous in absence of macroscopic stationary fluxes.

Similarly, any biopolymer in a living cell actually represents a complex (or smart) probe
that scans its non-equilibrium mesoscopic environment with its many internal degrees of
freedom.
%
The myriads of motile proteins and motors in a living cell cause deviations of the function
of its constituents with respect to conventional equilibrium predictions.  We call this
effect ``active crowding'', in analogy with the much studied ``molecular crowding''.
%
We could recently quantitatively demonstrate by theory and simulations that the effect of a
quenched static crowding onto an embedded semiflexible polymer can effectively be captured
by a relatively simple parameter renormalization.  Here we ask: can something similar be
achieved for active crowding?

There is currently no consensus on how the fundamental obstacles to the formulation of
non-equilibrium equations of state can systematically be overcome, as underscored by a
number of research articles published this year, which make claims towards either the
existence or non-existence of NEOS for active-particle suspensions.
%
These complications severely limit a first-principles approach to living systems and
technological applications.
%
Serious conceptual problems need to be solved before, say, active particle suspension will
become the basis for new materials with built-in actuation mechanisms, for fluid mixing,
viscosity reduction, or shifting the glass transition in industrial processing, or for
noninvasive surgeries and targeted drug delivery.
%
We address them as systematically as we can (though not comprehensively) with the above
objectives 1 \& 2 (theoretically) and 3 \& 4 (experimentally and by massively parallel
computer simulations).

\end{wpdescription}

\begin{tasklist}

\begin{task}[title=Non-Equilibrium Equations of State (NEOS),id=task1,PM=15,lead=Leipzig,wphases=0-30!0.5]
As a paradigm for a non-equilibrium medium we consider the (experimentally vast) class of
active-particle suspensions.
%
Currently studied microscopic models all have special explicit, and often additional
implicit symmetries and moreover violate fundamental physical principles, such as momentum
conservation.
%
They are therefore sometimes characterized as ``dry active matter'', as opposed to (usually)
more realistic ``wet active matter''.  In our approach, we follow this common procedure (as
e.g.\ in actual papers by Solon et al.) with the aim to successively add elements of ``wet
reality'' later on.
%
Our starting point is a many-body Langevin equation with a Markovian Gaussian noise
(describing the coupling to an equilibrium heat bath) and individual propulsion forces for
the particles representing, on an all-atom-level, the active fluid far from equilibrium.
%
We then let the fluid interact perturbatively with one large colloidal probe particle to
deduce its effective friction kernel, and with two or more probe particles in order to
detect their induced mutual interactions, non-gradient forces, and a possible breaking of
the action-reaction principle on the coarse-grained probe level.  Non-equilibrium response
theory lets us, in particular, include frenetic contributions in the response functions.
%
Of particular interest to us is the mechanical pressure exerted by an active fluid on a
probe, in particular if we relax the very special geometric symmetries tacitly imposed in
previous approaches.
%
To add elements of ``wet reality'', we will further account for more realistic
particle-probe interactions, as appropriate e.g.\ for hot Janus particles.
%
The required ingredients are currently being developed by us, using hydrodynamic theory,
simulations, and experiments, in an ongoing joint project within the German priority program
``Microswimmers''.

It is to date still not clear, whether systems in which the macroscopic non-equilibrium is
caused by self-propulsion of the fluid particles are fundamentally different from those
exposed to an external driving, e.g.\ by shearing or laser heating.
%
We plan to elucidate this question in collaboration with our theoretical colleagues from the
node in Stuttgart by establishing a connection of our particle-based approach with their
field theoretical approach.

\end{task}

\begin{task}[title=Active Crowding,id=task2,PM=15,lead=Leipzig,wphases=12-42!0.5]
A first attempt to estimate how the activity in a living cell affects the physical behavior
and function of its constituents was put forward by Mikhailov and Kapral in a paper in PNAS,
earlier this year.
%
They predict that activity will, via hydrodynamic interactions, enhance the diffusion of
passive globular probes or cell organelles.
%
Yet, it remains open, whether such perturbative results get enhanced (as in equilibrium) or
largely diminished on a non-perturbative level, if one considers, for example, that the
active stirring may affect the effective friction of a probe particle, or that motors or
other sources of non-equilibrium do often act directly on cytoskeletal polymers or their
weak transient networks.
%
Armed with our previous experience with the derivation of effective friction coefficients
under far-from-equilibrium conditions, we therefore aim to find out how the activity affects
the effective non-equilibrium friction of a probe, and whether it enhances or counteracts
the perturbatively predicted diffusivity.
%
Further, we want to work out, from first principles, how three different types of active
crowding affect the fluctuation spectrum and dynamics of a semiflexible test polymer: a
laser-heated metal nanoparticle tracer attached to a micron-sized semiflexible polymer such
as an actin microfilament or a microtubulus; power strokes exerted by attached molecular
motors and inducing backbone tension spikes along the polymer contour; ``inelastic''
unbinding of weak bonds along the test polymer, which are the ubiquitous cement of
cytoskeletal structures, under static or dynamic loads.
%
For these tasks we can rely on our strong track record of first-principle analytical studies
of the non-linear non-equilibrium dynamics of biopolymers, non-quasistatic bond kinetics
under strong external forcing, and hot Brownian motion.
%
Think for example how the free recoil of a stretched DNA molecule, which is solely driven by
fluctuations (and for which we have recently demonstrated a perfect parameter-free match of
theory and experiment under isothermal conditions) is affected by the various types of
active excitations.
%
Finally, it will be of major interest to clarify to what extent our models of active
crowding admit a formal mapping to the situation in non-isothermal simple solvents, via
frequency-dependent noise temperatures and effective friction coefficients.
\end{task}


\begin{task}[title=Noise Temperature,id=task2,PM=15,lead=Leipzig,wphases=12-42!0.5]
We have recently established a tractable ``drosophila'' for the central theme of the
proposal, i.e.\ probing macroscopic non-equilibria, in terms so-called non-isothermal (or
``hot'') Brownian motion.
%
Using analytical calculations, we have demonstrated that the phenomenon is amenable to an
analytical framework that extends conventional equilibrium statistical mechanics formalisms
by introducing a generalized notion of temperature, chiefly a frequency-dependent
temperature for the Brownian noise strength (``noise temperature'').
%
Corner stones of statistical mechanics, such as the fluctuation-dissipation theorem, can
thereby be generalized to conditions far from equilibrium.  However, these results, which
are relevant to metal-nanoparticle tracking and trapping technologies and emerging
nanoscopic Brownian heat engines, could so far not be validated by experiments and
microscopic computer simulations, beyond the easily accessible Markov limit.
%
We therefore propose to develop a high-speed experimental measurement setup capable of
detecting inertial effects in the Brownian motion of hot metal nanoparticles and thereby to
detect the predicted increase of the apparent Brownian temperature of their short-time
dynamics.
%
This requires to follow the fluctuations of a heated Brownian particle on a nanosecond
timescale with a positional accuracy of picometers and below to measure the (modified)
Maxwell velocity distribution, a task that has only very recently been achieved under
equilibrium conditions, using latest optical-tweezers technology.
%
Extending the method to conditions far from equilibrium requires substantial additional
technological developments, e.g.\ due to heating effects and radiation pressure.
%
A dedicated optical trap with dual beam heating and a detection bandwidth of about 50 MHz
will therefore be constructed.

Further, to test the theory also by numerical simulations, we will quantify the deviations
from the idealized theory developed for incompressible fluids to Lennard--Jones solvents, as
used in our efficient massively parallel computer simulations, which exhibit a substantial
compressibility.
%
A successful comparison of the results from non-equilibrium molecular-dynamics simulations
and theory should also provide a template for analyzing future experiments realizing extreme
probe temperatures.
\end{task}

\begin{task}[title=Active-Particle Suspensions,id=task2,PM=15,lead=Leipzig,wphases=12-42!0.5]
Technological applications of active-particle suspensions and systematic tests of
uncontrolled approximations that are currently unavoidable in all theoretical studies are
currently severely hampered by a lack of control over these systems.
%
Therefore, dedicated experimental investigations are still very rare and some of their
results might be specific to the experimental realization of the self-propulsion mechanism.
We have recently established innovative force-free control systems, so-called photon nudging
and thermophoretic trapping technologies, which deliberately exploit the non-equilibrium
transport caused by non-isothermal conditions.
%
Thereby we can e.g.\ construct soft walls of passive or active particles that act as
semi-permeable membranes.
%
In combination with conventional manipulation tools such as structured steric confinements,
light forces, electromagnetic fields, and gravitation, these enable us to realize more
ambitious and better controlled setups involving self-propelled particles than hitherto
possible.
%
We will predominantly use hot Janus particles in water controlled by laser heating, which
allows for a continuous tuning of the strength of the activity in space and time.
%
While recent experimental work with a highly symmetric active-particle suspension setup has
suggested that activity can be accounted for in the thermodynamic bulk-pressure equation of
state by a renormalized temperature and renormalized interactions, we want to more
specifically target possible failures of this scheme.
%
Using a variety of designs of potential confinement together with our force-free control
technology, we will elucidate to what extent wall interactions (rather than the intrinsic
asymmetry of the propulsion mechanism) are the fundamental source of time-reversal asymmetry
at the coarse-grained level, and thus relevant for the breaking of detailed balance and
ensuing rectification (``ratchet'') effects.
%
Thereby, and by accompanying numerical simulations, we aim to detect discrepancies between
the thermodynamic bulk pressure and the actual mechanical wall pressure.
\end{task}


\end{tasklist}

\begin{wpdelivs}
  \begin{wpdeliv}[due=12,id=mydeliv1,dissem=PU,nature=DEM,lead=Leipzig]
      {First deliverable, after 1 year.}
  \end{wpdeliv}
  \begin{wpdeliv}[due=24,id=mydeliv2,dissem=PU,nature=DEM,lead=Leipzig]
      {Second deliverable, after 2 years.}
\end{wpdeliv}
\end{wpdelivs}

\end{workpackage}
