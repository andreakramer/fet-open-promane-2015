\begin{workpackage}[id=WPactive,wphases=0-48,
  short=Active Particle Suspensions,% for Figure 5.
  title=Probing active particle suspensions with colloids and polymers,
  lead=ULEI,
  ULEIRM=96,UNIPDRM=6,USTUTTRM=2]

\newrefsection

\begin{wpobjectives}
  \begin{compactitem}
  \item Derive \textbf{nonequilibrium equations of state (NEOS)} for active Brownian
  particle suspensions from microscopic many-particle models.
  \item Characterize the motion of embedded biopolymers in situations of \textbf{active
    crowding} by molecular motors in the cytoplasm of animal cells.
  \item Validate the concept of a frequency-dependent \textbf{noise temperature} in
  fluctuation-dissipation relations for colloidal probes.
  \end{compactitem}
\end{wpobjectives}

\begin{wpdescription}

Far from equilibrium, fluctuations are non-universal and standard thermodynamic notions,
such as temperature and pressure, become ambiguous.
This has many unintuitive practical consequences, which defy conventional thermodynamic
wisdom and potentially invalidate standard technological procedures.
%
For example, a colloidal particle in a non-isothermal solvent cannot be used for Brownian thermometry 
in the conventional way \cite{rings-etal:2010,kroy:2014}. In theory, we have shown how to overcome this problem
in terms of a frequency-dependent noise temperature \cite{falasco-etal:2014}.
%
We will validate these ideas experimentally and extend them for other
nonequilibria, e.g. in active-particle suspensions.
%
Further, far from equilibrium, the mechanical pressure on a mesoscopic probe will generally depend on the range and 
symmetry of the medium-probe interactions. 
%
In the context of living cells, motile proteins and motors affect the function
of its constituents by what  we call ``active crowding'', in analogy with the much studied ``molecular crowding''. 
%
Any biopolymer in a living cell actually represents a smart probe that scans its environment with its many 
degrees of freedom \cite{otto-etal:2013}.
%
We could account for the effect of quenched static crowding by a parameter renormalization \cite{schoebl-etal:2014}.  
%
Here we ask: can something similar be achieved for active crowding?
\end{wpdescription}

\begin{tasklist}

\begin{task}[title=Nonequilibrium equations of state (NEOS),id=task1,PM=8,lead=ULEI,partners={UNIPD,USTUTT},
wphases=0-48!0.5]
Design and characterize a model for active-particle suspensions, together with the group
in Padova (exchange of personel) that will provide essential expert knowledge on
nonequilibrium response theory.
%
We start with a many-body Langevin equation, with individual propulsion forces for the
active particles \cite{solon-etal:2015}, and compute for colloidal probes the effective
friction kernels, induced mutual interactions and non-gradient forces.
%
Nonequilibrium response theory (as theoretically developed in the Leuven-Padova nodes) lets us include 
frenetic contributions in the response functions \cite{baiesi-wynants:2009} and in the mechanical pressure 
exerted by an active fluid on a probe. 
%
Geometric symmetries (tacitly) imposed in previous approaches will be relaxed and realistic particle-probe interactions, 
as appropriate for the hot Janus particles studied in our experiments and simulations, will be introduced.
%
To clarify how active-particle suspensions are fundamentally different from externally driven systems,
we will compare our particle-based approach to the field theoretical approach pursued in Stuttgart (exchange of personel). 

\end{task}

\begin{task}[title=Active Crowding,id=task2,lead=ULEI,partners={KUL},wphases=0-48!0.5]
We investigate via experiments three different types of active crowding that affect the fluctuation spectrum and dynamics of
a semiflexible test polymer \cite{otto-etal:2013}: an attached laser-heated metal nanoparticle tracer, power strokes of
molecular motors inducing backbone tension spikes and ``inelastic'' \cite{gralka-kroy:2015} unbinding of weak (cytoskeletal)
bonds under arbitrarily fast dynamic loading \cite{bullerjahn-sturm-kroy:2014}.
\end{task}

\begin{task}[title=Noise Temperature,id=task3,lead=ULEI,wphases=0-24!0.5]
We test the recent predictions of non-isothermal Brownian motion \cite{rings-etal:2010,falasco-etal:2014}, 
beyond the Markov limit. Therefore, we need to adapt the theory 
to our massively parallel numerical simulations \cite{chakraborty-etal:2011}, 
which use (highly compressible) Lennard--Jones solvents.
Experimentally, this requires to follow the fluctuations of 
a heated Brownian particle on a nanosecond timescale and picometer spatial resolution, which has
\emph{even in equilibrium} only recently become possible  \cite{kheifets-etal:2014}.
%
A dedicated optical trap with dual beam heating and a detection bandwidth of about 50 MHz
will therefore be newly constructed. 
%
The classical method of Brownian thermometry will hereby be extended to media far from equilibrium.
\end{task}

\begin{task}[title=Active-Particle Suspensions,id=task4,lead=ULEI,wphases=24-48!0.5]
Technological applications of active-particle suspensions  are
currently severely hampered by a lack of control over these systems.
%
We have recently established innovative force-free control systems, so-called photon nudging
and thermophoretic trapping technologies \cite{Qian2013,Braun:NanoLetters:2015}, which deliberately exploit 
the nonequilibrium transport caused by non-isothermal conditions.
%
Using them in our experiments with hot Janus particles, and accompanying "all-atom" molecular-dynamics 
simulations \cite{chakraborty-etal:2011}, will allow for a continuous tuning of the strength of 
the activity and confinement to target failures of thermodynamic bulk-pressure equations of
state \cite{ginot-etal:2015}.
\end{task}


\end{tasklist}

\printbibliography[heading=proposal-bib,env=proposal-env]

\begin{wpdelivs}
\begin{wpdeliv}[due=12,id=D2.3,dissem=PU,nature=R,lead=ULEI]
      {Set-up for measuring noise-temperature}
  \end{wpdeliv}
  \begin{wpdeliv}[due=24,id=D2.1,dissem=PU,nature=R,lead=ULEI]
      {Set-up for dynamics in active crowding}
  \end{wpdeliv}
  \begin{wpdeliv}[due=24,id=D2.2,dissem=PU,nature=R,lead=ULEI]
      {Set up models for NEOS of active matter}
\end{wpdeliv}
  \begin{wpdeliv}[due=36,id=D2.4,dissem=PU,nature=R,lead=ULEI]
      {Set up experiment for NEOS test}
\end{wpdeliv}
\end{wpdelivs}

\end{workpackage}
