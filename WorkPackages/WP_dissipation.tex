\begin{workpackage}[id=WPkinetic,wphases=0-48,
short=Dissipation,
title=Harnessing dissipation,
lead=TUE,
TUERM=36]

\begin{wpobjectives}
% Objectives of the work package; 5-10 lignes, typically itemized
The objective of this work package is to:
\begin{compactitem}
\item Investigate how dissipation can be harnessed to improve the information-processing properties of non-equilibrium probes.
\end{compactitem}
\end{wpobjectives}

\begin{wpdescription}
  % Overall description; typically 10 lines to half a page
  % as appropriate, depending on the variety and number of tasks
Recently it has become recognized that living cells make use of energy dissipation to reduce noise, increase sensitivity, and generally improve their information-processing capabilities. This is a non-trivial feat, since dissipation typically destroys correlations and eliminates information. Man-made microprobes and bioprobes could greatly benefit from this same possibility, provided we can understand and tune it appropriately. In addition, dissipation is energetically wasteful, which constitutes a problem for autonomous sensors. Careful control is necessary. 

In this project we therefore investigate the mechanisms and principles that underlie the beneficial sides of the usually harmful dissipation. The focus will be on extracting from examples the general guiding principles, allowing us to leverage this understanding for new applications.

The work in this package will be possible by a combination of ingredients. We will leverage the link we recently discovered (ref) between large deviations and gradient flows to connect the macroscopic dissipation with microscopic features of the underlying dynamics. In addition, recent developments in chemical network theory give a tighter connection between stochastic dynamics and dissipation (ref). The work will also greatly benefit from the expertise of Leuven on non-equilibrium statistical physics and of Stuttgart on coarse-graining. 
Finally, the group in Eindhoven has an ongoing collaboration with an experimental group at the Institute for Complex Molecular Systems in Eindhoven, headed by Tom de Greef. This group focuses on bottom-up synthetic biology, and can provide essential chemical and biological insight. 

%
% test comments
%

% Sykes, A Guidebook to Mechanism in Organic Chemistry
\end{wpdescription}

% Please see UserInterfaces.tex for now as an example

\begin{tasklist}
  % 3-5 tasks

  % The description of each task can be 5 to 15 lines depending on the
  % complexity and amount of details deemed necessary, and involve and
  % refer to 1-3 deliverables.

  \begin{task}[title=Dissipation in the bistable switch]
  We first focus on a single information-processing step, the bistable switch (a flip-flop), which is a core element of silicon-based computing, and which also exists in chemical form. For single examples the relation between dissipation and performance has been studied (refs). In this task we generalize from examples to a general principle that explains the underlying reasons.
  \end{task}

  \begin{task}[title=Role of dissipation in time]
  We next include time. An auto-nulling amplifier is a simple input-output information-processing unit, that temporarily amplifies differences in the input, and resets itself after some time. In earlier work we showed that such an object can be chemically constructed completely without dissipation, but only in a zero-noise context. We now include noise, and study how dissipation can be used to reduce the noise and improve the gain.  \end{task}
  
  \begin{task}[title=Dissipation in an oscillator]
A chemical oscillator is an intrinsically non-equilibrium object, that processes information by reacting in period and phase to external input. Dissipation is essential for the oscillator itself, and in this task we first study the role and limits provided by dissipation on the oscillator behaviour. We then continue with the effect of external inputs, and the role of dissipation in controlling this effect.    \end{task}
  

\end{tasklist}

\eucommentary{ Deliverable numbers in order of delivery
  dates. Please use the numbering convention ``WP number''.``number of
  deliverable within that WP''.  For example, deliverable 4.2 would
  be the second deliverable from work package 4.
%
  Type:
  Use one of the following codes:
  R: Document, report (excluding the periodic and final reports)
  DEM: Demonstrator, pilot, prototype, plan designs
  DEC: Websites, patents filing, press \& media actions, videos, etc.
  OTHER: Software, technical diagram, etc.
  Dissemination level:
  Use one of the following codes:
  PU = Public, fully open, e.g. web
  CO = Confidential, restricted under conditions set out in Model Grant Agreement
  CI = Classified, information as referred to in Commission Decision 2001/844/EC.
  Delivery date
  Measured in months from the project start date (month 1)
}
\begin{wpdelivs}
  \begin{wpdeliv}[due=24,id=wp-kincon-1,dissem=PU,nature=R,lead=TUE]
    {One line deliverable description.}
  \end{wpdeliv}
  \begin{wpdeliv}[due=48,id=wp-kincon-2,dissem=PU,nature=DEM,lead=TUE]
    {One line deliverable description.}
  \end{wpdeliv}
\end{wpdelivs}
\end{workpackage}

%%% Local Variables:
%%% mode: latex
%%% TeX-master: "../proposal"
%%% End:
