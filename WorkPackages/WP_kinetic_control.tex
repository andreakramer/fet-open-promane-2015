\begin{workpackage}[id=WPkinetic,wphases=0-48,
short=Kin. control,
title=Kinetic control,
lead=TUE,
TUERM=36]

\begin{wpobjectives}
% Objectives of the work package; 5-10 lignes, typically itemized
The objectives of this work package are to:
\begin{compactitem}
\item Establish the extent of \textbf{kinetic control} in collective dislocation motion
\item Understand the origin of \textbf{nonlinear flow laws} in dislocation interaction
\item Leverage this understanding to design \textbf{better forming processes}.
\end{compactitem}
\end{wpobjectives}

\begin{wpdescription}
  % Overall description; typically 10 lines to half a page
  % as appropriate, depending on the variety and number of tasks
Many non-equilibrium systems evolve under \emph{kinetic control:} their evolution at
laboratory time scales is determined as much by kinetic effects as by thermodynamic
forces. Examples are glasses at moderate temperatures, various chemical reactions
(e.g.~\cite{Sykes}) and dislocations in crystals. This is an intrinsically non-equilibrium
phenomenon, which interferes with the design and function of devices that interact with the
system, sometimes to the extent of completely preventing any normal function.
  
In this work package we investigate this phenomenon on a simple, but paradigmatic example:
the movement of dislocations in metal crystals. This system was the focus of a recent joint
TUe Mechanical-Engineering-Mathematics research effort, focussing on the simplest nontrivial
case, that of parallel edge dislocations, and leading to two
theses~\cite{VanMeurs15TH,Kooiman15TH}. Among other results, these two theses clearly
indicate the kinetic control of this system, through \emph{configurational temperature} as
in glasses~\cite[Ch.~2--4]{Kooiman15TH}, \emph{emergent nonlinear flow
rules}~\cite[Ch.~6]{Kooiman15TH}, and \emph{propagation
failure}~\cite[Ch.~9]{VanMeurs15TH}.

\textbf{Configurational temperature} is the phenomenon that a system has a much wider distribution over state space than the  actual, physical temperature can explain; by raising the temperature in the Boltzmann distribution to a higher, \emph{configurational}, temperature, the distribution better describes observed behaviour. Configurational temperature is an intrinsically non-equilibrium concept, for which a good theoretical treatment is still completely lacking. 

Plastic (permanent) deformation in metals macroscopically follows a \textbf{nonlinear flow rule}, which is at odds with the microscopic, linear flow behaviour of the underlying dislocations. How the nonlinearity arises in the micro-to-macro upscaling is not yet understood. The numerically observed \textbf{failure of propagation} of dislocations is expected to play a role in this.

A long-term technical application is to leverage the microscopic understanding of dislocation motion for the creation of \textbf{designer metals} with a priori specified deformation properties. Design of highly specified metals is currently a trial-and-error process, and we will show a proof-of-concept that  microscopic insight leads to improved design.


%
% test comments
%

% Sykes, A Guidebook to Mechanism in Organic Chemistry
\end{wpdescription}

% Please see UserInterfaces.tex for now as an example

\begin{tasklist}
  % 3-5 tasks

  % The description of each task can be 5 to 15 lines depending on the
  % complexity and amount of details deemed necessary, and involve and
  % refer to 1-3 deliverables.

  \begin{task}[title=Reconcile the simulations]
  Extend and compare the simulations of \cite[Ch.~9]{VanMeurs15TH}
  and~\cite[Ch.~6]{Kooiman15TH}. On some cases, these two sets of results appear to
  contradict each other. Understand the origin of their differences, and reconcile the two
  methodologies.
  \end{task}

  \begin{task}[title=Upscaling the energy]
  Taking the lead from~\cite[Ch.~2]{Kooiman15TH} and~\cite{SandierSerfaty12TR}, develop
  mathematically rigorous theory for the upscaling of straight edge dislocations at finite
  temperature, focussing initially on the free energy.
  \end{task}
  
  \begin{task}[title=Upscaling the evolution]
  Develop mathematically rigorous theory for the evolution of straight edge dislocations.
  \end{task}
  
  \begin{task}[title=Forming processes]
  The forming of a metal
  \end{task}

\end{tasklist}

\eucommentary{ Deliverable numbers in order of delivery
  dates. Please use the numbering convention ``WP number''.``number of
  deliverable within that WP''.  For example, deliverable 4.2 would
  be the second deliverable from work package 4.
%
  Type:
  Use one of the following codes:
  R: Document, report (excluding the periodic and final reports)
  DEM: Demonstrator, pilot, prototype, plan designs
  DEC: Websites, patents filing, press \& media actions, videos, etc.
  OTHER: Software, technical diagram, etc.
  Dissemination level:
  Use one of the following codes:
  PU = Public, fully open, e.g. web
  CO = Confidential, restricted under conditions set out in Model Grant Agreement
  CI = Classified, information as referred to in Commission Decision 2001/844/EC.
  Delivery date
  Measured in months from the project start date (month 1)
}
\begin{wpdelivs}
  \begin{wpdeliv}[due=24,id=wp-kincon-1,dissem=PU,nature=R,lead=TUE]
    {One line deliverable description.}
  \end{wpdeliv}
  \begin{wpdeliv}[due=48,id=wp-kincon-2,dissem=PU,nature=DEM,lead=TUE]
    {One line deliverable description.}
  \end{wpdeliv}
\end{wpdelivs}
\end{workpackage}

%%% Local Variables:
%%% mode: latex
%%% TeX-master: "../proposal"
%%% End:
