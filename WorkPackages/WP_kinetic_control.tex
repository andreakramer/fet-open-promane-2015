\begin{workpackage}[id=WPkinetic,wphases=0-48,
short=Kin. control,
title=Kinetic control,
lead=TUE,
TUERM=24]

\begin{wpobjectives}
% Objectives of the work package; 5-10 lignes, typically itemized
The objectives of this work package are to:
\begin{compactitem}
\item Establish the extent of kinetic control in collective dislocation motion
\item Understand the origin of nonlinear flow laws in dislocation interaction
\end{compactitem}
\end{wpobjectives}

\begin{wpdescription}
  % Overall description; typically 10 lines to half a page
  % as appropriate, depending on the variety and number of tasks
Many non-equilibrium systems evolve under \emph{kinetic control:} their evolution at
laboratory time scales is determined as much by kinetic effects as by thermodynamic
forces. Examples are glasses at moderate temperatures, various chemical reactions
(e.g.~\cite{Sykes}) and dislocations in crystals. This is an intrinsically non-equilibrium
phenomenon, which interferes with the design and function of devices that interact with the
system, sometimes to the extent of completely preventing any normal function.
  
<<<<<<< HEAD
  In this work package we investigate this phenomenon on a simple, but paradigmatic example: the movement of dislocations in metal crystals. This system was the focus of a recent joint TUe Mechanical-Engineering-Mathematics research effort, focussing on the simplest nontrivial case, that of parallel edge dislocations, and leading to two theses~\cite{VanMeurs15TH,Kooiman15TH}. Among other results, these two theses clearly indicate the kinetic control of this system, through \emph{configurational temperature} as in glasses~\cite[Ch.~2--4]{Kooiman15TH}, \emph{emergent nonlinear flow rules}~\cite[Ch.~6]{Kooiman15TH}, and \emph{propagation failure}~\cite[Ch.~9]{VanMeurs15TH}.
  
  
    
  
    
  % Sykes, A Guidebook to Mechanism in Organic Chemistry
=======
In this work package we investigate this phenomenon on a simple, but paradigmatic example:
the movement of dislocations in metal crystals. This system was the focus of a recent joint
TUe Mechanical-Engineering-Mathematics research effort, focussing on the simplest nontrivial
case, that of parallel edge dislocations, and leading to two
theses~\cite{VanMeurs15TH,Kooiman15TH}. Among other results, these two theses clearly
indicate the kinetic control of this system, through \emph{configurational temperature} as
in glasses~\cite[Ch.~2--4]{Kooiman15TH}, \emph{emergent nonlinear flow
rules}~\cite[Ch.~6]{Kooiman15TH}, and \emph{propagation
failure}~\cite[Ch.~9]{VanMeurs15TH}.

% Sykes, A Guidebook to Mechanism in Organic Chemistry
>>>>>>> pdebuyl-lab/master
\end{wpdescription}

% Please see UserInterfaces.tex for now as an example

\begin{tasklist}
  % 3-5 tasks

  % The description of each task can be 5 to 15 lines depending on the
  % complexity and amount of details deemed necessary, and involve and
  % refer to 1-3 deliverables.

  \begin{task}[title=Reconcile the simulations]
  Extend and compare the simulations of \cite[Ch.~9]{VanMeurs15TH}
  and~\cite[Ch.~6]{Kooiman15TH}. On some cases, these two sets of results appear to
  contradict each other. Understand the origin of their differences, and reconcile the two
  methodologies.
  \end{task}

  \begin{task}[title=Upscaling the energy]
  Taking the lead from~\cite[Ch.~2]{Kooiman15TH} and~\cite{SandierSerfaty12TR}, develop
  mathematically rigorous theory for the upscaling of straight edge dislocations at finite
  temperature, focussing initially on the free energy.
  \end{task}
  
  \begin{task}[title=Upscaling the evolution]
  Develop mathematically rigorous theory for the evolution of straight edge dislocations.
  \end{task}
  
  \begin{task}[title=Response]
  Study response of the system under loading.
  \end{task}

\end{tasklist}

\eucommentary{ Deliverable numbers in order of delivery
  dates. Please use the numbering convention ``WP number''.``number of
  deliverable within that WP''.  For example, deliverable 4.2 would
  be the second deliverable from work package 4.
%
  Type:
  Use one of the following codes:
  R: Document, report (excluding the periodic and final reports)
  DEM: Demonstrator, pilot, prototype, plan designs
  DEC: Websites, patents filing, press \& media actions, videos, etc.
  OTHER: Software, technical diagram, etc.
  Dissemination level:
  Use one of the following codes:
  PU = Public, fully open, e.g. web
  CO = Confidential, restricted under conditions set out in Model Grant Agreement
  CI = Classified, information as referred to in Commission Decision 2001/844/EC.
  Delivery date
  Measured in months from the project start date (month 1)
}
\begin{wpdelivs}
  \begin{wpdeliv}[due=24,id=wp-kincon-1,dissem=PU,nature=R,lead=TUE]
    {One line deliverable description.}
  \end{wpdeliv}
  \begin{wpdeliv}[due=48,id=wp-kincon-2,dissem=PU,nature=DEM,lead=TUE]
    {One line deliverable description.}
  \end{wpdeliv}
\end{wpdelivs}
\end{workpackage}

%%% Local Variables:
%%% mode: latex
%%% TeX-master: "../proposal"
%%% End:
