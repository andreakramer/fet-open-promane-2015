\begin{workpackage}[id=WPcore,wphases=0-48,
  short=Gen. Theory, %XXX act. WP,% for Figure 5.
  title=General Theory, % XXX actual Work Package,
  lead=KUL,
  KULRM=36,UNIPDRM=6]

\newrefsection

\begin{mdframed}
\mobjectives
%
\begin{compactitem}
\item Develop the theoretical framework for the project.
\end{compactitem}

\mdescription
%
The Leuven node manages the network (see WP ``Management''), from the administrative point
of view, but also for the stimulation and inspiration of the project. The work at the \site{KUL}
can then be called the most theoretical, as major emphasis there will be on
providing frameworks and conceptual schemes.

Within this work package, we will work on complementing and extending the theory of
irreversible thermodynamics that has been developed in the 19th century, starting from
Onsager's work \cite{onsager1,onsager2}, also presented by \cite{degrootmazur} and \cite{kubo}.
%
That theory puts on a firm basis the nonequilibrium theory close-to-equilibrium, that is
when driving forces are small or when the system is started in local equilibrium.
%
Irreversible thermodynamics finds use in a myriad of technological applications involving
thermal and chemical transport, thermo-electric and thermo-magnetic phenomena, etc.
\TODO{brief list of applications}

We envision a radical step forward from this 50 year old basis. Time has come to incorporate
new ideas that have emerged over the last 20 years in the construction of a nonequilibrium
statistical mechanics, also away from equilibrium.
%
In general terms, these developments concern a fluctuation and response theory of systems
beyond the linear regime around reversibility. It reveals itself in constructive and
predictive methods for higher order response coefficients and for the stabilization and
control of effective dynamics in nonequilibrium environments.
%
Modern control technologies, biomedical and food processing techniques and new directions in
materials research will undoubtedly need to cross that boundary in the future. This
work package will provide the main resources on mathematical and theoretical levels to make
free way for the implementation of these techniques.

\printbibliography[heading=proposal-bib,env=proposal-env]

\end{mdframed}

\begin{tasklist}

\begin{task}[title=Give the theory of statistical forces outside equilibrium,id=core-t1,lead=KUL,partners={UNIPD},wphases={0-24!0.5,12-30}]
When probes get in contact with nonequilibrium media, their effective dynamics is governed
by forces that in general are no longer gradient (derivable from a potential), nor additive,
nor satisfying the action-reaction principle. Far from these being major setbacks, we
believe in turning these properties into new dynamical behavior showing unexpected phases of
matter.
\end{task}

\begin{task}[title=Formulate stability and control theory,id=core-t2,PM=12,lead=KUL,wphases=12-36!0.5]
The theory of dynamical systems is playing a great role in robotics and cybernetics, and is
used in many applications with feedback mechanisms. We put this theory on a higher extended
level, where the control also includes nonequilibrium reservoirs that can steer effective
interactions of subsystems.
%
It allows further and different stabilization mechanisms which are needed when dealing with
nonequilibrium or strongly transient media.
\end{task}

\end{tasklist}

\begin{wpdelivs}
  \begin{wpdeliv}[due=18,id=core-d1,dissem=PU,nature=DEM,lead=KUL,miles=framework]
      {Consistent nonequilibrium formulation for all model systems}
  \end{wpdeliv}
  \begin{wpdeliv}[due=36,id=core-d2,dissem=PU,nature=DEM,lead=KUL,miles=final]
      {Interpret the results from other work packages}
  \end{wpdeliv}
\end{wpdelivs}

\end{workpackage}
