\begin{workpackage}[id=WPcore,wphases=0-48,
  short=Gen. Theory, %XXX act. WP,% for Figure 5.
  title=General Theory, % XXX actual Work Package,
  lead=KUL,
  KULRM=36]

\begin{wpobjectives}

\end{wpobjectives}

\begin{wpdescription}

The Leuven node will have the role of coordinating and managing the network. That requires
not only some administrative tasks, and the (main-)organization of conferences, schooling or
workshops, but also the stimulation and inspiration of the work of the various other
nodes. The work in the Leuven and Prague nodes can then be called the most theoretical, as
major emphasis there will be on providing frameworks and conceptual schemes for
understanding and applying.

The core business of the Leuven and Prague nodes defining also this 6th workpackage in the
proposed network will be to provide a complement and extension of the irreversible
thermodynamics that has been developed in the previous century starting from Onsager's work
in the 1930's to the formulation of linear response theory 1960-70.  Major references from
around that time include the books by Mazur and de Groot (1964) and by Kubo (19??).
%
That theory put on a firm basis the nonequilibrium theory close-to-equilibrium, meaning in
the so called linear regime where driving forces are effectively small or where the initial
conditions are of local equilibrium type. Much technology has been based on the physics of
that irreversible thermodynamics, either for transport equations and the understanding of
linear response coefficients, or for a range of thermo-electric and thermo-magnetic
phenomena that have wide applications in industry.
%
We envision a radical step forward from this 50 year old basis. Time has come to incorporate
new ideas that have emerged over the last 20 years in the construction of a nonequilibrium
statistical mechanics, also away from equilibrium. In very general terms these developments
concern a fluctuation and response theory of systems beyond the linear regime around
reversibility. It reveals itself in constructive and predictive methods for higher order
response coefficients and for stabilization and possible control of effective dynamics in
nonequilibrium environments.
%
Modern control technologies, biomedical and food processing techniques and new directions in
materials research will undoubtly need to cross that boundary in the future. Workpackage 6
will provide the main resources on mathematical and theoretical levels to make free way for
the implementation of these techniques.

\end{wpdescription}

\begin{tasklist}

\begin{task}[title=Theory of statistical forces outside equilibrium,id=core-t1,PM=12,lead=KUL,wphases=0-24!0.5]
When probes get in contact with nonequilibrium media, their effective dynamics is governed
by forces that in general are no longer gradient (derivable from a potential), nor additive,
nor satisfying the action-reaction principle. Far from these being major setbacks, we
believe in turning these properties into new dynamical behavior showing unexpected phases of
matter.
\end{task}

\begin{task}[title=Stability and control theory,id=core-t2,PM=12,lead=KUL,wphases=12-36!0.5]
The theory of dynamical systems is playing a great role in robotics and cybernetics, and is
used in many applications with feedback mechanisms. We put this theory on a higher extended
level, where the control also includes nonequilibrium reservoirs that can steer effective
interactions of subsystems.
%
It allows further and different stabilization mechanisms which are needed when dealing with
nonequilibrium or strongly transient media.
\end{task}

\end{tasklist}

\begin{wpdelivs}
  \begin{wpdeliv}[due=24,id=core-d1,dissem=PU,nature=DEM,lead=KUL]
      {First deliverable, after 2 years}
  \end{wpdeliv}
  \begin{wpdeliv}[due=24,id=core-d2,dissem=PU,nature=DEM,lead=KUL]
      {Second deliverable, after 2 years}
\end{wpdeliv}
  \begin{wpdeliv}[due=48,id=core-d3,dissem=PU,nature=DEM,lead=KUL]
      {Third deliverable, after 4 years}
\end{wpdeliv}
\end{wpdelivs}

\end{workpackage}
