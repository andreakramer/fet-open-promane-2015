
\eucommentary{\begin{compactitem}
\item
Describe the consortium. How will it match the project's objectives?
How do the members complement one another (and cover the value chain,
where appropriate)? In what way does each of them contribute to the
project? How will they be able to work effectively together?
\item
If applicable, describe the industrial/commercial involvement in the
project to ensure exploitation of the results and explain why this is
consistent with and will help to achieve the specific measures which
are proposed for exploitation of the results of the project (see section 2.3).
\item
Other countries: If one or more of the participants requesting EU funding
is based in a country that is not automatically eligible for such funding
(entities from Member States of the EU, from Associated Countries and
from one of the countries in the exhaustive list included in General
Annex A of the work programme are automatically eligible for EU funding),
 explain why the participation of the entity in question is essential to carrying out the project
\end{compactitem}
}

The consortium consists in a diversity of researchers, from the Czech republic, Germany, the
Netherlands, Italy and Belgium, all much engaged in international collaborations and
organizations.
%
The ages range between 55 and 35 years for the principal investigators with a large backbone
of researchers and infrastructure, allowing easy intake of students and postdocs. An equal
opportunity policy is maintained, with special emphasis on including excellent women
researchers in the team. For example, recent collaborations of the coordinator included also
supervision of and joint work with Soghra Safaverdi (woman from Teheran), with Simi Thomas
(woman student from India) and with Urna Basu (woman postdoc from India).  Members of the
consortium know each other from sharing the same objectives and from complimentary
expertise.
%
Applied mathematics, mathematical physics, theoretical physics, soft condensed matter and
liquid matter labs join here in a global effort around nonequilibrium physics, with the
special aim of developing knowledge and tools for control and manipulation of nonequilibrium
systems. Some members already work together, and have established research connections,
e.g. Leuven-Prague, Stuttgart-Leipzig and Leuven-Padova.  Other collaborations are more
recent, such as Leuven-Stuttgart and Padova-Leipzig.  Still other visits and discussions
started since about one year, Leuven-Eindhoven en Leuven-Leipzig.
%
There will continue to be many exchanges of students and postdocs and mutual visits to bring
the results forward.  In particular, important exchanges between the theoretical institutes
and experimental labs are foreseen.  Many of these members have contacts with research
institutes that allow easy access to industrial and commercial involvement.  For example,
the coordinator has collaborations and supervises students at imec (Leuven), one of the main
players in technological innovation in Europe. From November 2015 we will enter in direct
contact with the managers at imec for discussing new avenues in quantum technology, with the
damping of decoherence through contacts with nonequilibria as one of the new (and not te be
disclosed) possibilities for a major breakthrough.

%%% Local Variables:
%%% mode: latex
%%% TeX-master: "proposal"
%%% End:

