\eucommentary{Milestones means control points in the project that help to chart progress. Milestones may
correspond to the completion of a key deliverable, allowing the next phase of the work to begin.
They may also be needed at intermediary points so that, if problems have arisen, corrective
measures can be taken. A milestone may be a critical decision point in the project where, for
example, the consortium must decide which of several technologies to adopt for further
development.}

\begin{milestones}
\milestone[id=start, month=1, verif={Public announcement of open positions.}]
%
{Starting up}
%
{Formal start of the project: Open the research positions, set up the website.}

\milestone[id=models,month=12, verif={}]
%
{Systems setup}
%
{Set up the experiments and the simulation software.}

\milestone[id=framework,month=24, verif={Availability of theoretical models. Publication of
  first technological report.}]
%
{Common theoretical framework}
%
{The physical properties of all model systems, theoretical and experimental, are formulated
  in a unified manner. The role of nonequilibrium in the properties of the systems is given
  a meaning.}

\milestone[id=data1,month=24, verif={Availability of experimental data.}]
%
{Experimental results}
%
{The first round of experiments provides validated data.}

\milestone[id=final,month=48, verif={Publication of final report and second
  technological report.}]%
{Final milestone}
%
{Join the theoretical model and experimental results in joint publications}

\end{milestones}

%%% Local Variables:
%%% mode: latex
%%% TeX-master: "proposal"
%%% End:

