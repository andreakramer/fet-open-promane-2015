\subsection{Project work plan}
\label{sec:wp}

\eucommentary{
Please provide the following:
\begin{compactitem}
\item brief presentation of the overall structure of the work plan;
\item timing of the different work packages and their components (Gantt chart or
  similar);
\item detailed work description, i.e.:
  \begin{compactitem}
  \item a description of each work package (table 3.1a);
  \item a list of work packages (table 3.1b);
  \item a list of major deliverables (table 3.1c);
  \end{compactitem}
\item graphical presentation of the components showing how they inter-relate (Pert
  chart or similar).
\end{compactitem}
%
Give full details. Base your account on the logical structure of the project and
the stages in which it is to be carried out. Include details of the resources to
be allocated to each work package. The number of work packages should be
proportionate to the scale and complexity of the project.\\
%
You should give enough detail in each work package to justify the proposed
resources to be allocated and also quantified information so that progress can
be monitored, including by the Commission.\\
%
You are advised to include a distinct work package on 'management' (see section
3.2) and to give due visibility in the work plan to 'dissemination and
exploitation' and 'communication activities', either with distinct tasks or
distinct work packages.\\
%
You will be required to include an updated (or confirmed) 'plan for the
dissemination and exploitation of results' in both the periodic and final
reports. (This does not apply to topics where a draft plan was not required.)
This should include a record of activities related to dissemination and
exploitation that have been undertaken and those still planned. A report of
completed and planned communication activities will also be required.\\
%
If your project is taking part in the Pilot on Open Research Data%
\footnote{%
  Certain actions under Horizon 2020 participate in the 'Pilot on Open Research
  Data in Horizon 2020'. All other actions can participate on a voluntary basis
  to this pilot. Further guidance is available in the H2020 Online Manual on the
  Participant Portal.},
you must include a 'data management plan' as a distinct deliverable within the
first 6 months of the project. A template for such a plan is given in the
guidelines on data management in the H2020 Online Manual. This deliverable will
evolve during the lifetime of the project in order to present the status of the
project's reflections on data management.\\
%
Definitions:
\begin{description}
\item[Work package] means a major sub-division of the proposed project.
\item[Deliverable] means a distinct output of the project, meaningful in terms of the project's overall
  objectives and constituted by a report, a document, a technical diagram, a software etc.
\item[Milestones] means control points in the project that help to chart progress. Milestones
    may correspond to the completion of a key deliverable, allowing the next phase of the
    work to begin. They may also be needed at intermediary points so that, if problems have
    arisen, corrective measures can be taken. A milestone may be a critical decision point in
    the project where, for example, the consortium must decide which of several technologies
    to adopt for further development.
\end{description}
%
Report on work progress is done primarily through the periodic and final reports. Deliverables
should complement these reports and should be kept to the minimum necessary.
}

\TheProject will begin by the setup of theoretical (\WPref{WPcompress},
\WPref{WPdissipation}) and experimental (\WPref{WPactive}, \WPref{WPbrown}) physical
models. At the same time, \WPref{WPcore} will lay the basis for the related theoretical
framework.
%
Then, building on the first round of results the work packages will feed on one another.

\subsection{Management and risk assessment}

\eucommentary{
  \begin{compactitem}
  \item Describe the organisational structure and the decision-making (including a list of
  milestones (table 3.2a)) .
\item Describe any critical risks, relating to project implementation, that the stated project's
  objectives may not be achieved. Detail any risk mitigation measures. Please provide a
  table with critical risks identified and mitigating actions (table 3.2b).
  \end{compactitem}
}

During the kickoff meeting, the participants will set the details of the action plan
described in the work packages and each participant is responsible for carrying out its
tasks.
%
We have designed our work packages to allow intermediate interesting results. For instance,
analytical computations in task~\taskref{WPdissipation}{diss-t3} can be replaced by
supporting numerical simulations.
%
Also, some experimental deliverables are of a very generic nature. The output of, e.g.,
\delivref{WPbrown}{brown-d1} is interesting {\em independently} of nonequilibrium
physics.

\subsection{Consortium as a whole}


\eucommentary{\begin{compactitem}
\item
Describe the consortium. How will it match the project's objectives?
How do the members complement one another (and cover the value chain,
where appropriate)? In what way does each of them contribute to the
project? How will they be able to work effectively together?
\item
If applicable, describe the industrial/commercial involvement in the
project to ensure exploitation of the results and explain why this is
consistent with and will help to achieve the specific measures which
are proposed for exploitation of the results of the project (see section 2.3).
\item
Other countries: If one or more of the participants requesting EU funding
is based in a country that is not automatically eligible for such funding
(entities from Member States of the EU, from Associated Countries and
from one of the countries in the exhaustive list included in General
Annex A of the work programme are automatically eligible for EU funding),
 explain why the participation of the entity in question is essential to carrying out the project
\end{compactitem}
}

%%% Local Variables:
%%% mode: latex
%%% TeX-master: "proposal"
%%% End:



\subsection{Resources to be committed}

\eucommentary{
  Please make sure the information in this section matches the costs as stated
  in the budget table in section 3 of the administrative proposal forms, and the
  number of person/months, shown in the detailed work package descriptions.\\
%
  \begin{compactitem}
  \item a table showing number of person/months required (table 3.4a)
  \item a table showing 'other direct costs' (table 3.4b) for participants where
    those costs exceed 15\% of the personnel costs (according to the budget
    table in section 3 of the administrative proposal forms)
  \end{compactitem}
}
