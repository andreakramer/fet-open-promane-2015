\providecommand{\classoptions}{keys}
\documentclass[noworkareas,deliverables,\classoptions]{euproposal}       % for writing
%\documentclass[submit,noworkareas,deliverables]{euproposal}        % for submission
%\documentclass[submit,public,noworkareas,deliverables]{euproposal} % for public version

\usepackage[utf8]{inputenc}

\usepackage{float}  % used to suppress floating of tables in Resources section.
\usetikzlibrary{calc,fit,positioning,shapes,arrows,snakes}

\addbibresource{bibliography.bib}
%%% institutions
\WAinstitution[id=KUL,
        countryshort=BE,
        acronym=KUL]
        {KU Leuven}

\WAinstitution[id=PAR2,
        countryshort=BE,
        acronym=Partner2]
        {Partner Two}

\WAinstitution[id=PAR3,
        countryshort=BE,
        acronym=Partner3]
        {Partner Three}

\WAperson[id=cmaes,
           personaltitle=Prof.,
           birthdate=1 Jan. 2000,
           academictitle=Professor,
           affiliation=KUL,
           department=Instituut voor Theoretische Fysica,
           privaddress=None of your business,
           privtel=that neither,
           email=christian.maes@fys.kuleuven.be,
           workaddress={Celestijnenlaan 200D, B-3001 Leuven, Belgium},
           worktel=+32 16 32 72 33,
           workfax=N/A
           ]
           {Christian Maes}

%%% Local Variables: 
%%% mode: latex
%%% TeX-master: "proposal"
%%% End: 
 % Some sections of the included files depend on this.
\usepackage{comments}
\usepackage{framed}

\defbibenvironment{proposal-env}
{\inparaenum}
{\endinparaenum}
{ \item}

\setdefaultleftmargin{1em}{0.5em}{0.5em}{0.5em}{0.5em}{0.5em}

\defbibheading{proposal-bib}[References]{\paragraph*{#1}}

\begin{document}

\begin{proposal}[
  % These PM numbers (person months) are for the coordinator to help planning
  % Participants should not change these, but add PM numbers in the CVS in
  % the site descriptions at CVs/*.tex
  site=KUL,
  site=TUE,
  site=ULEI,
  site=UNIPD,
  site=FZU,
  site=USTUTT,
  botupPM, % we want to work via bottom up PM distribution,
  coordinator=cmaes,
  coordinatorsite=KUL,
  acronym={PROMANE},
  acrolong={PROMANE},
  title=Probing Macroscopic Nonequilibria,
  callname=FET-OPEN Research and Innovation actions 2014-2015,
  callid=H2020-FETOPEN-2014-2015-RIA,
  keywords={Thermodynamics, Materials, Control},
  instrument=Horizon 2020 - Future and Emerging Technologies,
  months=48,
  compactht]
\newcommand{\TheProject}{\pn}% \pn is defined automatically

\begin{abstract}

\TheProject aims at changing the traditional thermodynamical approach for industrial
applications to include recent progress in nonequilibrium statistical thermodynamics in a
systematic manner.
%
Today interesting challenges and opportunities exist in exploring the new physics of
macroscopic nonequilibria. The latter refers to systems composed of many constituents which
are driven or active. The components could be self-propelling or could be subject to
rotational forces or just in contact with multiple reservoirs that impose conflicting
thermodynamic behavior. The main hypothesis of this project is that macroscopic
nonequilibria can be used as reservoirs or media for steering and controlling immersed
subsystems or probes. In that sense this project deals with an emerging control theory using
nonequilibria, and with the possibility of creating new macroscopic phases.
%
Among the expected deliverables are included enhanced stabilization of (for equilibrium)
metastable phases and the control of response parameters and transport coefficients. These
novel features are guided by new insights in the physics of nonequilibrium and will be
crucial in developing new materials or material properties in systematic ways. For
technological avenues our studies will reveal important insights and experimental and
computational methods for the charaterization of real materials; for example by providing
far-from-equilibrium relaxation times of viscoelastic media. Study of active nonequilibria
will serve as a common basis for enhanced targeted catalysis, for clustering properties or
for viscosity reduction in industrial processing. Other possible applications involve
noninvasive surgeries and drug delivery in medicine, or also the design of smart materials
with built-in actuation mechanisms.
%
While much emphasis before went to the destructive role of dissipation and its role in
erasing information and return to equilibrium, time has come to comprehend and exploit also
the constructive role and possibilities of nonequilibrium conditions. Complementary to
entropy production, changes in kinetics and dynamical activity make statistical forces
nongradient and nonadditive, allowing unseen and novel behavior.  Controlling it opens major
opportunities for technology.

\end{abstract}

\newpage

\ifsubmit\else\setcounter{tocdepth}{4}\fi
\tableofcontents


% ---------------------------------------------------------------------------
%  Section 1: Excellence
% ---------------------------------------------------------------------------

\section{S\&T Excellence}

Body of section ``Excellence''.

\subsection{Targeted breakthrough, Long term vision and Objectives}\label{sec:objectives}

\eucommentary{
  \begin{compactitem}
  \item Describe the targeted scientific breakthrough of the project.
  \item Describe how the targeted breakthrough of the project contributes to a
    long-term vision for new technologies.
  \item Describe the specific objectives for the project, which should be clear,
    measurable, realistic and achievable within the duration of the project.
  \end{compactitem}
}

The project is aimed at studying the statistical dynamics induced by contact with nonequilibrium media thus entering a major third stage of nonequilibrium research, earlier ones being the problem of approach to equilibrium and the dissipative transport of mass, energy and momentum between various equilibrium reservoirs.  The goal is to derive the precise connection between systematic force, friction and noise on probes and the fluctuation and response behavior of nonequilibrium baths.  The newly emerging physics for systems in active contact with nonequilibria is both important for theoretical advances in steady state thermodynamics, such as for discovering the nature and role of nonequilibrium free energies, for understanding the behaviour of particles in nonequilibrium environments such as in life and atmospheric processes, as possibly for interesting implementations in materials science and for obtaining stability of otherwise (under equilibrium conditions) unstable macroscopic conditions and phases. 



\subsection{Relation to the work programme}\label{sec:relation-wp}

\eucommentary{
  \begin{compactitem}
  \item Indicate the work programme topic to which your proposal relates, and
    explain how your proposal addresses the specific challenge and scope of that
    topic, as set out in the work programme.
  \end{compactitem}
}

Many aspects of technology, whether in production, optimization of automation, require understanding and controlling nonequilibrium processes.  That includes both the traditional engineering processes, but also and more than ever, food processing, biochemical and biomedical applications.  Scientific knowledge is the only and proven road to succes in these applications and ranks far aove, huight throug-put or trial and erro based methods.  Moreover, having today a choice of so many materials, so big data  and unseen challenges, tradition and experience alone do not suffice.  Solid science and high-level understanding are the keys to further important developments for sustaining or improving technological levels.  The present project emphasizes the nonequilibrium aspects, complementing and moving beyond the great succes of equilibrium statistical mechanics for obtaining and revolutionizing technology for the last 100 years.
The present proposal has as major challenge to control and steer devices or materials in contact with and through nonequilibirum environments.  That includes a great number of possible scenarios, including:
\begin{inparaenum}[(a)]
\item new materials and material properties
\item Biomedical processes and transport
\item Biomechanical structures and motility
\item Quantum stability and coherence control
\item Transient and fast processes
\item Economy and society
\item Climate science and weather prediction
\item Eco-science and sustainable energy.
\end{inparaenum}


\subsection{Novelty, level of ambition and foundational character}\label{sec:progress}

\eucommentary{
  \begin{compactitem}
  \item Describe the advance your proposal would provide beyond the
    state-of-the-art, and to what extent the proposed work is ambitious, novel
    and of a foundational nature. Your answer could refer to the ground-breaking
    nature of the objectives, concepts involved, issues and problems to be
    addressed, and approaches and methods to be used.
  \end{compactitem}
}

There are three major aspects of novelty and ambition.  Our work will be foundational and ground-breaking in 1) bringing forward the role of non-dissipative aspects in nonequilibrium physics, 2) its study of the dynamics of probes in nonequilibrium environments, 3) enabling systematic studies of certain technological questions and future activities:
ad 1) Now is the time to launch a decisive study on whether indeed the dynamical activity complements in essential ways entropic considerations in the construction of nonequilibrium statistical  mechanics. We propose therefore to devote a major and markedly innovative effort to unravel the conceptual, computational and operational meaning of the time-symmetric fluctuation sector, and how it enters observationally in the response of open systems driven away from equilibrium. At the same time, we expect such studies to unify various existing approaches and earlier suggestions such as in the escape rate formalism, or from the blowtorch theorem and in nonequilibrium reaction rate theory. The main objective then is to further develop fluctuation-response theory to systematize investigations on the operational-thermodynamic role and the statistical-mathematical nature of nonequilibrium kinetics for the elucidation of steady states beyond close-to-equilibrium regimes. The impact of a valid fluctuation-response theory away from equilibrium will be important in the more general context of complex systems and dissipative dynamics
A major part of theoretical and mathematical nonequilibrium studies of the last 150 years have been devoted to the relaxation to and the response in equilibrium. Where steady regimes further away from equilibrium are considered, the emphasis has remained on entropic considerations, either in the study of entropy production (as dissipated heat) or from the point of view of information-theoretic aspects.
Since some two decades nonequilibrium studies have revisited steady regimes further away from equilibrium, and a fluctuation-response theory is starting to emerge. Yet most results concern specific (often toy-)models, be it stochastic or highly chaotic. In general the broader theoretical ideas concentrate on entropy production as inherited from irreversible thermodynamics. Where time-symmetric quantities have entered, the understanding remains rather formal.
The usual formulation of statistical physics in its derivation of thermodynamics emphasizes the great disparity in phase space volumes corresponding to different macroscopic behavior. Away from thermodynamic equilibrium, phase space volume relations play a lesser role and moreover, as volumes get smaller, the surface area (in terms of exit and entrance rates) of phase space regions becomes more important.
Dynamical activity can in general be described as a measure of the system’s reactivity or of its escape rates, which can significantly change under driving conditions. Very recently we have learned that dynamical activity matters in nonequilibrium fluctuation-response theory, but no major observational consequences have been found yet.
Even though the notion of dynamical activity has been around in various guises in theoretical work in nonequilibrium physics, no operational or thermodynamic meaning was clearly formulated. To say it simply: we do not know how to measure it from some macroscopic quantity nor do we know how to represent it more statistical mechanically for a general system. No thermodynamic principles involve the dynamical activity.  The proposal chooses specific questions, some of which are mathematically very challenging, in fluctuation–response theory to open the meaning of dynamical activity, and to make it operational. The latter means to search for specific observational consequences and experimental control.
ad 2) We want to set up a fluctuation and response theory for nonequilibrium media from their influence on probe  dynamics, and we want to be able to control and monitor a system’s behavior from its contact with a nonequilibrium medium. The point is to explore the possible very new behaviour and properties of systems in contact with nonequilibrium media. A first question one can ask about a probe in active contact with a nonequilibrium sea is about the induced systematic statistical forces of the sea on the probe.  One starts from the more microscopic mechanical force and, assuming that the probe changes its state (e.g. position) on a much slower timescale than the medium, one integrates that mechanical force over the degrees of freedom of the stationary medium.  In equilibrium (that is, when the probe is in contact with an equilibrium reservoir) the resulting (statistical) forces are of gradient type.  The standard thermodynamic potentials are (indeed) the potentials from which the force can be derived as in Newtonian mechanics. The result there is that it allows us to draw free energy landscapes to understand the changes in the system, as if it concerned a conservative mechanical system for which we give the potential energy.  A special interesting example are the entropic forces, which work by the power of large numbers; then, the free energy has negligible energetic or enthalpic contributions.  These are known to be important in elasticity and polymer physics. Moreover, always for equilibrium, the statistical forces are additive for example in the sense that bulk contributions can easily be separated from boundary contributions.  The latter is crucial for thermodynamic behaviour.  For example, the very possibility of thermodynamic behaviour, the presence of an equation of state etc depend on that distinction.  All of that need not be true for probes (systems, walls, collective coordinates) in contact with (genuine) nonequilibrium media or reservoirs.  We have added the word “genuine” to indicate that we leave open the possibility that there are exceptional situations, for example in terms of specific models or special regimes where additive behaviour of the statistical forces would be restored.  We know that the close-to-equilibrium static large deviation functionals remain local and additive, so that it is fair to expect there the more usual thermodynamics compatible with e.g. 0th law behaviour or with the Clausius heat theorem that yields a calorimetric meaning to the fluctuation functionals close-to-equilibrium.   Yet, in general, and as generic as there are long range effects in nonequilibria we expect a breaking of additivity and of the gradient nature of statistical forces. Small perturbations in the sense of changes in boundary conditions or in changes of external potential can drastically change the stationary distribution also in the bulk of the system.  That has possibly interesting consequences in the manipulation of such forces, adding oscillatory components, going from attractive to repulsive and obtaining nonvanishing resulting forces that otherwise, in equilibrium, would be zero from mutually cancelling contributions. 
It also means that the stationary positions of the probe (quasi-static) would be at different locations compared with equilibrium, allowing possibly increased stability of phases that would otherwise (in equilibrium again) be unstable.  The paradigmatic example is here the Kapitza oscillator (pendulum) which remains stable upright when being shaken.  That example is however likely to be systematized also outside the theory of dynamical systems, reaching then in this project the stability of macroscopic behaviour of the probe (collective coordinate) in for equilibrium very unlikely phases and values for order parameters.  Needless to  add here that this may have dramatic consequences on the phase diagram for systems in contact with nonequilibrium media, not only breaking the Gibbs phase rule but also introducing new phases of matter.  That is certainly one of the most exciting possibilities of explorations in the present project. As another theme of stability we want to understand some mechanism of homeostasis. In our set-up we treat spatially extended systems with time-dependent boundary driving. We will see under what conditions the bulk of the system reaches a steady (time-independent) regime. A related consequence of contact with nonequilibrium reservoirs is the possibility of population inversion in the system, which will mean that the effective temperature of the probe could be much larger for certain purposes.  There will for sure not be the usual Einstein or second fluctuation-dissipation relation between noise and friction on the probe which also entails that the friction coefficient can show further nontrivial behaviour.  The phenomena of shear thinning and possibly shear thickening could very well be related to these.
ad 3) 










\subsection{Research methods}\label{sec:methods}
\eucommentary{
  \begin{compactitem}
  \item Describe the overall research approach, the methodology and explain its
    relevance to the objectives.
  \item Where relevant, describe how sex and/or gender analysis is taken into
    account in the project's content.
  \end{compactitem}
}

The overall research approach is that of modern physics, with its traditional ways of open exchange of ideas and results.  There are fundamentally two components, which are deeply connected: the theoretical and experimental side.
We certainly must work to keep up and even improve the exchanges between these efforts and to bring about and foster multiple exchanges of ideas and questions.  The way from theory to experiment is also the method to reach more applied science and eventually to influence industrial and economic research centers and technology.  We have been asked before to hep with industrial projects (e.g. water transfer in porous media, new challenges for construction physics, crack evolution and creep in glassy materials etc) but the present project will allow systematic developments in contacting and helping industrial players.
For the more daily research methods, the basic tools are stochastic processesm analysis and computer programming.  For the latter molecular dynamics simulations will paly a very big role, also because they reach more realistic scenario's that can be implemented and further verified in experimental work and tests.  The experimental side is much based on methods in fluid mechanics and optical control.

\begin{asparaenum}
\item Mathematical methods: Here we mostly follow stochastic calculus and the
  specialized probabilistic techniques of dealing with spatially extended
  stochastic dynamics of interacting particle systems. Limit theorems and large
  deviation theory are central tools.
\item Theoretical framework: Much emphasis remains on path-integral techniques
  in dynamical ensembles.  Such a Langrangian statistical mechanics takes up the
  entropy fluxes and the changes in DA to build the action functional for
  dynamical ensembles.  The nonequilibrium techniques will be essentially
  non-perturbative in the driving, and keeping a distance from the Keldysh
  formalisms.
\item Computational support: For various model systems that are not covered by
  theorems and analytic results, we use numerical methods and large scale
  simulation.
\item Experimental collaborations: Here we mostly seek collaborations with soft
  condensed matter labs.  Other questions relate more to biophysics and still
  others more to gas reactivity experiments.
\end{asparaenum}





\subsection{Interdisciplinary nature}\label{sec:interdisc}

\eucommentary{
  \begin{compactitem}
  \item Describe the research disciplines involved and the added value of the inter-disciplinarity.
  \end{compactitem}
}
Statistical and mathematical physics are by their nature rather inter-disciplinary.  Not only is their often a large intersection with mathematics, statistics and computer science, but also the subjects are often very diverse and reach in many various directions.  For the present project where we emphasize the nonequilibrium nature of phenomena and processes, there is the immediate contact with condensed matter physics , with fluid mechanics and with biological and medical/engineering sciences.  That is quite obvious as nonequilibria are ubiquitous.  Discovering common grounds of study and constructing for them a nonequilibrium statistical mechanics is exactly the heart of the present programme.
Research disciplines that are involved are very diverse. The most amazing examples of nonequilibrium are probably found in life processes, or perhaps in the origin of life itself.  There we see an open system full of transport processes, with little engines, pumps and cycles and the emergence of order on diverse spatio-temporal scales out of molecular complexity. The variety of phenomena where nonequilibrium considerations are essential is however much larger.  We find them at cosmological scales of the observable universe  and beyond the smallest sizes of nanotechnology.  Subatomic processes, the creation and annihilation of particles and fluctuations at the smallest dimensions produce sources of noise and matter, from which our world is finally made.  The second law of thermodynamics, the increase of entropy for closed systems, puts fundamental limitations on all models of the universe.  Various scenario's for the early universe  mimic transitions of metastable states to more   stable states, via
 nucleation processes as we see them also  in condensed matter.  On more earthly scales, evolution and the fight for low entropy create cycles of life and destruction that we recognize in all of nature.    Turbulent
  flow and nonlinear systems give other macroscopic realizations of nonequilibrium effects.   Climatology and ecology ask questions about the atmosphere and ocean dynamics and about food web chains as  nonequilibrium systems.
  In biology, molecular motors   give mesoscopic realizations of chemical engines and of transport on the molecular scales.   The cell itself, how it moves and how its membrane fluctuates on $\mu m$-scales, brings nonequilibrium statistical mechanics to life.   Chemical reactions are traditionally also sources of nonequilibrium changes.
 Further down, nanophysics studies processes of dissipation and transport at still smaller scales.




%%% Local Variables:
%%% mode: latex
%%% TeX-master: "proposal"
%%% End:


% ---------------------------------------------------------------------------
%  Section 2: Impact
% ---------------------------------------------------------------------------

\section{Impact}
\subsection{Expected impacts}

\eucommentary{
  Please be specific, and provide only information that applies to the proposal
  and its objectives. Wherever possible, use quantified indicators and targets.
%
  \begin{compactitem}
  \item Describe how your project will contribute to the expected impacts set out
    in the work programme under the relevant topic.
  \item Describe the importance of the technological outcome with regards to its
    transformational impact on science, technology and/or society.
  \item Describe the empowerment of new and high-potential actors towards future
    technological leadership.
  \end{compactitem}
}
Statistical mechanics has by itself had an enormous impact on technological and economic developments in the last century.  Its framework has not only inspired the logical framework of modeling and analysing microscopics to reach mesoscopic and macroscopic scales, but it has very specifically also enabled ab initio schemes and algorithms to reach and predict macroscopic behavior.  The foundations of solid state physics and quantum field theory that have reshaped the modern world and brought major changes in daily life invariably are consequences of the fluctuation theoretical ideas pioneered in statistical mechanics.  Yet most, if not all, of that is restricted to equilibrium statistical mechanics with its powerful and general Gibbs formalism.  While that formalism and the consequent algorithms and implementations remain a superb tool in technological research and innovation, the present project turns to nonequilibrium theory and phenomenology.  There we reach even much richer grounds, also connecting with biological systems and hence also medical applications.  The expected impact here will therefore add to the well-proven and established record of statistical mechanics in general, while opening new avenues in the following directions:\\

- new materials and material properties: we have in mind here extending the systematics and the solid theoretical base in the search of goog and useful biomaterials and meta-materials.  Transport theory beyond linear response and the emerging theory of statistical forces beyond equilibrium will allow systematic and algorithmic searches for new properties and materials.
- Biomedical processes and transport:  the cybernetics of active media and the steering of active particles in biological (living) environments is expected to revolutionize medical interventions and pharmacy.  Transport of drugs and signals in biomaterials is essential for new treatments.  That requires the understanding and modeling of interacting particle systems in nonequilibtium environments and the competence of realizing complex algorithms for transport.\\
- Biomechanical structures and motility: Similar to the previous point, we need to understand much better the structural and architectural aspects of living materials such as the cytoskeleton of a cell or in general the mechanical behavior of tissue.  That is essential also for understanding the motility of cells and for the control of transduction of signals to cell interior.  In our vision, basic nonequilibrium physics insights are crucial here, and complement the more biochemical and experimental research there.\\
- Quantum stability and coherence control: One of the major difficulties in upscaling the wonders of quantum mechanics is the loss of coherence when in contact with large thermal environments of when involving a great number of degrees of freedom.  A new idea is to bring the system in contact, not with an equilibrium theral environment, but with a driven nonequilibrium medium with well-defined and controlled properties that in fact would control the obtained coherence of immersed (quantum) devices.\\
- Transient and fast processes: A lot of industrial production relies on process that occur far from equilibrium.  Parameters like stresses and temperatures are changing very rapidly, and the products are subject to transient or very time-dependent environments. That can only be controlled more systematically from the point of view of nonequilibrium physics where response theory also for transient and nonequilibrium processes are in full development.\\
- Economy and society:  Econophysics, financial mathemtatics and socio-phyiscs all rely on models and insights of nonequilibrium physics. We expect that better insights in nonequilibrium processes can contribute to implement beneficial algorithms in economic or societal policy.  As one example, the idea of active and interacting agents that give rise to a collective macroscopic behavior, be it economic or political, is an interesting or even crucial addition to existing models if only for grasping nonequilibirum phenomena like clustering, flocking, heavy tails or non-additivity of forces.\\
- Climate science and weather prediction: It goes almost without saying that climate science and meteorolgy are fundamentally based on nonequilibrium physics as realized more specifically in fluid mechanics and atmosphere dynamics. Statistical modeling and understanding rare events are more and more central and most relevant issues in these. The behavior of probes in turbulent flow, the modeling of a global weather process and the setting up of reliable predictive capacity  are important research themes for some of today's world problems and nonequilibrium physics will obviously be a key-ingredient in any solution or advancement.\\ 
- Eco-science and sustainable energy: New ways of protecting the planet, keeping life interesting and have sustainable resources will rely on understanding non-dissipative aspects of physical processes.  Dissipation governs most of equilibrium physics, but beyond there are possibilities of self-organization and stabilization of structures and phases that are unstable or even non-existing in equilibrium. \\ 

In all of the above one recognizes the start of a major new ingredient in dealing with certain processes and technology.  The importance of the nonequilibrium paradigm and its transformational impact on science, technology and society are enormous and largely unexploited.  The present project wants to take that serious and start the far reaching way of implementing far from equilibrium physics. 

\subsection{Measures to maximise impact}

\paragraph{a) Dissemination and exploitation of results}

The major method of dissemination of our results is first the standard scientific practice of open access publishing, and participation in international conferences and discussion fora.  
The universities have important research and development centres where starting spin-offs find resources and support.
There will be important contacts with these centres. \\


Dissemination of the results is done by the free web archive, in publications in the standard specialized scientific journals (mostly non-commercial), by talks, by schooling and in contacts in conferences. As members of the Editorial Boards of JPhysA (IOP), of JMathPhys (APS), of the Journal of Statistical Physics (Springer), of Annales Henri Poincar\'e and of Fundamental Theories of Physics (Springer), we can organize special issues and create special volumes dedicated to aspects of the project. Indeed, as the project deals with some very novel aspect that has not appeared in standard or even not so standard treatments of irreversible thermodynamics, an important effort will be necessary and will be made to reach also less specialized researchers as well as scientists involved in possible applications.
Obviously an important cost then also goes to travelling and participation in conferences and international discussions. Again, since the project is not entirely mainstream and some concepts/questions are quite new, the project needs to invest in visibility and in numerous exchanges on international platforms. Standard recurring international conferences such as the Rutgers Statistical Mechanics Meeting, les Journ\'ees de Physique Statistique (Paris), MECO (Mid-European), the Statphys IUPAP meeting and the IAMathPhys-meeting will certainly be attended.  Key-world leading teams for the present project, certainly in Europe but also outside Europe including those in Japan and India will be frequented. We also hope to extend our network to groups in Beijing (e.g. Beijing computational center and the Biodynamic Optical Imaging Center of Peking University) and Singapore (IMS).  All team members will be expected to show great mobility.  Similar considerations apply for inviting scientists where costs include travelling, housing and subsistence. World-leading scientists will be invited to give intensive courses.  Those and other visiting experts will enable excellent collaborations.  We also foresee 20 seminars per year in Leuven directly related to the project, mostly from European scientists. They will of course be invited to stay a few days to make time for informal discussions. The project will also invest (still under other direct costs) in bringing together various points of view on the role of dynamical activity and the construction of nonequilibrium statistical mechanics. For that reason we foresee to organize two major conferences, one at around month 20, the other towards the end of the project.  In between, mini-workshops will be organized to discuss progress and challenges in smaller groups of experts: one at the beginning of the project, a second one when the project has gone through initial stages, and a final one towards the end to discuss its future. The costs should cover the travel of the expert participants and their presence at the conference site.


\eucommentary{
  \begin{compactitem}
  \item Provide a plan for disseminating and exploiting the project results. The
    plan, which should be proportionate to the scale of the project, should
    contain measures to be implemented both during and after the project.
  \item Explain how the proposed measures will help to achieve the expected
    impact of the project.
  \item Where relevant, include information on how the participants will manage
    the research data generated and/or collected during the project, in
    particular addressing the following issues\footnote{For further guidance on
      research data management, please refer to the H2020 Online Manual on the
      Participant Portal.}:
    \begin{compactitem}
    \item What types of data will the project generate/collect?  o What
      standards will be used?
    \item How will this data be exploited and/or shared/made accessible for
      verification and re-use?  If data cannot be made available, explain why.
    \item How will this data be curated and preserved?
    \end{compactitem}
 %      
    You will need an appropriate consortium agreement to manage (amongst other
    things) the ownership and access to key knowledge (IPR, data etc.). Where
    relevant, these will allow you, collectively and individually, to pursue
    market opportunities arising from the project's results.\\
%
    The appropriate structure of the consortium to support exploitation is
    addressed in section~3.3.
%
  \item Outline the strategy for knowledge management and protection. Include
    measures to provide open access (free on-line access, such as the 'green' or
    'gold' model) to peer- reviewed scientific publications which might result
    from the project.%
    \footnote{Open access must be granted to all scientific publications
      resulting from Horizon 2020 actions. Further guidance on open access is
      available in the H2020 Online Manual on the Participant Portal.}\\
%
    Open access publishing (also called 'gold' open access) means that an
    article is immediately provided in open access mode by the scientific
    publisher. The associated costs are usually shifted away from readers, and
    instead (for example) to the university or research institute to which the
    researcher is affiliated, or to the funding agency supporting the research.\\
%
    Self-archiving (also called 'green' open access) means that the published
    article or the final peer-reviewed manuscript is archived by the researcher
    - or a representative - in an online repository before, after or alongside
    its publication.  Access to this article is often - but not necessarily -
    delayed ('embargo period'), as some scientific publishers may wish to recoup
    their investment by selling subscriptions and charging pay-per-download/view
    fees during an exclusivity period.
%
  \end{compactitem}
}

\paragraph{b) Communication activities}

\eucommentary{
  \begin{compactitem}
  \item Describe the proposed communication measures for promoting the project
    and its findings during the period of the grant. Measures should be
    proportionate to the scale of the project, with clear objectives. They
    should be tailored to the needs of various audiences, including groups
    beyond the project's own community. Where relevant, include measures for
    public/societal engagement on issues related to the project.
  \end{compactitem}
}

Various members of the team have larger networks of communication both towards the general public (public outreach) as towards industrial players and groups of interest.  The first aim there is to transfer knowledge and tools, and to help in optimizing public interest. An important initiative will be the organization of ``physics meets industry days''
where each time during a week, specific problems of industry or economic activity will be presented.  They will be treated by students and experts towards helping to solve these problems, with direct feedback toward the industry of company.  Such initiatives exist already in some countries but will be started up in other European countries, and with an additional selection and expertise platform related to complex and nonequilibrium phenomena.



%%% Local Variables:
%%% mode: latex
%%% TeX-master: "proposal"
%%% End:


% ---------------------------------------------------------------------------
%  Section 3: Implementation
% ---------------------------------------------------------------------------

\section{Implementation}

\chapter{Implementation}\label{chap:implementation}

\section{Management Structure and Procedures}\label{chap:management}
\begin{todo}{from the proposal template}
  Describe the organizational structure and decision-making mechanisms
  of the project. Show how they are matched to the nature, complexity
  and scale of the project.  Maximum length of this section: five pages.
\end{todo}

The Project Management of {\pn} is based on its Consortium Agreement, which will be
signed before the Contract is signed by the Commission. The Consortium Agreement will
enter into force as from the date the contract with the European Commission is signed.
\subsection{Organizational structure}\label{sec:management-structure}
\subsection{Risk Assessment and Management}
\subsection{Information Flow and Outreach}\label{sec:spread-excellence}
\subsection{Quality Procedures}\label{sec:quality-management}
\subsection{Internal Evaluation Procedures}
\newpage
\section{Individual Participants}\label{sec:partners}
\begin{todo}{from the proposal template}
For each participant in the proposed project, provide a brief description of the legal entity, the main
tasks they have been attributed, and the previous experience relevant to those tasks. Provide also a
short profile of the individuals who will be undertaking the work.\\
Maximum length for Section 2.2: one page per participant. However, where two or more departments within
an organisation have quite distinct roles within the proposal, one page per department is acceptable.\\
The maximum length applying to a legal entity composed of several members, each of which is a separate
legal entity (for example an EEIG1), is one page per member, provided that the members have quite distinct
roles within the proposal.
\end{todo}
\newpage
\begin{sitedescription}{jacu}

\paragraph{Organization} Jacobs University Bremen is a private research university patterned
after the Anglo-Saxon university system.  The university opened in
2001 and has an international student body ($1,245$ students from 102
nations as of 2011, admitted in a highly selective process).

The KWARC (KnoWledge Adaptation and Reasoning for
Content\footnote{\url{http://kwarc.info}}) Group headed by
{\emph{Prof.\ Dr.\ Michael Kohlhase}} specializes in building
knowledge management systems for e-science applications, in particular
for the natural and mathematical sciences.  Formal logic, natural
language semantics, and semantic web technology provide the
foundations for the research of the group.
  
  Since doing research and developing systems is much more fun than writing proposals,
  they try go do that as efficiently as possible, hence this meta-proposal. 

\paragraph{Main tasks}

\begin{itemize}
\item creating {\LaTeX} class files
\end{itemize}

\paragraph{Relevant previous experience}

The KWARC group is the main center and lead implementor of the OMDoc
(Open Mathematical Document) format for representing mathematical
knowledge.  The group has developed added-value services powered by such semantically rich representations, different paths to obtaining them, as well as platforms that integrate both aspects.  Services include the adaptive context-sensitive presentation framework JOMDoc and the semantic search engine MathWebSearch.  For obtaining rich mathematical content, the group has been pursuing the two alternatives of assisting manual editing (with the sTeXIDE editing environment) and automatic annotation using natural language processing techniques.  The latter is work in progress but builds on the arXMLiv system, which is currently capable of converting 70\% out of the 600,000 scientific publications in the arXiv from {\LaTeX} to XHTML+MathML without errors.  Finally, the KWARC group has been developing the Planetary integrated environment.

\paragraph{Specific expertise}

\begin{itemize}
\item writing intelligent proposals
\end{itemize}

\paragraph{Staff members involved}

\textbf{Prof.\ Dr.\ Michael Kohlhase} is head of the KWARC research
group.  He is the head developer of the OMDoc mathematical markup
language.  He was a member of the Math Working Group at W3C, which finished its work with the publication of the MathML 3 recommendation.  He is president of the OpenMath society and trustee of the MKM
interest group.

\keypubs{KohDavGin:psewads11,Kohlhase:pdpl10,Kohlhase:omdoc1.2,CarlisleEd:MathML10,StaKoh:tlcspx10}
\end{sitedescription}

%%% Local Variables: 
%%% mode: LaTeX
%%% TeX-master: "propB"
%%% End: 

% LocalWords:  site-jacu.tex sitedescription emph textbf keypubs KohDavGin
% LocalWords:  psewads11 pdpl10 StaKoh tlcspx10
\newpage
\begin{sitedescription}{efo}
\paragraph{Organization}
 The EFO is the world leader in futurology, \ldots
\paragraph{Main tasks}
\paragraph{Relevant previous experience}
\paragraph{Specific expertise}
\paragraph{Staff members undertaking the work}
\keypubs{providemore}
\end{sitedescription}

%%% Local Variables: 
%%% mode: LaTeX
%%% TeX-master: "propB"
%%% End: 
\newpage
\begin{sitedescription}{bar}

\paragraph{Organization}
  Universit\'e de BAR specializes on drinking lots of red wine. It is a partner in the
  consortium, because it has a very nice chateau on the Cote d'Azure, where it can host
  gorgeous project meetings.

\paragraph{Main tasks}
\paragraph{Relevant previous experience}
\paragraph{Specific expertise}
\paragraph{Staff members undertaking the work}
\keypubs{providemore}

\end{sitedescription}

%%% Local Variables: 
%%% mode: LaTeX
%%% TeX-master: "propB"
%%% End: 
\newpage
\begin{sitedescription}{baz}
\paragraph{Organization}
\paragraph{Main tasks}
\paragraph{Relevant previous experience}
\paragraph{Specific expertise}
\paragraph{Staff members undertaking the work}
\keypubs{providemore}
\end{sitedescription}

%%% Local Variables: 
%%% mode: LaTeX
%%% TeX-master: "propB"
%%% End: 
\newpage

\section{The {\protect\pn} consortium as a whole}
\begin{todo}{from the proposal template}
  Describe how the participants collectively constitute a consortium capable of achieving
  the project objectives, and how they are suited and are committed to the tasks assigned
  to them. Show the complementarity between participants. Explain how the composition of
  the consortium is well-balanced in relation to the objectives of the project.  

  If appropriate describe the industrial/commercial involvement to ensure exploitation of
  the results. Show how the opportunity of involving SMEs has been addressed
\end{todo}

The project partners of the \pn project have a long history of successful collaboration;
Figure~\ref{tab:collaboration} gives an overview over joint projects (including proposals) and
joint publications (only international, peer reviewed ones).

\jointorga{jacu,efo,baz}
\jointpub{efo,baz,jacu}
\jointproj{efo,bar}
\coherencetable

\subsection{Subcontracting}\label{sec:subcontracting}
\begin{todo}{from the proposal template}
  If any part of the work is to be sub-contracted by the participant responsible for it,
  describe the work involved and explain why a sub-contract approach has been chosen for
  it.
\end{todo}
\subsection{Other Countries}\label{sec:other-countries}
\begin{todo}{from the proposal template}
  If a one or more of the participants requesting EU funding is based outside of the EU
  Member states, Associated countries and the list of International Cooperation Partner
  Countries\footnote{See CORDIS web-site, and annex 1 of the work programme.}, explain in
  terms of the project’s objectives why such funding would be essential.
\end{todo}

\subsection{Additional Partners}\label{sec:assoc-partner}
\begin{todo}{from the proposal template}
  If there are as-yet-unidentified participants in the project, the expected competences,
  the role of the potential participants and their integration into the running project
  should be described
\end{todo}
\section{Resources to be Committed}\label{sec:resources}
\begin{todo}{from the proposal template}
Maximum length: two pages

Describe how the totality of the necessary resources will be mobilized, including any resources that
will complement the EC contribution. Show how the resources will be integrated in a coherent way,
and show how the overall financial plan for the project is adequate.

In addition to the costs indicated on form A3 of the proposal, and the effort shown in Section 1.3
above, please identify any other major costs (e.g. equipment). Ensure that the figures stated in Part B
are consistent with these.
\end{todo}

\subsection{Travel Costs and Consumables}\label{sec:travel-costs}
\subsection{Subcontracting Costs}
\subsection{Other Costs}

%%% Local Variables: 
%%% mode: LaTeX
%%% TeX-master: "propB"
%%% End: 

% LocalWords:  pn newpage site-jacu site-efo site-baz jointpub efo baz
% LocalWords:  jointproj coherencetable assoc-partner


\gantttaskchart[draft,xscale=.33,yscale=.33,milestones]

\subsubsection{Deliverables}\label{sec:deliverables}
\inputdelivs{9.3cm}

\subsubsection{Milestones}\label{sec:milestones}
\eucommentary{Milestones means control points in the project that help to chart progress. Milestones may
correspond to the completion of a key deliverable, allowing the next phase of the work to begin.
They may also be needed at intermediary points so that, if problems have arisen, corrective
measures can be taken. A milestone may be a critical decision point in the project where, for
example, the consortium must decide which of several technologies to adopt for further
development.}

\begin{milestones}
\milestone[id=start, month=1, verif={Public announcement of open positions. Website online.}]
%
{Starting up}
%
{Formal start of the project: Open the research positions, set up the website.}

\milestone[id=models,month=12, verif={N/A}]
%
{Systems setup}
%
{Set up the model systems for both experiments and simulation software.}

\milestone[id=data1,month=20, verif={Availability of data.}]
%
{Data I}
%
{The first round of experiments and simulation campaigns provides data.}

\milestone[id=framework,month=28, verif={Availability of theoretical models. Publication of
  first technological report.}]
%
{Common theoretical framework}
%
{The physical properties of all model systems, theoretical and experimental, are formulated
  in a unified manner. The role of nonequilibrium in the properties of the systems is given
  a meaning.}

\milestone[id=data2,month=42, verif={Availability of data.}]
%
{Data II}
%
{The secound round of experiments and simulation campaigns provides data.}

\milestone[id=final,month=48, verif={Publication of final report and second technological
  report. Joint theoretical/experimental publication on the project.}]%
{Final milestone}
%
{Join the theoretical model and experimental results in joint publications}

\end{milestones}

%%% Local Variables:
%%% mode: latex
%%% TeX-master: "proposal"
%%% End:



% ---------------------------------------------------------------------------
% Include Work package descriptions
% ---------------------------------------------------------------------------

\newpage
\subsubsection{Work Package Descriptions}\label{sec:workpackages}
%% WP titles and order are defined in deliverables.tex
%%% work package style may be broken -- fix this!!

%% Local WP number counter - should possibly be global and hidden?
\begin{workplan}
\begin{workpackage}[id=management,type=MGT,wphases=0-48!.2,
  title=Project Management,short=Management,
  lead=KUL,KULRM=24]

\begin{wpobjectives}

Provide efficient communication and practical support for the project's events,
including IT support.
%
Serve as contact for the media.
%
Ensure timely and consistent reporting on the project advance and monitor the
progress of other work packages and deliverables.
%
Coordinate the technolog

\end{wpobjectives}

\begin{wpdescription}

The coordinator node, via the project manager, oversees the project from the
administrative point of view. Each role remains responsible for its internal
(accounting, hiring, etc) management while general project issues will be
handled at the KUL, including communication to and from the European Commission.

\end{wpdescription}

\begin{tasklist}
\begin{task}[title=Reporting,id=mgt-reporting,lead=KUL,PM=6,wphases=0-48!0.125]

Preparation of yearly reports, including the project's final report.

\end{task}

\begin{task}[title=Meetings,id=mgt-meetings,lead=KUL,PM=6,wphases=0-48!0.125]

Organization of the kickoff and subsequent meetings.

\end{task}

\begin{task}[title=IT,id=mgt-IT,lead=KUL,PM=6,wphases=0-48!0.125]

Setup and regular update of the project's website.
%
Preparation of data, when needed, for sharing via open data repositories.

\end{task}

\begin{task}[title=comm,id=mgt-comm,lead=KUL,PM=6,wphases=0-48!0.125]

Preparation of media content for the communication of the project: press
releases, videos, illustrations.

\end{task}
\end{tasklist}

\begin{wpdelivs}
  \begin{wpdeliv}[due=1,id=ca,dissem=CO,nature=R,lead=KUL]{Consortium Agreement}
  \end{wpdeliv}
  \begin{wpdeliv}[due=1,id=tickets,dissem=PU,nature=DEC,lead=KUL]{Web site}
  \end{wpdeliv}
  \begin{wpdeliv}[due=6,id=data,dissem=PU,nature=R,lead=KUL]{Hosting}
  \end{wpdeliv}
  \begin{wpdeliv}[due=24,id=comm,dissem=PU,nature=R,lead=KUL]{General public media}
  \end{wpdeliv}
  \begin{wpdeliv}[due=24,id=ca,dissem=CO,nature=R,lead=KUL]{Technological Report 1}
  \end{wpdeliv}
  \begin{wpdeliv}[due=48,id=finalreport,dissem=PU,nature=R,lead=KUL]{Final report of the project}
  \end{wpdeliv}
  \begin{wpdeliv}[due=48,id=ca,dissem=CO,nature=R,lead=KUL]{Technological Report 2}
  \end{wpdeliv}
\end{wpdelivs}

\end{workpackage}

%%% Local Variables: 
%%% mode: latex
%%% TeX-master: "../proposal"
%%% End: 

\begin{workpackage}[id=WP2ID,wphases=0-48,
  short=Nonequilibrium compressibility, %XXX act. WP,% for Figure 5.
  title=Nonequilibrium compressibility, % XXX actual Work Package,
  lead=Padova,
  PadovaRM=12,
  PAR2RM=6,
  PAR3RM=24]

\begin{wpobjectives}
  The objectives of this WP are:
  \begin{compactitem}
  \item Find the response to compression for bodies holding a heat flow
  \item Find if an averaged description (no microscopic details) may lead to effective fluctuation-response relations
  \item Understand if negative differential compressibilities can be realized
  \item Understand how active systems do respond to compression
  \end{compactitem}
\end{wpobjectives}

\begin{wpdescription}
Monitoring the volume of a system in equilibrium represents a standard procedure for assessing its state variables.
We know for example that for a gas in equilibrium there are state equations relating pressure and temperature to
the volume. Also the response to a compression is predictable by knowing the equilibrium correlations between the
compressing force and the volume, thanks to the fluctuation-dissipation theorem. The situation is less clear far
from equilibrium: how should we characterize the response to compression of a gas maintained between two temperatures?
Another example of unknown response would be that of ``active gases'', represented for example by bacteria within
a permeable box. In this case a compression would likely expel the water from the pores and expose our sensors
to a different interaction with the compressed bacteria. It is not trivial to predict by common sense the behavior 
of the system under these conditions. We would naively expect a reduction of the volume by increasing the external
compression, but there are already several examples of anomalous, or negative, response in nonequilibrium systems.
We have in mind the example of negative differential mobility, where a particle starts to drift slower by increasing
the external drive.

Of course another class of systems for which we need to understand the nonequilibrium response is that of solids.
There are important technological applications that involve keeping a metal in a temperature gradient, a condition
that generates heat flows and constant entropy production. To name a couple, strongly
heated micro-cantilevers and high-quality table-top oscillators of gravitational wave detectors are definitely
systems for which the study of nonequilibrium fluctuations and response is timely and relevant.
Cantilevers of the size up to $\approx 1 mm$ are used as devices to measure microscopic forces. To operate, a laser
may be pointed at their tip, so that a heat flux is installed from there to the base attached to a frame acting
as a heat sink. Recently it was found that vibrational modes of this system hold uneven amounts of energy.
Similar procedure and results, but on a larger scale, pertain the dynamics of table-top gravitational wave detectors.

All these examples stimulate the characterization of nonequilibrium systems with entropic flows, either macroscopically
 evident as in temperature gradients or microscopically ``hidden'' in the chemical reactions that drive bacteria. 
We focus on a particular feature, namely compressibility, of such nonequilibrium systems and we aim at understanding
how the response to compression takes place and, more importantly,
may be predicted by just knowing steady-state correlations
between observable and suitable descriptors of the system. The queen observable in this case would be the size of the
system. For this, the approach could be rephrased as a study of a nonequilibrium compressibility. The theoretical 
approaches from which to start, however, would allow a broader treatment and a focus on more general observables 
(e.g. internal energy, variances, flows of energy, etc.).

The theoretical tools we will use in the study of nonequilibrium compressibility are based on a recent line of research
that aims at studying nonequilibrium linear response for stochastic systems. From such approach there emerges an
important point: to understand the nonequilibrium response we need two families of system descriptors. In the first
kind we have entropy production, a familiar concept from the study of standard equilibrium systems. There is
 nevertheless the need to introduce a second class of indicators that measures the internal activity of the system.
During the last decade it became clear that such estimate of how the system is frenzy, or ``frenetic'', is necessary 
to understand nonequilibrium environments in general. In the context of response theory one thus needs to know
how a system, upon perturbation, becomes more or less active in its dynamics. This might involve needing
to measure delicate microscopic details, which could not be easily accessible experimentally. Hence, on the practical
side we have the important task to adapt the theoretical framework to the available experimental capabilities.





\end{wpdescription}

\begin{tasklist}

  \begin{task}[title=TASK1,id=task1,PM=15,lead=Padova,wphases=0-30!0.5]

    First task of the project: 
    for models of solids between two plates at different temperatures, 
    thus experiencing a heat flow, we plan to characterize the response to compression, mostly by applying 
    available formulations to the study of numerical simulations. The main aim is to make a setup for the
    other tasks.
    
  \end{task}

  \begin{task}[title=TASK2,id=task2,PM=15,lead=PAR2,wphases=12-42!0.5]

    Second task of the project: 
    we are interested in finding if microscopic details, available of course in simulations, 
    are needed or if a more experimental-like coarse grained description is sufficient or even convenient for
    predicting the response of the system's length to an increased compression. In practice, for example,
    by running a simulation of coupled oscillators we assume that only ``macroscopic'' quantities as the system length
    are measurable and from there we try to see if a coarse-grained fluctuation-response approach is able to
    describe the behavior of the system when a compression is added.
    
  \end{task}

  \begin{task}[title=TASK3,id=task3,PM=15,lead=Padova,wphases=0-30!0.5]

    Third task of the project: 
    we aim at finding peculiar inter-particle potentials that may lead to unusual
    compressibility in systems experiencing heat flows. The possibility to extract energy
    from an installed heat channel might in fact lead to the behavior of negative differential compressibility,
    i.e.~the system expands when compressed. 
    Though we plan to mainly operate at the conceptual level of toy
    models, a successful achievement on this direction might trigger technological applications with immense
    relevance. Metamaterials are indeed nowadays much studied because, for example, micro-actuators would be
    more easily realized if materials with a negative compressibility were introduced. We will investigate
    if steady nonequilibrium conditions could produce this effect, as opposed to the recently considered 
    mechanism of rearrangements from metastable states. It is not easy to predict the success of this line,
    which is thus a high-risk high reward task of this WP.
    
  \end{task}

  \begin{task}[title=TASK4,id=task4,PM=15,lead=Padova,wphases=0-30!0.5]

    {\bf [THIS MIGHT BE SHIFTED TO OTHER COLLABORATIONS]}
    Fourth task of the project: 
    understanding how active systems do respond to compression is pretty much
    synonymous of assessing whether state equations relating pressure to temperature, volume, and internal
    interaction parameters, are available for active matter. We aim at studying the compressibility of
    models of bacterial colonies, both from the theoretical side and by analyzing simulations of active matter.
    
  \end{task}


\end{tasklist}

\begin{wpdelivs}
  \begin{wpdeliv}[due=12,id=mydeliv1,dissem=PU,nature=DEM,lead=Padova]
      {First deliverable, after 1 year.}
  \end{wpdeliv}
  \begin{wpdeliv}[due=24,id=mydeliv2,dissem=PU,nature=DEM,lead=PAR2]
      {Second deliverable, after 2 years.}
\end{wpdeliv}
\end{wpdelivs}

\end{workpackage}

\begin{workpackage}[id=WPcontrol,wphases=0-48,
  short=Noneq. control, %XXX act. WP,% for Figure 5.
  title=Nonequilibrium control, % XXX actual Work Package,
  lead=KUL,
  KULRM=12]

\begin{wpobjectives}
  The objectives of this WP are:
  \begin{compactitem}
  \item Find 
  \item Find 
  \end{compactitem}
\end{wpobjectives}

\begin{wpdescription}

Write description here.

\end{wpdescription}

\begin{tasklist}

  \begin{task}[title=TASK1,id=task1,PM=15,lead=KUL,wphases=0-12!0.5]

    First task of the project: 

  \end{task}

  \begin{task}[title=TASK2,id=task2,PM=15,lead=KUL,wphases=9-24!0.5]

    Second task of the project: 

  \end{task}

  \begin{task}[title=TASK3,id=task3,PM=15,lead=KUL,wphases=12-45!1.0]

    Third task of the project: 

    
  \end{task}


\end{tasklist}

\begin{wpdelivs}
  \begin{wpdeliv}[due=12,id=mydeliv1,dissem=PU,nature=DEM,lead=Padova]
      {First deliverable, after 1 year.}
  \end{wpdeliv}
  \begin{wpdeliv}[due=24,id=mydeliv2,dissem=PU,nature=DEM,lead=PAR2]
      {Second deliverable, after 2 years.}
\end{wpdeliv}
\end{wpdelivs}

\end{workpackage}

\begin{workpackage}[id=WPkinetic,wphases=0-48,
short=Kin. control,
title=Kinetic control,
lead=TUE,
TUERM=24]

\begin{wpobjectives}
% Objectives of the work package; 5-10 lignes, typically itemized
The objectives of this work package are to:
\begin{compactitem}
\item Establish the extent of kinetic control in collective dislocation motion
\item Understand the origin of nonlinear flow laws in dislocation interaction
\end{compactitem}
\end{wpobjectives}

\begin{wpdescription}
  % Overall description; typically 10 lines to half a page
  % as appropriate, depending on the variety and number of tasks
Many non-equilibrium systems evolve under \emph{kinetic control:} their evolution at
laboratory time scales is determined as much by kinetic effects as by thermodynamic
forces. Examples are glasses at moderate temperatures, various chemical reactions
(e.g.~\cite{Sykes}) and dislocations in crystals. This is an intrinsically non-equilibrium
phenomenon, which interferes with the design and function of devices that interact with the
system, sometimes to the extent of completely preventing any normal function.
  
<<<<<<< HEAD
  In this work package we investigate this phenomenon on a simple, but paradigmatic example: the movement of dislocations in metal crystals. This system was the focus of a recent joint TUe Mechanical-Engineering-Mathematics research effort, focussing on the simplest nontrivial case, that of parallel edge dislocations, and leading to two theses~\cite{VanMeurs15TH,Kooiman15TH}. Among other results, these two theses clearly indicate the kinetic control of this system, through \emph{configurational temperature} as in glasses~\cite[Ch.~2--4]{Kooiman15TH}, \emph{emergent nonlinear flow rules}~\cite[Ch.~6]{Kooiman15TH}, and \emph{propagation failure}~\cite[Ch.~9]{VanMeurs15TH}.
  
  
    
  
    
  % Sykes, A Guidebook to Mechanism in Organic Chemistry
=======
In this work package we investigate this phenomenon on a simple, but paradigmatic example:
the movement of dislocations in metal crystals. This system was the focus of a recent joint
TUe Mechanical-Engineering-Mathematics research effort, focussing on the simplest nontrivial
case, that of parallel edge dislocations, and leading to two
theses~\cite{VanMeurs15TH,Kooiman15TH}. Among other results, these two theses clearly
indicate the kinetic control of this system, through \emph{configurational temperature} as
in glasses~\cite[Ch.~2--4]{Kooiman15TH}, \emph{emergent nonlinear flow
rules}~\cite[Ch.~6]{Kooiman15TH}, and \emph{propagation
failure}~\cite[Ch.~9]{VanMeurs15TH}.

% Sykes, A Guidebook to Mechanism in Organic Chemistry
>>>>>>> pdebuyl-lab/master
\end{wpdescription}

% Please see UserInterfaces.tex for now as an example

\begin{tasklist}
  % 3-5 tasks

  % The description of each task can be 5 to 15 lines depending on the
  % complexity and amount of details deemed necessary, and involve and
  % refer to 1-3 deliverables.

  \begin{task}[title=Reconcile the simulations]
  Extend and compare the simulations of \cite[Ch.~9]{VanMeurs15TH}
  and~\cite[Ch.~6]{Kooiman15TH}. On some cases, these two sets of results appear to
  contradict each other. Understand the origin of their differences, and reconcile the two
  methodologies.
  \end{task}

  \begin{task}[title=Upscaling the energy]
  Taking the lead from~\cite[Ch.~2]{Kooiman15TH} and~\cite{SandierSerfaty12TR}, develop
  mathematically rigorous theory for the upscaling of straight edge dislocations at finite
  temperature, focussing initially on the free energy.
  \end{task}
  
  \begin{task}[title=Upscaling the evolution]
  Develop mathematically rigorous theory for the evolution of straight edge dislocations.
  \end{task}
  
  \begin{task}[title=Response]
  Study response of the system under loading.
  \end{task}

\end{tasklist}

\eucommentary{ Deliverable numbers in order of delivery
  dates. Please use the numbering convention ``WP number''.``number of
  deliverable within that WP''.  For example, deliverable 4.2 would
  be the second deliverable from work package 4.
%
  Type:
  Use one of the following codes:
  R: Document, report (excluding the periodic and final reports)
  DEM: Demonstrator, pilot, prototype, plan designs
  DEC: Websites, patents filing, press \& media actions, videos, etc.
  OTHER: Software, technical diagram, etc.
  Dissemination level:
  Use one of the following codes:
  PU = Public, fully open, e.g. web
  CO = Confidential, restricted under conditions set out in Model Grant Agreement
  CI = Classified, information as referred to in Commission Decision 2001/844/EC.
  Delivery date
  Measured in months from the project start date (month 1)
}
\begin{wpdelivs}
  \begin{wpdeliv}[due=24,id=wp-kincon-1,dissem=PU,nature=R,lead=TUE]
    {One line deliverable description.}
  \end{wpdeliv}
  \begin{wpdeliv}[due=48,id=wp-kincon-2,dissem=PU,nature=DEM,lead=TUE]
    {One line deliverable description.}
  \end{wpdeliv}
\end{wpdelivs}
\end{workpackage}

%%% Local Variables:
%%% mode: latex
%%% TeX-master: "../proposal"
%%% End:

\begin{workpackage}[id=WPcore,wphases=0-48,
  short=Gen. Theory, %XXX act. WP,% for Figure 5.
  title=General Theory, % XXX actual Work Package,
  lead=KUL,
  KULRM=36,UNIPDRM=6]

\newrefsection

\begin{wpobjectives}
\begin{compactitem}
\item Develop the theoretical framework for the project.
\end{compactitem}
\end{wpobjectives}

\begin{wpdescription}

The Leuven node manages the network (see WP ``Management''), from the administrative point
of view, but also for the stimulation and inspiration of the project. The work in the Leuven
and Prague nodes can then be called the most theoretical, as major emphasis there will be on
providing frameworks and conceptual schemes.

Within this work package, we will work on complementing and extending the theory of
irreversible thermodynamics that has been developed in the 19th century, starting from
Onsager's work \cite{onsager1,onsager2}, also presented by \cite{degrootmazur} and \cite{kubo}.
%
That theory puts on a firm basis the nonequilibrium theory close-to-equilibrium, that is
when driving forces are small or when the system is started in local equilibrium.
%
Irreversible thermodynamics finds use in a myriad of techonological applications involving
thermal and chemical transport, thermo-electric and thermo-magnetic phenomena, etc.
\TODO{brief list of applications}

We envision a radical step forward from this 50 year old basis. Time has come to incorporate
new ideas that have emerged over the last 20 years in the construction of a nonequilibrium
statistical mechanics, also away from equilibrium.
%
In general terms, these developments concern a fluctuation and response theory of systems
beyond the linear regime around reversibility. It reveals itself in constructive and
predictive methods for higher order response coefficients and for the stabilization and
control of effective dynamics in nonequilibrium environments.
%
Modern control technologies, biomedical and food processing techniques and new directions in
materials research will undoubtly need to cross that boundary in the future. This
workpackage will provide the main resources on mathematical and theoretical levels to make
free way for the implementation of these techniques.

\printbibliography[heading=proposal-bib,env=proposal-env]

\end{wpdescription}

\begin{tasklist}

\begin{task}[title=Theory of statistical forces outside equilibrium,id=core-t1,lead=KUL,partners={UNIPD},wphases={0-24!0.5,12-30}]
When probes get in contact with nonequilibrium media, their effective dynamics is governed
by forces that in general are no longer gradient (derivable from a potential), nor additive,
nor satisfying the action-reaction principle. Far from these being major setbacks, we
believe in turning these properties into new dynamical behavior showing unexpected phases of
matter.
\end{task}

\begin{task}[title=Stability and control theory,id=core-t2,PM=12,lead=KUL,wphases=12-36!0.5]
The theory of dynamical systems is playing a great role in robotics and cybernetics, and is
used in many applications with feedback mechanisms. We put this theory on a higher extended
level, where the control also includes nonequilibrium reservoirs that can steer effective
interactions of subsystems.
%
It allows further and different stabilization mechanisms which are needed when dealing with
nonequilibrium or strongly transient media.
\end{task}

\end{tasklist}

\begin{wpdelivs}
  \begin{wpdeliv}[due=18,id=core-d1,dissem=PU,nature=DEM,lead=KUL,miles=framework]
      {Consistent nonequilibrium formulation for all model systems}
  \end{wpdeliv}
  \begin{wpdeliv}[due=36,id=core-d2,dissem=PU,nature=DEM,lead=KUL,miles=final]
      {Interpret the results from other work packages}
  \end{wpdeliv}
\end{wpdelivs}

\end{workpackage}

\end{workplan}

%%% Local Variables:
%%% mode: latex
%%% TeX-master: "../proposal"
%%% End:


\newpage
\subsection{Management Structure and Procedures}
\label{sect:mgt}




%%% Local Variables:
%%% mode: latex
%%% TeX-master: "proposal"
%%% End:


\draftpage
\subsection{Consortium as a Whole}

\eucommentary{\begin{compactitem}
\item
Describe the consortium. How will it match the project's objectives?
How do the members complement one another (and cover the value chain,
where appropriate)? In what way does each of them contribute to the
project? How will they be able to work effectively together?
\item
If applicable, describe the industrial/commercial involvement in the
project to ensure exploitation of the results and explain why this is
consistent with and will help to achieve the specific measures which
are proposed for exploitation of the results of the project (see section 2.3).
\item
Other countries: If one or more of the participants requesting EU funding
is based in a country that is not automatically eligible for such funding
(entities from Member States of the EU, from Associated Countries and
from one of the countries in the exhaustive list included in General
Annex A of the work programme are automatically eligible for EU funding),
 explain why the participation of the entity in question is essential to carrying out the project
\end{compactitem}
}

%%% Local Variables:
%%% mode: latex
%%% TeX-master: "proposal"
%%% End:


\draftpage

\subsection{Resources to be Committed}
\eucommentary{Please provide the following:
\begin{compactitem}
\item
a table showing number of person/months required (table 3.4a)
\item
a table showing 'other direct costs' (table 3.4b) for participants where
those costs exceed 15\% of the personnel costs (according to the budget
table in section 3 of the administrative proposal forms)
\end{compactitem}}

\subsubsection{Management Level Description of Resources and Budget}
\label{sect:budget-details}

\paragraph{Staff efforts}

\eucommentary{Please indicate the number of person/months over the whole
duration of the planned work, for each work package, for each participant.
Identify the work-package leader for each WP by showing the relevant
person-month figure in bold.}

\wpfig[label=fig:staffeffort,caption=Summary of Staff Efforts]

\subsubsection{Resource summaries for consortium member sites}
\label{resources.summary}

%%%%%%%%%%%%%%%%%%%%%%%%%%%%%%%%%%%%%%%%%%%%%%%%%%%%%%%%%%%%%%%%
%
% Guidelines for completion of partner specific resource summary:
%
%
% Please explain how many person months for each person are
% requested. Say who is the local lead. Say anything that helps to
% understand why people are recruited as you plan, in particular if
% this deviates from having one research for 48 months.  We can also
% use this bit of the proposal (and the table, see below) to address
% any other unusual arrangements.
%
%
% The table should contain all non-staff costs (the EU requests that
% this table must be present if the non-staff costs exceed
% 15% of the total cost, but it is good practice and will show
% openness and transparency that we show the data for all partners).
%
% Link back from the table to the work packages and tasks for which
% the expenses are required. Add information that makes it easier to
% understand why the expenses are justified.
%
%     To refer to a task in a work package, use "\taskref{WP-ID}{TASK-ID}" where
%     WP-ID is the ID of the work package:
%        WP#: WP-ID - full title
%        ----------------------
%        WP1: 'management' - Management
%        WP2: 'community' - Community Building and Engagement
%        WP3: 'component-architecture' - Component Architecture
%        WP4: 'UI' - User interfaces
%        WP5: 'hpc' - High Performance Computing
%        WP6: 'dksbases' - Data/Knowledge/Software-Bases
%        WP7: 'social-aspects' - Social Aspects
%        WP8: 'dissem' - Dissemination
%
%
%     and "TASK-ID" is the ID of the task. You can set this using
%
%       \begin{task}[id=TASK-ID,title=Math Search Engine,lead=JU,PM=10,lead=JU]
%
%     To refer to deliverables, use "\delivref{WP-ID}{DELIV-ID}" where DELIV-ID is
%     the ID of the deliverable that can be set like this:
%
%       \begin{wpdeliv}[due=36,id=DELIV-ID,dissem=PU,nature=DEM]
%           {Exploratory support for semantic-aware interactive widgets providing views on objects
%           represented and or in databases}
%       \end{wpdeliv}
%
%
% The table is pre-populated with entries most sites are likely
% to need. If a line does not apply to you, just delete it. If you need
% an extra line, then add it. Use common sense: the number of rows should not
% be very big, but at the same time it is useful to give some breakdown/explanation
% of costs.
%
%
% Eventually, try to create you entry similar in style to the others.
% (The Southampton entry is fully populated, so use this as guidance
% if in doubt.)
%
%
%%%%%%%%%%%%%%%%%%%%%%%%%%%%%%%%%%%%%%%%%%%%%%%%%%%%%%%%%%%%%%%%

%%%%%%%%%%%%%%%%%%%%%%%%%%%%%%%%%%%%%%%%%%%%%%%%%%%%%%%%%%%%%%%%%%%%%%%%%%%%%%
\paragraph{Resources PAR1}

PAR1 will consist of PAR1P1 and PAR1P2.

\paragraph{Resources PAR2}

PAR2 will consist of PAR2P1.

\paragraph{Resources PAR3}

PAR1 will consist of PAR3P1 and PAR3P2.

%%% Local Variables:
%%% mode: latex
%%% TeX-master: "proposal"
%%% End:


% ---------------------------------------------------------------------------
%  Section 4: Members of the Consortium
% ---------------------------------------------------------------------------

\newpage

\eucommentary{This section is not covered by the page limit.\\
The information provided here will be used to judge the operational capacity.}

\section{Members of the Consortium}

\subsection{Participants}

\eucommentary{Please provide, for each participant, the following (if available):\\
\begin{compactitem}
\item
a description of the legal entity and its main tasks,
with an explanation of how its profile matches the tasks in the proposal;
\item
a curriculum vitae or description of the profile of the persons,
including their gender, who will be primarily responsible for carrying
out the proposed research and/or innovation activities;
%
this includes a description of the profile of the to-be-recruited personnel
\item
a list of up to 5 relevant publications, and/or products, services
(including widely-used datasets or software), or other achievements
relevant to the call content;
\item
a list of up to 5 relevant previous projects or activities, connected
to the subject of this proposal;
\item
a description of any significant infrastructure and/or any major items
of technical equipment, relevant to the proposed work;
\item
any other supporting documents specified in the work programme for this call.
\end{compactitem}}

\begin{sitedescription}{KUL} \label{desc:KUL}

KU Leuven is a major and leading university and research center in Belgium and Europe. The Department of Phyiscs and Astronomy plays a crucial role in the Faculty of Science.

\subsubsection*{Curriculum vitae of the investigators}

\begin{participant}[type=PI,PM=12,gender=female,salary=5500]{Person One}

  Person One is an experienced professor and will lead the project.

\end{participant}

%%% Local Variables:
%%% mode: latex
%%% TeX-master: "../proposal"
%%% End:

\begin{participant}[type=R,PM=48,gender=male,salary=5500]{Person Two}

  Person Two is an experienced researcher and will participate full time to the
  project.

\end{participant}

%%% Local Variables:
%%% mode: latex
%%% TeX-master: "../proposal"
%%% End:


\begin{participant}[type=res,PM=48,salary=5500]{NN}
\end{participant}
\begin{participant}[type=res,PM=36,salary=5500]{NN}

We need researchers. Two, for instance.

\end{participant}

\begin{participant}[type=res,PM=24,salary=3932]{NN}
  We will hire an experienced part time project manager to help with
  the overall management during the whole duration of \TheProject.
\end{participant}

\subsubsection*{Publications, achievements}

\begin{compactenum}
\item Leadership.
\item Coauthoring.
\end{compactenum}


\subsubsection*{Previous projects or activities}

\begin{compactenum}
\item Hosting.
\item Co-organising.
\end{compactenum}

\subsubsection*{Significant infrastructure}

We have building, at PAR1.

\end{sitedescription}



\begin{draft}
\vspace{1cm}\TOWRITE{PAR1P1}{Complete check list below -- delete completed items if you wish}

\begin{verbatim}
- [ ] checked that sum of person months put into finance request is
  the same as sum of person months associated with the Work Packages
  (in proposal.tex, as defined as part of the \begin{workpackage}"
  command.
  
- [ ] completed site specific resource summary in resources.tex,
  including table of non-staff costs.

\end{verbatim}
\end{draft}

%%% Local Variables: 
%%% mode: latex
%%% TeX-master: "../proposal"
%%% End: 

\begin{sitedescription}{TUE}

TUE is an internationally leading research university specialised in engineering science and technology. TUE is a constant presence in the top of various rankings, all acknowledging in particular the research carried out in collaboration with the industry, and currently manages more than 150 EU-funded projects. TUE offers 50 different educational programs to about 9000 students of various levels, and employs more than 3000 academic and administrative personnel.

The TUE research team in this proposal is part of both the Department of Mathematics and Computer Science and the Institute for Complex Molecular Systems. 
The Department Mathematics and Computer Science (MCS) unites all activities on campus in the classical scientific disciplines of mathematics and computer science. Both research and education have recently been evaluated as excellent by international committees. 40\% of the employees at MCS are of non-Dutch origin, thus creating a truly multi-cultural environment. 

The Institute for Complex Molecular Systems (ICMS) is a hotbed for interdisciplinary interaction in research and education across the university. Its mission is to become the leading international multidisciplinary institute for research and education in the area of the engineering of complex molecular systems.

The joint affiliation with both the Department of Mathematics and Computer Sciences and the ICMS provides the TUE research team with both excellent mathematical infrastructure and broad experimental embedding.


\subsubsection*{Curriculum vitae}

% Curriculum of the personnel at this institution. This includes
% to-be-hired people for which there is a tentative candidate.

\begin{participant}[type=leadPI,PM=24,gender=male,salary=5500]{Mark Peletier}


Mark Peletier is full professor of mathematics at the Department of Mathematics and Computer Science, head of the Center for Analysis, Scientific Computing, and Applications, and founding member of the Institute for Complex Molecular Systems at TUE. 

His research interest lies in the application of mathematical methods to the understanding of the physical world. He has contributed to the characterization of patterns in extended systems, the study of instabilities in mechanical structures, the relation between large deviations and variational evolution, and the connection between dissipation and functionality. 

He has published more than 80 papers in peer-reviewed journals, has supervised 9 PhD students, is editor of three journals, and is a regular member of committees of the Dutch funding body NWO. He is a recipient of two of the high-profile Personal Innovation grants (VIDI and VICI) of NWO. In 2005, Peletier was invited to join the prestigious Young Academy of the Royal Dutch Academy of Sciences. 

\end{participant}

%%% Local Variables:
%%% mode: latex
%%% TeX-master: "../proposal"
%%% End:


\begin{participant}[type=R,PM=12,gender=male,salary=5500]{Adrian Muntean}


Adrian Muntean is lecturer at the Center for Analysis, Scientific Computing, and Applications and the Institute for Complex Molecular Systems at TUE. His research interests lie in 
the theoretical understanding of basic interactions between large scale transport and local interactions, such as phase transitions, chemical reactions, physical/social collisions, through separated length scales ranging from discrete and stochastic systems to multiscale continua. 

With cum laude PhD student Joep Evers (2015) he has contributed to the mathematical  understanding of models for crowd evolution, including the essential interaction with hard and soft boundaries. Other objects of study are defects in non-homogeneous materials and ``swimming'' of active particles in compressible flows.

He has published more than 60 papers in peer-reviewed journals, edited two books and five special issues, supervised 10 master students and five PhD students, and organized more than 20 conferences and workshops. 

\end{participant}

%%% Local Variables:
%%% mode: latex
%%% TeX-master: "../proposal"
%%% End:

% For other to-be-hired person, please include here something like:
% \begin{participant}[type=res,PM=3,salary=5900]{NN}
%  <a _short_ description of the qualifications of whom you want to hire>
% \end{participant}

\begin{participant}[type=res,PM=48]
A PhD student will be hired for the duration of the project that matches the duration of a
PhD thesis in the Netherlands. 
%
The opening will be international and open to competent candidates from any
origin.
\end{participant}


\subsubsection*{Publications, products, achievements}

\begin{compactenum}
\item {In a series of papers, starting with \emph{Adams et al., Communications in Mathematical Physics, 307:791 (2011)}, Peletier and co-authors identified and explored the deep relations between gradient flows on one hand and large deviations of stochastic processes on the other. These relations create new understanding of the mathematical structures describing physical phenomena at the mesoscale, and point the way towards understanding the strongly nonequilbrium systems of this proposal.}
\item {M. A. Peletier, G. Savar\'e, and M. Veneroni, From diffusion to reaction via $\Gamma$-convergence, SIAM Journal on Mathematical Analysis, 42(4), pp. 1805--1825, 2010.
This paper has sparked renewed interest in the analysis community in the old problem of a stochastic particle escaping from a potential well, by introducing a new method to combine reactive and diffusive effects. It was selected as a SIGEST paper in SIAM Review 54(2) in 2012.}
\item {Peletier is a co-founder and board member of the ICMS, and both Peletier and Muntean are prominent members of this interdisciplinary institute.}
\end{compactenum}

\subsubsection*{Previous projects or activities}

\begin{compactenum}
\item {Peletier is and has been PI on many national and international projects, including an EU Seventh-framework ITN project `Fronts and Interfaces in Science and Technology'.}
\item {Peletier and Muntean have together and separately organized over 50 meetings, workshops, and conferences.}
\end{compactenum}

\subsubsection*{Significant infrastructure}

{Both CASA and the ICMS have significant in-house computing resources (several computing clusters) as well as access to national computing facilities at SARA (Amsterdam).}
\end{sitedescription}

%%% Local Variables:
%%% mode: latex
%%% TeX-master: "../proposal"
%%% End:

\begin{sitedescription}{ULEI} \label{desc:ULEI}

 Universität Leipzig was founded in 1409 and is the second oldest university in Germany where teaching has continued 
 without interruption. 
  Today it offers a wide spectrum of academic disciplines at 14 faculties with more than 150 institutes. 
  for close to 30 000 students taught by approximately 400 Professors, 
  resulting in more than 570 advanced degrees (PhD, etc.) annually.
  It is a member of the German U15, a strategic alliance of 15 major German research universities. 
  In the recently finished 7. Research Framework Programme of the EU the university participated 
  in 75 projects with an overall EC-contribution of some 25 million Euro.

  The node participants are located at the Institute for Experimental Physics I and at the Institute for Theoretical Physics, 
  of the Faculty of Physics and Earth Sciences, which also comprises institutes for meteorology, geology, geophysics 
  and geography. 
  It counts among the leading faculties in terms of research output and external funding, within the university, 
  and hosts several ERC grantees. 
  It is the first faculty of the university that recently got its research activities evaluated by an external board of
  international experts. 
  The physics institutes make major contributions to several collaborative research centers (SFBs) funded by the 
  German Science Foundation (DFG) and to the interdisciplinary Graduate School ``Leipzig School of Natural Sciences -- 
  Building with Molecules and Nano-objects'' (www.buildmona.de), founded by a grant from the German Excellence Initiative,
  which has so far enrolled close to 200 PhD candidates. 
  They maintain collaborations and joint grants with a large number of independent international 
  and local research institutes for fundamental and applied science and industrial partners. 

\subsubsection*{Curriculum vitae of the investigators}

\begin{participant}[type=R,PM=12,gender=male,salary=5500]{Klaus Kroy}

Professor at the Institute of Theoretical Physics, Universität Leipzig.

Klaus Kroy is a theoretical physicist and an expert in the field of
soft mesoscopics (non-equilibrium dynamics of colloids and polymers; active particles; cytoskeleton and tissue mechanics; single-molecule force spectroscopy; aeolian sand transport and structure formation)

He has published about 60 articles in peer-reviewed journals (also in
Nature Physics, Nature Communications, PNAS, PRL) 

He has in the past supervised 3 postdocs, 5 PhD students, and 19
master students 

He is a Member of the German Physical Society, of the International Max--Planck Research Group
Mathematics in the Sciences (Leipzig), and he recently received grants from
the German Excellence Initiative (Graduate School ``BuildMoNa''),
the DFG-Forschergruppe FOR 877, the German
priority programm SPP1726 (DFG), the German Israel
Foundation, the ESF, and a DFG-individual-grant.


\end{participant}

\begin{participant}[type=PI,PM=12,gender=male,salary=5500]{Frank Cichos}

Professor at the Institute of Experimental Physics I, Universität Leipzig. \url{http://www.uni-leipzig.de/~mona}

Frank Cichos is an experimental physicist and an expert optical microscopy and optical single molecule detection
(photothermal single molecule detection; active particles; single molecule trapping; single molecule dynamics in soft matter).
He has in the past published about 76 articles in peer-reviewed journals (also in Nano Letters, ACS nano and PRL) 
and supervised 2 postdocs, 15 PhD students, and about 20 master students.
He is a Member of the German Physical Society and the American Physical Society. He recently received grants from the German
Excellence Initiative (Graduate School ``BuildMoNa''), the DFG-Forschergruppe FOR 877, the German priority program SPP1726 
(DFG), the DFG Sonderforschungsbereich TRR102 and a joint  DFG-ANR-individual-grant. He is the co-speaker of the 
DFG Sonderforschungsbereich TRR102 and has been the speaker of the DFG-Forschergruppe FOR 877.

\end{participant}


\begin{participant}[type=res,PM=48,salary=5500]{NN}
A postdoc will be hired to work on the project. We aim to hire someone who has a strong
background in theoretical statistical mechanics and a diverse research experience. The person will keep contact to the other project nodes and the experimental project partners.
\end{participant}

\begin{participant}[type=res,PM=36,salary=5500]{NN}
A postdoc researcher is required to carry out the experiments on hot Brownian motion at nanosecond timescales. The postdoc should have a solid background in optical tweezer experiments and fast positional detection. He/She shall work closely with the theoretical project partners and keep close contact to the other nodes.
\end{participant}

\subsubsection*{Publications, achievements}

\begin{compactenum}
\item K. Kroy and F. Cichos have published a series of papers on Hot Brownian Motion, including joint articles such as 
Physical Review Letters {\bf 105} 090604 (2010), which has been cited 81 times according to Google Scholar, 
highlighting the interest in fundamental non-equilibrium fluctuation dissipation theorems and their experimental verification.

\item Further pertinent publications by the PIs include Nano Lett. {\bf 15} 5499 (2015), 
Biochimica et Biophysica Acta -- Molecular Cell Research {\bf 1853} 3025 (2015),
Nature Communications {\bf 4} 1780 (2013) and {\bf 5} 4463 (2014),
Nature Nanotech. {\bf 9} 415 (2014), Physical Review Letters {\bf 113} 238302 (2014), 
ACS Nano {\bf 6} 2714 (2012) and {\bf 8} 6542 (2014), EPL {\bf 96} 60009 (2011).

%\item M. Gralka, K. Kroy, Inelastic mechanics: A unifying principle in biomechanics. 
% Biochimica et Biophysica Acta (BBA)--Molecular Cell Research 2015

%\item S. Schöbl, S. Sturm, W. Janke, K. Kroy, Persistence-Length Renormalization of Polymers in a Crowded Environment of Hard
% Disks. Physical Review Letters {\bf 113} 238302 (2014).

%\item J. T. Bullerjahn, S. Sturm, K. Kroy, Theory of rapid force spectroscopy. Nature Communications {\bf 5} 4463 (2014).

%\item O. Otto, S. Sturm, N. Laohakunakorn, U. F. Keyser, K. Kroy, Rapid internal contraction boosts DNA friction.
% Nature Communications {\bf 4} 1780 (2013).

%\item D. Chakraborty, M. V. Gnann, D. Rings, J. Glaser, F. Otto, F. Cichos, K. Kroy, 
% Generalised Einstein relation for hot Brownian motion. EPL (Europhysics Letters) {\bf 96} 60009 (2011).

%\item D. Rings, R. Schachoff, M. Selmke, F. Cichos, K. Kroy. 
%Hot Brownian Motion.  Physical Review Letters {\bf 105},  090604 (2010)

%\item M. Braun, A. Bregulla, K. Günther, N. Mertig, F. Cichos. Single Molecules Trapped by Dynamic Inhomogeneous 
%Temperature Fields Nano Lett. 15 5499 (2015).

%\item A. Bregulla, H. Yang, F. Cichos Stochastic Localization of Micro-Swimmers by Photon Nudging ACS Nano 8 6542 (2014).

%\item M. Selmke, M. Braun, F. Cichos Photothermal Single Particle Microscopy: Detection of a Nanolens ACS Nano 6 2714 (2012).

\end{compactenum}

\subsubsection*{Previous projects or activities}

\begin{compactenum}
%
\item Joint organization of the Diffusion Fundamentals Conference V (2013). 
Joint Initiation of a workshop series "Hot Nanostructures"
(in 2011) continued every two years. It is focussed on non-equilibrium physics highlighting theory and experiments 
involving large temperature gradients and will return to Leipzig in 2017. 

\item Project lead (speaker) of the research unit 877 ``From Local Constraints to Macroscopic Transport'' of the German 
Science Foundation (DFG).

\item Partners in international collaborative projects (German Excellence Initiative, Agence National de la Recherche - 
German Science Foundation, DFG, German Israel Foundation, ESF)
\end{compactenum}

\subsubsection*{Significant infrastructure}
The Institute of Theoretical Physics hosts its own water-cooled computer cluster, 
involving a subcluster of GPU servers reserved for the PI's massively parallel NEMD simulations, 
which provides sufficient resources for the project tasks.
\end{sitedescription}

\begin{draft}
\vspace{1cm}\TOWRITE{PAR1P1}{Complete check list below -- delete completed items if you wish}

\begin{verbatim}
- [ ] checked that sum of person months put into finance request is
  the same as sum of person months associated with the Work Packages
  (in proposal.tex, as defined as part of the \begin{workpackage}"
  command.
  
- [ ] completed site specific resource summary in resources.tex,
  including table of non-staff costs.

\end{verbatim}
\end{draft}

%%% Local Variables: 
%%% mode: latex
%%% TeX-master: "../proposal"
%%% End: 

\begin{sitedescription}{Padova} \label{desc:Padova}

KU Leuven boasts a rich tradition of education and research that dates back six
centuries. The university's basic research orientation has always been and will
remain fundamental research. At the same time, the university remains vigilantly
open to contemporary cultural, economic and industrial realities, as well as to
the community's needs and expectations. From a basis of social responsibility
and scientific expertise, KU Leuven provides high-quality, comprehensive health
care, including specialised tertiary care, in its University Hospitals. In doing
so it strives toward optimum accessibility and respect for all patients.

KU Leuven is currently by far the largest university in Belgium in terms of
research funding and expenditure (EUR 426.5 million in 2014), and is a charter
member of LERU. KU Leuven conducts fundamental and applied research in all
academic disciplines with a clear international orientation.  Leuven
participates in over 540 highly competitive European research projects (FP7,
2007-2013), ranking sixth in the league of HES institutions participating in
FP7. In Horizon 2020, KU Leuven currently has been approved 79 projects.

KU Leuven takes up the 9th place of European institutions hosting ERC grants (as
first legal signatories of the grant agreement). To date, the
\href{http://www.kuleuven.be/english/research/EU/p/erc}{78 ERC Grantees}
(including affiliates with VIB and IMEC) in our midst confirm that KU Leuven is
a breeding ground (51 Starting Grants) and attractive destination for the
world's best researchers. The success in the FP7 and Horizon 2020 Marie
Sklodowska Curie Actions is a manifestation of the three pillars of KU Leuven:
research, education and service to society. In our
\href{http://www.kuleuven.be/english/research/EU/p/horizon2020/es/msca}{170
Actions}, of which 76 Initial/European Training Networks, hundreds of young
researchers have been trained through research and have acquired the necessary
skills to transfer their knowledge into the world outside academia.


KU Leuven employs 8,671 researchers on its academic staff (2014). To strengthen
international collaboration, KU Leuven has its own international research
fellowship programme and supports international scholars in international
funding applications. KU Leuven Research \& Development (LRD) is the technology
transfer office (TTO) of the KU Leuven. Since 1972 a multidisciplinary team of
experts guides researchers in their interaction with industry and society, and
the valorisation of their research results (101 spin offs, \dots).

\subsubsection*{Curriculum vitae of the investigators}

\begin{participant}[type=PI,PM=12,gender=male,salary=5500]{Christian Maes}

Full professor at the KUL and director of the Institute for Theoretical Physics.
%
Christian Maes is a leading scientist in the field of statistical mechanics, regularly
invited as a keynote to scientific events.
%
He has published 150 articles in peer-reviewed journals,
is currently an associate editor or member of the editorial board of 4 international journals,
has supervised 14 PhD theses (2 more ongoing) and 11 postdoctoral researchers,
is expert and reviewer for many scientific instituttions and
is a member of the evalution commission for ERC Starting Grants in Mathematics since 2014.

\end{participant}

%%% Local Variables:
%%% mode: latex
%%% TeX-master: "../proposal"
%%% End:

\begin{participant}[type=R,PM=48,gender=male,salary=5500]{Person Two}

  Person Two is an experienced researcher and will participate full time to the
  project.

\end{participant}

%%% Local Variables:
%%% mode: latex
%%% TeX-master: "../proposal"
%%% End:


\begin{participant}[type=res,PM=48,salary=5500]{NN}
\end{participant}
\begin{participant}[type=res,PM=36,salary=5500]{NN}

We need researchers. Two, for instance.

\end{participant}

\begin{participant}[type=res,PM=24,salary=3932]{NN}
  We will hire an experienced part time project manager to help with
  the overall management during the whole duration of \TheProject.
\end{participant}

\subsubsection*{Publications, achievements}

\begin{compactenum}
\item Leadership.
\item Coauthoring.
\end{compactenum}


\subsubsection*{Previous projects or activities}

\begin{compactenum}
\item Hosting.
\item Co-organising.
\end{compactenum}

\subsubsection*{Significant infrastructure}

We have building, at PAR1.

\end{sitedescription}



\begin{draft}
\vspace{1cm}\TOWRITE{PAR1P1}{Complete check list below -- delete completed items if you wish}

\begin{verbatim}
- [ ] checked that sum of person months put into finance request is
  the same as sum of person months associated with the Work Packages
  (in proposal.tex, as defined as part of the \begin{workpackage}"
  command.
  
- [ ] completed site specific resource summary in resources.tex,
  including table of non-staff costs.

\end{verbatim}
\end{draft}

%%% Local Variables: 
%%% mode: latex
%%% TeX-master: "../proposal"
%%% End: 

\begin{sitedescription}{FZU}

Fyzikální Ústav AV ČR, v. v. i. (FZU; in English: Institute of Physics of the Czech Academy
of Sciences) is a public research institute, oriented on the fundamental and applied
research in physics.

\subsubsection*{Curriculum vitae}

\begin{participant}[type=leadPI,PM=24,gender=male,salary=5500]{Karel Netočný}

Scientist in the division of Condensed Matter Physics (2).

\end{participant}

%%% Local Variables:
%%% mode: latex
%%% TeX-master: "../proposal"
%%% End:


\subsubsection*{Publications, products, achievements}

\begin{compactenum}
\item \TOWRITE{XXX}{...}
\end{compactenum}

\subsubsection*{Previous projects or activities}

\begin{compactenum}
\item \TOWRITE{XXX}{...}
\end{compactenum}

\subsubsection*{Significant infrastructure}

\TOWRITE{XXX}{...}
\end{sitedescription}

%%% Local Variables:
%%% mode: latex
%%% TeX-master: "../proposal"
%%% End:

\begin{sitedescription}{USTUTT} \label{desc:USTUTT}

{\bf Universität Stuttgart -- A Research University of International Standing:}\\
The Universität Stuttgart lies right in the centre of the largest high-tech region of Europe. We are surrounded by a number of renowned research facilities and have such global players as Daimler or IBM as our neighbours. We were founded in 1829 and over the years this technical institution has developed to the research intensive university that it is today. Our main emphasis is on engineering and the natural sciences.
%
Indicators of our excellent status are the two projects that were successful in the recent {\it Excellence Initiative} sponsored by both the Federal and the State governments. One project is the Cluster of Excellence {\it Simulation Technology} and the other, the Graduate School {\it Advanced Manufacturing Engineering}.

{\bf Experience with EU research funding:}\\
The University of Stuttgart has extensive experience with the various funding programs of the European Commission and has been the {\it leading German university in FP6}, both in number of projects (184) and in terms of funding (54 Mio. \euro). In FP7, it was yet again {\it among the most successful German universities} with 246 projects funded and a total budget of 94 Mio. \euro. The University has consistently been involved in Marie-Curie-projects in previous framework programs. In FP7, it participated in 13 Marie Curie actions.

\subsubsection*{Curriculum vitae of the investigators}

\begin{participant}[type=leadPI,PM=12,gender=male,salary=5500]{Matthias Krüger}

Matthias Krüger joined the University of Stuttgart in October 2012 and is leading the
independent research group {\it Non-equilibrium Systems} located at the Max Planck Institute
for Intelligent Systems, Stuttgart. Before coming to Stuttgart, he was a postdoc at the
Massachusetts Institute of Technology in Cambridge, USA.  In 2009, he gained his Doctoral
degree as a theoretical physicist at the University of Konstanz, Germany.

He theoretically studies different aspects of nonequilibrium statistical physics, including
fluids far from equilibrium, fluctuation (Casimir) forces under nonequilibrium conditions,
as well as radiative energy transport on the nano-scale.
%
He recently received an Emmy Noether Grant from the German Research Foundation (DFG), a
prestigious program that allows young researchers early independence.  Before, he was
supported through other programs of DFG and Fulbright and is a regular committee member of
the German National Academic Foundation.

He published around 30 articles in peer-reviewed journals, and, including ongoing projects, supervised 2 postdocs, 1 Phd student and 3 undergraduate students.

\end{participant}
\begin{participant}[type=PI,PM=12,gender=male,salary=5500]{Clemens Bechinger}

Full professor and head of the 2nd Experimental Institute at the University
of Stuttgart. Fellow of the Max Planck Institute of Intelligent Systems.

%

Clemens Bechinger is an expert in the field of experimental soft matter
systems and regularly invited as keynote and plenary speaker in
international scientific meetings.

%

He has published about 125 articles in peer-reviewed journals (also in
Nature, Science, PNAS, PRL) and is member of the liquid matter board of the
EPS and the Panel ``Statistical Physics, Soft Matter, Biophysics, Nonlinear
Dynamics'' of the German Research Society. Since his arrival in Stuttgart in
2003 he has supervised 10 Postdocs, 18 Phd Students and 22 Master Students.

\end{participant}

\begin{participant}[type=res,PM=48,salary=5500]{NN}
Postdoctoral Researcher.
\end{participant}
\begin{participant}[type=res,PM=36,salary=5500]{NN}
PhD Student.
\end{participant}

\subsubsection*{Publications, achievements}

\begin{compactenum}
\item Leadership.
\item Coauthoring.
\end{compactenum}

\subsubsection*{Previous projects or activities}

\begin{compactenum}
\item Organization.
\item Partner.
\end{compactenum}

\subsubsection*{Significant infrastructure}

\end{sitedescription}


\subsection{Third Parties Involved in the Project (including use of third party resources)}
\label{section:ThirdParties}

\paragraph{Third Party 1}\ 

\eucommentary{Please complete, for each participant, the table
(see page 27 of "VRETemplate.PDF"),
or simply state "No third parties involved", if applicable.}

Third Party 1 (hereafter TP1) will work on the project.

\paragraph{Other participants}\ 

For other participants, the only subcontracting costs will be for audit.

\bgroup
\def\arraystretch{1.5}  % 1 is the default
\noindent \begin{tabular}{|p{0.6\textwidth}|c|}
\hline
Does the participant plan to subcontract certain
tasks & Yes \\
\hline
\multicolumn{2}{|l|}{Audit} \\
\hline
Does the participant envisage that part of its work
is performed by linked third parties & No \\
\hline
\multicolumn{2}{|l|}{} \\
\hline
Does the participant envisage the use of
contributions in kind provided by
third parties & No \\
\hline
\multicolumn{2}{|l|}{} \\
\hline
\end{tabular}
\egroup

%No third parties involved.

% ---------------------------------------------------------------------------
%  Section 5: Ethics and Security
% ---------------------------------------------------------------------------

\newpage

\section{Ethics and Security}

\eucommentary{This section is not covered by the page limit.}

\subsection{Ethics}

\eucommentary{
If you have entered any ethics issues in the ethical issue table in the administrative proposal forms, you must:\\
$\bullet$ submit an ethics self-assessment, which: \\
-- describes how the proposal meets the national legal and ethical requirements of the
country or countries where the tasks raising ethical issues are to be carried out;\\
-- explains in detail how you intend to address the issues in the ethical issues table, in
particular as regards:
research objectives (e.g. study of vulnerable populations, dual use, etc.),
research methodology (e.g. clinical trials, involvement of children and related
consent procedures, protection of any data collected, etc.),
the potential impact of the research (e.g. dual use issues, environmental damage,
stigmatisation of particular social groups, political or financial retaliation,
benefit-sharing, malevolent use , etc.)\\
$\bullet$ provide the documents that you need under national law (if you already have them), e.g.:\\
-- an ethics committee opinion;\\
-- the document notifying activities raising ethical issues or authorizing such activities\\
If these documents are not in English, you must also submit an English summary of them
(containing, if available, the conclusions of the committee or authority concerned).\\
If you plan to request these documents specifically for the project
you are proposing, your request must contain an explicit reference to the project title}

\subsection{Security}

Please indicate if your proposal will involve:

\begin{compactitem}
\item activities or results raising security issues: NO
\item 'EU-classified information' as background or results: NO
\end{compactitem}
\end{proposal}
\TOWRITE{ALL}{Search through final.pdf ('make final') and look for questions marks ?? and XX and YY and XYZ as place holders where people intended to later add a link, or where a link is broken.}
\end{document}

%%% Local Variables:
%%% mode: latex
%%% TeX-master: t
%%% End:

