\begin{sitedescription}{KUL} \label{desc:KUL}

KU Leuven is currently by far the largest university in Belgium in terms of
research funding and expenditure (EUR 426.5 million in 2014), and is a charter
member of LERU. KU Leuven conducts fundamental and applied research in all
academic disciplines with a clear international orientation.  Leuven
participates in over 540 highly competitive European research projects (FP7,
2007-2013), ranking sixth in the league of HES institutions participating in
FP7. In Horizon 2020, KU Leuven currently has been approved 79 projects.

KU Leuven takes up the 9th place of European institutions hosting ERC grants (as
first legal signatories of the grant agreement). To date, the
\href{http://www.kuleuven.be/english/research/EU/p/erc}{78 ERC Grantees}
(including affiliates with VIB and IMEC) in our midst confirm that KU Leuven is
a breeding ground (51 Starting Grants) and attractive destination for the
world's best researchers. The success in the FP7 and Horizon 2020 Marie
Sklodowska Curie Actions is a manifestation of the three pillars of KU Leuven:
research, education and service to society. In our
\href{http://www.kuleuven.be/english/research/EU/p/horizon2020/es/msca}{170
Actions}, of which 76 Initial/European Training Networks, hundreds of young
researchers have been trained through research and have acquired the necessary
skills to transfer their knowledge into the world outside academia.
%
KU Leuven Research \& Development (LRD) is the technology
transfer office (TTO) of the KU Leuven. Since 1972 a multidisciplinary team of
experts guides researchers in their interaction with industry and society, and
the valorisation of their research results (101 spin offs, \dots).

Within the KU Leuven, the Institute for Theoretical Physics (ITF) has 8 permanent staff
members and about 15 PhD students and 10 postdocs in three different areas of Modern
Theoretical Physics: High-Energy Physics, Mathematical Physics and Statistical Physics.
%
The support staff of the ITF (one secretary and one IT support person) ensures excellent
working conditions and, in combination with the administration of the KU Leuven, provides an
ideal environment for developing ambitious research projects.
%
The Department of Physics is host to many excellent researchers and six of its professors
are currently running a ERC grant.

The ITF enjoys regular contacts with the nearby IMEC, a research institute dedicated to
nanoelectronics. Prof. Christian Maes supervises a student jointly with IMEC.

\subsubsection*{Curriculum vitae of the investigators}

\begin{participant}[type=PI,PM=12,gender=male,salary=5500]{Christian Maes}
\url{http://fys.kuleuven.be/itf/staff/christ}

Full professor at the KU Leuven and director of the Institute for Theoretical Physics.

Christian Maes is a leading scientist in the field of mathematical and nonequilibrium statistical mechanics, regularly
invited as a keynote to scientific events worldwide.
%
He has published 150 articles in peer-reviewed journals,
is currently an associate editor or member of the editorial board of 4 leading international journals,
has supervised 14 PhD theses (2 more ongoing) and 11 postdoctoral researchers,
is expert and reviewer for many scientific institutions and
is a member of various international evaluation commissions.

\end{participant}

%%% Local Variables:
%%% mode: latex
%%% TeX-master: "../proposal"
%%% End:

\begin{participant}[type=R,gender=male]{Pierre de Buyl}

Postdoctoral researcher at the KU Leuven. \url{http://pdebuyl.be/}

Pierre de Buyl holds a PhD in Physics (2010) and is specialized in theoretical and
computational studies in statistical physics. He has published 15 papers in international
peer-reviewed journals and also made conference contributions (posters, articles in
conference proceedings and two invited talks). He has worked at the Université libre de
Bruxelles, the University of Toronto and is now at the KUL.
%
His early work is about the consequences of long-ranged interactions on the dynamical
evolution of models relating to plasmas and gravitation.
%
He contributed innovative insight through the direct resolution of the Vlasov equation.
This work provides an alternative to particle-based simulations and a unique point of view
on timely research questions. The resulting simulation code is available under an
open-source license and is the subject of a dedicated publication.
%
His current focus is on so-called nanomotors, a class of devices that transforms a fuel into
motion for the motor (hence the name). The operating conditions are fluctuating and strongly
out of equilibrium, in direct relation with \TheProject.

In addition to his research skills, de Buyl also participates in the organization of
scientific events (web site and database for the {\em European Conference on Complex Systems
  2012}, organizer of the conferences {\em EuroSciPy 2012} and {\em EuroSciPy 2013},
webmaster for EuroSciPy since 2013 and proceedings editor in 2013 and 2014).

de Buyl started his career as a teaching assistant and has taught for several hundred hours
already, from first year physics classes to doctoral training. He has supervised students
for research projects from the third year of Bachelor and for Master Thesis work (one at the
Université libre de Bruxelles in 2013 and one in the coming year at the KU Leuven).

\end{participant}


\begin{participant}[type=res,PM=48,salary=5500]{NN}
A postdoc will be hired to work on the project. We aim to hire someone who has a strong
background in theoretical statistical mechanics and an interest for technology.
\end{participant}

\begin{participant}[type=res,PM=48,salary=3500]{NN}
A PhD student will be hired for the duration of the project that matches the duration of a
PhD thesis in Belgium. The ITF graduates students with an advanced knowledge of statistical
mechanics, ensuring that skilled candidates will exist for the position.
%
The opening will indeed be international and open to competent candidates from any
origin.
\end{participant}

\begin{participant}[type=res,PM=24,salary=3932]{NN}
We will hire an experienced part time project manager to help with the overall management
during the whole duration of \TheProject.
\end{participant}

\subsubsection*{Publications, achievements}

\begin{compactenum}
\item M. Baiesi, C. Maes and B. Wynants {\em Fluctuations and response of nonequilibrium
  states}, Physical Review Letters {\bf 103}, 010602 (2009). This article is already cited
137 times according to Google Scholar, an achievement for a theoretical work in statistical
physics. This article highlights the increasing interest in nonequilibrium physics.
\item {\tt vmf90} software for the numerical resolution of the Vlasov equation:
\url{https://github.com/pdebuyl/vmf90}. This demonstrates the applicant's ability to write
open-source software of scientific relevance.
\end{compactenum}


\subsubsection*{Previous projects or activities}

\begin{compactenum}
\item Organization of the international school ``Fundamental problems in Statistical
Physics''.
%
This school is organized every four years since the 1970s (since 2005 in Leuven) and brings
together the world's most influential experts on Statistical Physics.
\item Membership of expert committees for the Irisch Research Council, the ERC Starting
Grants in Mathematics, of the steering committee of the European Science Foundation
Programme Random Geometry of Large Interacting Systems and Statistical Physics (RGLIS),
among others.
\item Partner in national collaborative projects (Belgian federal government).
\end{compactenum}

\subsubsection*{Significant infrastructure}

As a theoretical research group, there is no significant infrastructure at the ITF.

\end{sitedescription}



\begin{draft}
\vspace{1cm}\TOWRITE{PAR1P1}{Complete check list below -- delete completed items if you wish}

\begin{verbatim}
- [ ] checked that sum of person months put into finance request is
  the same as sum of person months associated with the Work Packages
  (in proposal.tex, as defined as part of the \begin{workpackage}"
  command.
  
- [ ] completed site specific resource summary in resources.tex,
  including table of non-staff costs.

\end{verbatim}
\end{draft}

%%% Local Variables: 
%%% mode: latex
%%% TeX-master: "../proposal"
%%% End: 
