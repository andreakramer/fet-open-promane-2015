\begin{sitedescription}{USTUTT} \label{desc:USTUTT}

{\bf Universität Stuttgart -- A Research University of International Standing:}\\
The Universität Stuttgart lies right in the centre of the largest high-tech region of Europe. We are surrounded by a number of renowned research facilities and have such global players as Daimler or IBM as our neighbours. We were founded in 1829 and over the years this technical institution has developed to the research intensive university that it is today. Our main emphasis is on engineering and the natural sciences.
%
Indicators of our excellent status are the two projects that were successful in the recent {\it Excellence Initiative} sponsored by both the Federal and the State governments. One project is the Cluster of Excellence {\it Simulation Technology} and the other, the Graduate School {\it Advanced Manufacturing Engineering}.

{\bf Experience with EU research funding:}\\
The University of Stuttgart has extensive experience with the various funding programs of the European Commission and has been the {\it leading German university in FP6}, both in number of projects (184) and in terms of funding (54 Mio. \euro). In FP7, it was yet again {\it among the most successful German universities} with 246 projects funded and a total budget of 94 Mio. \euro. The University has consistently been involved in Marie-Curie-projects in previous framework programs. In FP7, it participated in 13 Marie Curie actions.

\subsubsection*{Curriculum vitae of the investigators}

\begin{participant}[type=leadPI,PM=12,gender=male,salary=5500]{Matthias Krüger}

Matthias Krüger joined the University of Stuttgart in October 2012 and is leading the
independent research group {\it Non-equilibrium Systems} located at the Max Planck Institute
for Intelligent Systems, Stuttgart. Before coming to Stuttgart, he was a postdoc at the
Massachusetts Institute of Technology in Cambridge, USA.  In 2009, he gained his Doctoral
degree as a theoretical physicist at the University of Konstanz, Germany.

He theoretically studies different aspects of nonequilibrium statistical physics, including
fluids far from equilibrium, fluctuation (Casimir) forces under nonequilibrium conditions,
as well as radiative energy transport on the nano-scale.
%
He recently received an Emmy Noether Grant from the German Research Foundation (DFG), a
prestigious program that allows young researchers early independence.  Before, he was
supported through other programs of DFG and Fulbright and is a regular committee member of
the German National Academic Foundation.

He published around 30 articles in peer-reviewed journals, and, including ongoing projects, supervised 2 postdocs, 1 Phd student and 3 undergraduate students.

\end{participant}
\begin{participant}[type=PI,PM=12,gender=male,salary=5500]{Clemens Bechinger}

Full professor and head of the 2nd Experimental Institute at the University
of Stuttgart. Fellow of the Max Planck Institute of Intelligent Systems.

%

Clemens Bechinger is an expert in the field of experimental soft matter
systems and regularly invited as keynote and plenary speaker in
international scientific meetings.

%

He has published about 125 articles in peer-reviewed journals (also in
Nature, Science, PNAS, PRL) and is member of the liquid matter board of the
EPS and the Panel ``Statistical Physics, Soft Matter, Biophysics, Nonlinear
Dynamics'' of the German Research Society. Since his arrival in Stuttgart in
2003 he has supervised 10 Postdocs, 18 Phd Students and 22 Master Students.

\end{participant}

\begin{participant}[type=res,PM=48,salary=5500]{NN}
Postdoctoral Researcher.
\end{participant}
\begin{participant}[type=res,PM=36,salary=5500]{NN}
PhD Student.
\end{participant}

\subsubsection*{Publications, achievements}

\begin{compactenum}
\item Matthias Krüger developped theoretical frameworks predicting properties of
far-from-equilibrium fluids, especially near interfaces and for glassy viscoelastic
materials and he contributed to the description of electromagnetic fluctuations in
non-equilibrium situations (e.g. in Physical Review Letters).
\item Clemens Bechinger recently published experimental articles on (non-equilibrium)
stochastic thermodynamics, self propelled swimmers, critical Casimir forces, quasicrystals,
a micrometer sized heat engine and nanotribology (in Nature, Nature Materials, Nature
Physics, Nature Comm. and Physical Review Letters).
\end{compactenum}

\subsubsection*{Previous projects or activities}

\begin{compactenum}
\item During the past five years, Clemens Bechinger was funded by several programs, e.g. by
two {\it Individual Grants} and a {\it Priority Progam} of the German Research Foundation
(DFG). His research was also supported as a {\it Max-Planck-Fellow} as well as by the {\it
  Initial Training Networks} of the Marie Curie Actions.
\item Matthias Krüger held an {\it Individual Fellowship} of the DFG starting 2010 and is
currently supported through a junior group grant ({\it Emmy Noether Program}) funded by the
DFG.
\end{compactenum}

\subsubsection*{Significant infrastructure}
The research group in Stüttgart is equipped for the study of active particles systems, of
visco-elastic media and of colloids in general.
%
This infrastructure plays a critical role in the development of \TheProject.

\end{sitedescription}
