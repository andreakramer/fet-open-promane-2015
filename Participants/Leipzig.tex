\begin{sitedescription}{ULEI} \label{desc:ULEI}

 Universität Leipzig was founded in 1409 and is the second oldest university in Germany where teaching has continued without interruption. 
  Today it offers a wide spectrum of academic disciplines at 14 faculties with more than 150 institutes. 
  It is a member of the German U15, a strategic alliance of 15 major German research universities. 
  The node participants are located at the Institute for Experimental Physics I and at the Institute for Theoretical Physics, respectively, which are parts of the Faculty of Physics and Earth Sciences. 
  The faculty also comprises institutes for meteorology, geology, geophysics and geography. 
  It counts among the leading faculties in terms of research output and external funding, within the university, and hosts several ERC grantees. 
  It is also the first faculty of the university that recently got its research activities evaluated by an external board of international experts. 
  The physics institutes make major contributions to several collaborative research centers (SFBs) funded by the German Science Foundation (DFG) and to the interdisciplinary Graduate School ``Leipzig School of Natural Sciences -- Building with Molecules and Nano-objects'' (www.buildmona.de), founded by a grant from the German Excellence Initiative, which has so far enrolled close to 200 PhD candidates. 
  They maintain collaborations and joint grants with a large number of independent international and local research institutes for fundamental and applied science and industrial partners. 

\subsubsection*{Curriculum vitae of the investigators}

\begin{participant}[type=R,PM=12,gender=male,salary=5500]{Klaus Kroy}

Professor at the Institute of Theoretical Physics, Universität Leipzig.

Klaus Kroy is a theoretical physicist and an expert in the field of
soft mesoscopics (non-equilibrium dynamics of colloids and polymers; active particles; cytoskeleton and tissue mechanics; single-molecule force spectroscopy; aeolian sand transport and structure formation)

He has published about 60 articles in peer-reviewed journals (also in
Nature Physics, Nature Communications, PNAS, PRL) 

He has in the past supervised 3 postdocs, 5 PhD students, and 19
master students 

He is a Member of the German Physical Society, of the International Max--Planck Research Group
Mathematics in the Sciences (Leipzig), and he recently received grants from
the German Excellence Initiative (Graduate School ``BuildMoNa''),
the DFG-Forschergruppe FOR 877, the German
priority programm SPP1726 (DFG), the German Israel
Foundation, the ESF, and a DFG-individual-grant.


\end{participant}

\begin{participant}[type=PI,PM=12,gender=male,salary=5500]{Frank Cichos}

Professor at the Institute of Experimental Physics I, Universität Leipzig. \url{http://www.uni-leipzig.de/~mona}

Frank Cichos is an experimental physicist and an expert optical microscopy and optical single molecule detection
(photothermal single molecule detection; active particles; single molecule trapping; single molecule dynamics in soft matter).
He has in the past published about 76 articles in peer-reviewed journals (also in Nano Letters, ACS nano and PRL) 
and supervised 2 postdocs, 15 PhD students, and about 20 master students.
He is a Member of the German Physical Society and the American Physical Society. He recently received grants from the German
Excellence Initiative (Graduate School ``BuildMoNa''), the DFG-Forschergruppe FOR 877, the German priority program SPP1726 
(DFG), the DFG Sonderforschungsbereich TRR102 and a joint  DFG-ANR-individual-grant. He is the co-speaker of the 
DFG Sonderforschungsbereich TRR102 and has been the speaker of the DFG-Forschergruppe FOR 877.

\end{participant}


\begin{participant}[type=res,PM=48,salary=5500]{NN}
\end{participant}
\begin{participant}[type=res,PM=36,salary=5500]{NN}

We need researchers. Two, for instance.

\end{participant}

\subsubsection*{Publications, achievements}

\begin{compactenum}
\item Leadership.
M. Gralka, K. Kroy, Inelastic mechanics: A unifying principle in biomechanics. 
Biochimica et Biophysica Acta (BBA)--Molecular Cell Research 2015

S. Schöbl, S. Sturm, W. Janke, K. Kroy, Persistence-Length Renormalization of Polymers in a Crowded Environment of Hard Disks.
Physical Review Letters 113 (2014) 238302.

J. T. Bullerjahn, S. Sturm, K. Kroy, Theory of rapid force spectroscopy. Nature Communications 5 (2014) 4463.

O. Otto, S. Sturm, N. Laohakunakorn, U. F. Keyser, K. Kroy, Rapid internal contraction boosts DNA friction.
Nature Communications 4 (2013) 1780.

D. Chakraborty, M. V. Gnann, D. Rings, J. Glaser, F. Otto, F. Cichos, K. Kroy, 
Generalised Einstein relation for hot Brownian motion. EPL (Europhysics Letters) 96 (2011) 60009.

\item Coauthoring.
\end{compactenum}

\subsubsection*{Previous projects or activities}

\begin{compactenum}
\item Organization.
\item Partner.
\end{compactenum}

\subsubsection*{Significant infrastructure}


\end{sitedescription}

\begin{draft}
\vspace{1cm}\TOWRITE{PAR1P1}{Complete check list below -- delete completed items if you wish}

\begin{verbatim}
- [ ] checked that sum of person months put into finance request is
  the same as sum of person months associated with the Work Packages
  (in proposal.tex, as defined as part of the \begin{workpackage}"
  command.
  
- [ ] completed site specific resource summary in resources.tex,
  including table of non-staff costs.

\end{verbatim}
\end{draft}

%%% Local Variables: 
%%% mode: latex
%%% TeX-master: "../proposal"
%%% End: 
