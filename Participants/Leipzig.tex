\begin{sitedescription}{ULEI} \label{desc:ULEI}

 Universität Leipzig was founded in 1409 and is the second oldest university in Germany where teaching has continued 
 without interruption. 
  Today it offers a wide spectrum of academic disciplines at 14 faculties with more than 150 institutes. 
  for close to 30 000 students taught by approximately 400 Professors, 
  resulting in more than 570 advanced degrees (PhD, etc.) annually.
  It is a member of the German U15, a strategic alliance of 15 major German research universities. 
  In the recently finished 7. Research Framework Programme of the EU the university participated 
  in 75 projects with an overall EC-contribution of some 25 million Euro.

  The node participants are located at the Institute for Experimental Physics I and at the Institute for Theoretical Physics, 
  of the Faculty of Physics and Earth Sciences, which also comprises institutes for meteorology, geology, geophysics 
  and geography. 
  It counts among the leading faculties in terms of research output and external funding, within the university, 
  and hosts several ERC grantees. 
  It is the first faculty of the university that recently got its research activities evaluated by an external board of
  international experts. 
  The physics institutes make major contributions to several collaborative research centers (SFBs) funded by the 
  German Science Foundation (DFG) and to the interdisciplinary Graduate School ``Leipzig School of Natural Sciences -- 
  Building with Molecules and Nano-objects'' (www.buildmona.de), founded by a grant from the German Excellence Initiative,
  which has so far enrolled close to 200 PhD candidates. 
  They maintain collaborations and joint grants with a large number of independent international 
  and local research institutes for fundamental and applied science and industrial partners. 

\subsubsection*{Curriculum vitae of the investigators}

\begin{participant}[type=R,PM=12,gender=male,salary=5500]{Klaus Kroy}

Professor at the Institute of Theoretical Physics, Universität Leipzig.

Klaus Kroy is a theoretical physicist and an expert in the field of
soft mesoscopics (non-equilibrium dynamics of colloids and polymers; active particles; cytoskeleton and tissue mechanics; single-molecule force spectroscopy; aeolian sand transport and structure formation)

He has published about 60 articles in peer-reviewed journals (also in
Nature Physics, Nature Communications, PNAS, PRL) 

He has in the past supervised 3 postdocs, 5 PhD students, and 19
master students 

He is a Member of the German Physical Society, of the International Max--Planck Research Group
Mathematics in the Sciences (Leipzig), and he recently received grants from
the German Excellence Initiative (Graduate School ``BuildMoNa''),
the DFG-Forschergruppe FOR 877, the German
priority programm SPP1726 (DFG), the German Israel
Foundation, the ESF, and a DFG-individual-grant.


\end{participant}

\begin{participant}[type=PI,PM=12,gender=male,salary=5500]{Frank Cichos}

Professor at the Institute of Experimental Physics I, Universität Leipzig. \url{http://www.uni-leipzig.de/~mona}

Frank Cichos is an experimental physicist and an expert optical microscopy and optical single molecule detection
(photothermal single molecule detection; active particles; single molecule trapping; single molecule dynamics in soft matter).
He has in the past published about 76 articles in peer-reviewed journals (also in Nano Letters, ACS nano and PRL) 
and supervised 2 postdocs, 15 PhD students, and about 20 master students.
He is a Member of the German Physical Society and the American Physical Society. He recently received grants from the German
Excellence Initiative (Graduate School ``BuildMoNa''), the DFG-Forschergruppe FOR 877, the German priority program SPP1726 
(DFG), the DFG Sonderforschungsbereich TRR102 and a joint  DFG-ANR-individual-grant. He is the co-speaker of the 
DFG Sonderforschungsbereich TRR102 and has been the speaker of the DFG-Forschergruppe FOR 877.

\end{participant}


\begin{participant}[type=res,PM=48,salary=5500]{NN}
A postdoc will be hired to work on the project. We aim to hire someone who has a strong
background in theoretical statistical mechanics and a diverse research experience. The person will keep contact to the other project nodes and the experimental project partners.
\end{participant}

\begin{participant}[type=res,PM=36,salary=5500]{NN}
A postdoc researcher is required to carry out the experiments on hot Brownian motion at nanosecond timescales. The postdoc should have a solid background in optical tweezer experiments and fast positional detection. He shall work closely with the theoretical project partners and keep close contact to the other nodes.
\end{participant}

\subsubsection*{Publications, achievements}

\begin{compactenum}
\item K. Kroy and F. Cichos have published a series of papers on Hot Brownian Motion, including joint articles such as 
Physical Review Letters {\bf 105} 090604 (2010), which has been cited 81 times according to Google Scholar, 
highlighting the interest in fundamental non-equilibrium fluctuation dissipation theorems and their experimental verification.

\item Further pertinent publications by the PIs include Nano Lett. {\bf 15} 5499 (2015), 
Biochimica et Biophysica Acta -- Molecular Cell Research {\bf 1853} 3025 (2015),
Nature Communications {\bf 4} 1780 (2013) and {\bf 5} 4463 (2014),
Nature Nanotech. {\bf 9} 415 (2014), Physical Review Letters {\bf 113} 238302 (2014), 
ACS Nano {\bf 6} 2714 (2012) and {\bf 8} 6542 (2014), EPL {\bf 96} 60009 (2011).

%\item M. Gralka, K. Kroy, Inelastic mechanics: A unifying principle in biomechanics. 
% Biochimica et Biophysica Acta (BBA)--Molecular Cell Research 2015

%\item S. Schöbl, S. Sturm, W. Janke, K. Kroy, Persistence-Length Renormalization of Polymers in a Crowded Environment of Hard
% Disks. Physical Review Letters {\bf 113} 238302 (2014).

%\item J. T. Bullerjahn, S. Sturm, K. Kroy, Theory of rapid force spectroscopy. Nature Communications {\bf 5} 4463 (2014).

%\item O. Otto, S. Sturm, N. Laohakunakorn, U. F. Keyser, K. Kroy, Rapid internal contraction boosts DNA friction.
% Nature Communications {\bf 4} 1780 (2013).

%\item D. Chakraborty, M. V. Gnann, D. Rings, J. Glaser, F. Otto, F. Cichos, K. Kroy, 
% Generalised Einstein relation for hot Brownian motion. EPL (Europhysics Letters) {\bf 96} 60009 (2011).

%\item D. Rings, R. Schachoff, M. Selmke, F. Cichos, K. Kroy. 
%Hot Brownian Motion.  Physical Review Letters {\bf 105},  090604 (2010)

%\item M. Braun, A. Bregulla, K. Günther, N. Mertig, F. Cichos. Single Molecules Trapped by Dynamic Inhomogeneous 
%Temperature Fields Nano Lett. 15 5499 (2015).

%\item A. Bregulla, H. Yang, F. Cichos Stochastic Localization of Micro-Swimmers by Photon Nudging ACS Nano 8 6542 (2014).

%\item M. Selmke, M. Braun, F. Cichos Photothermal Single Particle Microscopy: Detection of a Nanolens ACS Nano 6 2714 (2012).

\end{compactenum}

\subsubsection*{Previous projects or activities}

\begin{compactenum}
%
\item Joint organization of the Diffusion Fundamentals Conference V (2013). 
Joint Initiation of a workshop series "Hot Nanostructures"
(in 2011) continued every two years. It is focussed on non-equilibrium physics highlighting theory and experiments 
involving large temperature gradients and will return to Leipzig in 2017. 

\item Project lead (speaker) of the research unit 877 ``From Local Constraints to Macroscopic Transport'' of the German 
Science Foundation (DFG).

\item Partners in international collaborative projects (German Excellence Initiative, Agence National de la Recherche - 
German Science Foundation, DFG, German Israel Foundation, ESF)
\end{compactenum}

\subsubsection*{Significant infrastructure}
The Institute of Theoretical Physics hosts its own water-cooled computer cluster, 
involving a subcluster of GPU servers reserved for the PI's massively parallel NEMD simulations, 
which provides sufficient resources for the project tasks.
\end{sitedescription}

\begin{draft}
\vspace{1cm}\TOWRITE{PAR1P1}{Complete check list below -- delete completed items if you wish}

\begin{verbatim}
- [ ] checked that sum of person months put into finance request is
  the same as sum of person months associated with the Work Packages
  (in proposal.tex, as defined as part of the \begin{workpackage}"
  command.
  
- [ ] completed site specific resource summary in resources.tex,
  including table of non-staff costs.

\end{verbatim}
\end{draft}

%%% Local Variables: 
%%% mode: latex
%%% TeX-master: "../proposal"
%%% End: 
