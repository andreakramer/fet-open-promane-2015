\begin{sitedescription}{KUL} \label{desc:KUL}

KU Leuven boasts a rich tradition of education and research that dates back six
centuries. The university's basic research orientation has always been and will
remain fundamental research. At the same time, the university remains vigilantly
open to contemporary cultural, economic and industrial realities, as well as to
the community's needs and expectations. From a basis of social responsibility
and scientific expertise, KU Leuven provides high-quality, comprehensive health
care, including specialised tertiary care, in its University Hospitals. In doing
so it strives toward optimum accessibility and respect for all patients.

KU Leuven is currently by far the largest university in Belgium in terms of
research funding and expenditure (EUR 426.5 million in 2014), and is a charter
member of LERU. KU Leuven conducts fundamental and applied research in all
academic disciplines with a clear international orientation.  Leuven
participates in over 540 highly competitive European research projects (FP7,
2007-2013), ranking sixth in the league of HES institutions participating in
FP7. In Horizon 2020, KU Leuven currently has been approved 79 projects.

KU Leuven takes up the 9th place of European institutions hosting ERC grants (as
first legal signatories of the grant agreement). To date, the
\href{http://www.kuleuven.be/english/research/EU/p/erc}{78 ERC Grantees}
(including affiliates with VIB and IMEC) in our midst confirm that KU Leuven is
a breeding ground (51 Starting Grants) and attractive destination for the
world's best researchers. The success in the FP7 and Horizon 2020 Marie
Sklodowska Curie Actions is a manifestation of the three pillars of KU Leuven:
research, education and service to society. In our
\href{http://www.kuleuven.be/english/research/EU/p/horizon2020/es/msca}{170
Actions}, of which 76 Initial/European Training Networks, hundreds of young
researchers have been trained through research and have acquired the necessary
skills to transfer their knowledge into the world outside academia.


KU Leuven employs 8,671 researchers on its academic staff (2014). To strengthen
international collaboration, KU Leuven has its own international research
fellowship programme and supports international scholars in international
funding applications. KU Leuven Research \& Development (LRD) is the technology
transfer office (TTO) of the KU Leuven. Since 1972 a multidisciplinary team of
experts guides researchers in their interaction with industry and society, and
the valorisation of their research results (101 spin offs, \dots).

\subsubsection*{Curriculum vitae of the investigators}

\begin{participant}[type=PI,PM=12,gender=male,salary=5500]{Christian Maes}

Full professor at the KUL and director of the Institute for Theoretical Physics.
%
Christian Maes is a leading scientist in the field of statistical mechanics, regularly
invited as a keynote to scientific events.
%
He has published 150 articles in peer-reviewed journals,
is currently an associate editor or member of the editorial board of 4 international journals,
has supervised 14 PhD theses (2 more ongoing) and 11 postdoctoral researchers,
is expert and reviewer for many scientific instituttions and
is a member of the evalution commission for ERC Starting Grants in Mathematics since 2014.

\end{participant}

%%% Local Variables:
%%% mode: latex
%%% TeX-master: "../proposal"
%%% End:

\begin{participant}[type=R,PM=48,gender=male,salary=5500]{Person Two}

  Person Two is an experienced researcher and will participate full time to the
  project.

\end{participant}

%%% Local Variables:
%%% mode: latex
%%% TeX-master: "../proposal"
%%% End:


\begin{participant}[type=res,PM=48,salary=5500]{NN}
\end{participant}
\begin{participant}[type=res,PM=36,salary=5500]{NN}

We need researchers. Two, for instance.

\end{participant}

\begin{participant}[type=res,PM=24,salary=3932]{NN}
  We will hire an experienced part time project manager to help with
  the overall management during the whole duration of \TheProject.
\end{participant}

\subsubsection*{Publications, achievements}

\begin{compactenum}
\item Leadership.
\item Coauthoring.
\end{compactenum}


\subsubsection*{Previous projects or activities}

\begin{compactenum}
\item Hosting.
\item Co-organising.
\end{compactenum}

\subsubsection*{Significant infrastructure}

We have building, at PAR1.

\end{sitedescription}



\begin{draft}
\vspace{1cm}\TOWRITE{PAR1P1}{Complete check list below -- delete completed items if you wish}

\begin{verbatim}
- [ ] checked that sum of person months put into finance request is
  the same as sum of person months associated with the Work Packages
  (in proposal.tex, as defined as part of the \begin{workpackage}"
  command.
  
- [ ] completed site specific resource summary in resources.tex,
  including table of non-staff costs.

\end{verbatim}
\end{draft}

%%% Local Variables: 
%%% mode: latex
%%% TeX-master: "../proposal"
%%% End: 
