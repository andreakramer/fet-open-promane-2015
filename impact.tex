\subsection{Expected impacts}

\eucommentary{
  Please be specific, and provide only information that applies to the proposal
  and its objectives. Wherever possible, use quantified indicators and targets.
%
  \begin{compactitem}
  \item Describe how your project will contribute to the expected impacts set out
    in the work programme under the relevant topic.
  \item Describe the importance of the technological outcome with regards to its
    transformational impact on science, technology and/or society.
  \item Describe the empowerment of new and high-potential actors towards future
    technological leadership.
  \end{compactitem}
}

\TheProject is based on operational objectives and will be implemented experimentally and
via numerical simulations.
%
The nature of the tools that we develop is however adequate to serve for many more
applications, as the last century of statistical mechanics developments has shown.
%
Nonequilibrium theory and phenomenology reach even much richer grounds and connects us to
\begin{compactitem}
\item Biological systems and medical applications, in which noninvasive diagnosis and
micro-manipulation of macromolecules XXX.
\item Design of new materials and monitoring of material properties beyond our ideas on
compressibility. Indeed, the response of a material can also be of thermal or chemical
nature.
\item Cybernetics of active media and the steering of active particles in biological
(living) environments is expected to revolutionize medical interventions and pharmacy via
the targeted transport of drugs and new microinvasive treatments and therapies.
\item Micro-bioreactors are small chemical factories whose direct control is
unlikely. Thermodynamical control on the other hand is a robust tool.
\item Quantum devices require the control of many body coherence and stability for reaching
macroscopic phases that are only metastable under equilibrium conditions. This is among
other fields relevant for quantum computing applications.
\end{compactitem}

We see in all of the above items the ubiquitous role of the nonequilibrium paradigm. Its
transformational impact on science, technology and society are enormous and largely
unexploited. \TheProject promises a serious start toward these applications via
nonequilibrium physics.
%
Nonequilibrium theories are currently not used in the industry and we will provide the
building blocks to enable it. We expect that the developement of nonequilibrium theories
will deliver a competitive advantage, similar in principle to the one delivered by James
Watt developments for the steam engine.

\subsection{Measures to maximise impact}

\paragraph{a) Dissemination and exploitation of results}

The contents of \TheProject will be communicated mostly via academic publications. This is
the most natural medium for most of the partners.
%
Our published works meet the open-access character via scientific repositories
(i.e. \url{http://arxiv.org/}) and also via paid open-access fees to regular journals,
effectively making our work free to read.
%
The members of the consortium also have editorial responsibilities that would facilitate the
preparation of dedicated issues.
%
Our modeling strategy will be made available in an open-source simulation program.

As the project deals with some very novel aspects that have not appeared in standard or less
standard treatments of irreversible thermodynamics, we will targer a larger academic and
industrial audience via a dedicated summer school on nonequilibrium thermodynamics. This
school will take place in conjunction with one of the project's meetings and participation
will be open to all.
%
The project will also invest (still under other direct costs) in bringing together various
points of view on the role of dynamical activity and the construction of nonequilibrium
statistical mechanics. For that reason we foresee to organize two major conferences, one at
around month 20, the other towards the end of the project.

A specific deliverable is aimed at filling the gap that too often prevents academic results
to reach industry. We will publish twice a ``technological report'' in which we present the
links between our findings and the corresponding technologies.

\paragraph{b) Communication activities}

\eucommentary{
  \begin{compactitem}
  \item Describe the proposed communication measures for promoting the project
    and its findings during the period of the grant. Measures should be
    proportionate to the scale of the project, with clear objectives. They
    should be tailored to the needs of various audiences, including groups
    beyond the project's own community. Where relevant, include measures for
    public/societal engagement on issues related to the project.
  \end{compactitem}
}

Various members of the team have larger networks of communication both towards the general
public (public outreach) as towards industrial players and groups of interest.  The first
aim there is to transfer knowledge and tools, and to help in optimizing public interest. An
important initiative will be the organization of ``physics meets industry days'' where each
time during a week, specific problems of industry or economic activity will be presented.
They will be treated by students and experts towards helping to solve these problems, with
direct feedback toward the industry of company.  Such initiatives exist already in some
countries but will be started up in other European countries, and with an additional
selection and expertise platform related to complex and nonequilibrium phenomena.



%%% Local Variables:
%%% mode: latex
%%% TeX-master: "proposal"
%%% End:
