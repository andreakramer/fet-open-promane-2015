\subsection{Expected impacts}

\eucommentary{
  Please be specific, and provide only information that applies to the proposal
  and its objectives. Wherever possible, use quantified indicators and targets.
%
  \begin{compactitem}
  \item Describe how your project will contribute to the expected impacts set out
    in the work programme under the relevant topic.
  \item Describe the importance of the technological outcome with regards to its
    transformational impact on science, technology and/or society.
  \item Describe the empowerment of new and high-potential actors towards future
    technological leadership.
  \end{compactitem}
}

\TheProject is based on operational objectives and will be implemented experimentally and
via numerical simulations.
%
The nature of the tools that we develop is however adequate to serve for many more
applications, as the last century of statistical mechanics developments has shown.
%
Nonequilibrium theory and phenomenology reach even much richer grounds and connects us to
\begin{compactitem}
\item Biological systems and medical applications, in which noninvasive diagnosis and
micro-manipulation of macromolecules XXX.
\item Design of new materials and monitoring of material properties beyond our ideas on
compressibility. Indeed, the response of a material can also be of thermal or chemical
nature.
\item Cybernetics of active media and the steering of active particles in biological
(living) environments is expected to revolutionize medical interventions and pharmacy via
the targeted transport of drugs and new microinvasive treatments and therapies.
\item Micro-bioreactors are small chemical factories whose direct control is
unlikely. Thermodynamical control on the other hand is a robust tool.
\item Quantum devices require the control of many body coherence and stability for reaching
macroscopic phases that are only metastable under equilibrium conditions. This is among
other fields relevant for quantum computing applications.
\end{compactitem}

We see in all of the above items the ubiquitous role of the nonequilibrium paradigm. Its
transformational impact on science, technology and society are enormous and largely
unexploited. \TheProject promises a serious start toward these applications via
nonequilibrium physics.
%
Nonequilibrium theories are currently not used in the industry and we will provide the
building blocks to enable it. We expect that the developement of nonequilibrium theories
will deliver a competitive advantage, similar in principle to the one delivered by James
Watt developments for the steam engine.

\subsection{Measures to maximise impact}

\paragraph{a) Dissemination and exploitation of results}

The major method of dissemination of our results is first the standard scientific practice
of open access publishing, and participation in international conferences and discussion
fora. The universities have important research and development centres where starting
spin-offs find resources and support.
%
There will be important contacts with these centres.

Dissemination of the results is done by the free web archive, in publications in the
standard specialized scientific journals (mostly non-commercial), by talks, by schooling and
in contacts in conferences. As members of the Editorial Boards of Journal of Physics A  (IOP), of the Journal of Statistical Physics (Springer), of Annales Henri Poincar\'e, of
Fundamental Theories of Physics (Springer) and of the liquid matter board of the European Physical Society, etc we can organize special issues and create
special volumes dedicated to aspects of the project. Indeed, as the project deals with some
very novel aspects that have not appeared in standard or even not so standard treatments of
irreversible thermodynamics, an important effort will be necessary and will be made to reach
also less specialized researchers as well as scientists involved in possible applications.
%
Obviously an important cost then also goes to traveling and participation in conferences and
international discussions. Again, since the project is not entirely mainstream and some
concepts/questions are quite new, the project needs to invest in visibility and in numerous
exchanges on international platforms.  All team members will be
expected to show great mobility.  During the years, the various members meet once at each
node with everybody, at a rate of roughly one meeting every 10 months. PIs stay a few days
for talks of everyone and discussions. Postdocs and students can stay
longer, extending their visits for a month. That will allow perfect exchange of ideas and
progress.  Similar considerations apply for inviting scientists where
costs include traveling, housing and subsistence.  The project will also invest (still under other
direct costs) in bringing together various points of view on the role of dynamical activity
and the construction of nonequilibrium statistical mechanics. For that reason we foresee to
organize two major conferences, one at around month 20, the other towards the end of the
project.  


\eucommentary{
  \begin{compactitem}
  \item Provide a plan for disseminating and exploiting the project results. The
    plan, which should be proportionate to the scale of the project, should
    contain measures to be implemented both during and after the project.
  \item Explain how the proposed measures will help to achieve the expected
    impact of the project.
  \item Where relevant, include information on how the participants will manage
    the research data generated and/or collected during the project, in
    particular addressing the following issues\footnote{For further guidance on
      research data management, please refer to the H2020 Online Manual on the
      Participant Portal.}:
    \begin{compactitem}
    \item What types of data will the project generate/collect?  o What
      standards will be used?
    \item How will this data be exploited and/or shared/made accessible for
      verification and re-use?  If data cannot be made available, explain why.
    \item How will this data be curated and preserved?
    \end{compactitem}
 %      
    You will need an appropriate consortium agreement to manage (among other
    things) the ownership and access to key knowledge (IPR, data etc.). Where
    relevant, these will allow you, collectively and individually, to pursue
    market opportunities arising from the project's results.\\
%
    The appropriate structure of the consortium to support exploitation is
    addressed in section~3.3.
%
  \item Outline the strategy for knowledge management and protection. Include
    measures to provide open access (free on-line access, such as the 'green' or
    'gold' model) to peer- reviewed scientific publications which might result
    from the project.%
    \footnote{Open access must be granted to all scientific publications
      resulting from Horizon 2020 actions. Further guidance on open access is
      available in the H2020 Online Manual on the Participant Portal.}\\
%
    Open access publishing (also called 'gold' open access) means that an
    article is immediately provided in open access mode by the scientific
    publisher. The associated costs are usually shifted away from readers, and
    instead (for example) to the university or research institute to which the
    researcher is affiliated, or to the funding agency supporting the research.\\
%
    Self-archiving (also called 'green' open access) means that the published
    article or the final peer-reviewed manuscript is archived by the researcher
    - or a representative - in an online repository before, after or alongside
    its publication.  Access to this article is often - but not necessarily -
    delayed ('embargo period'), as some scientific publishers may wish to recoup
    their investment by selling subscriptions and charging pay-per-download/view
    fees during an exclusivity period.
%
  \end{compactitem}
}

\paragraph{b) Communication activities}

\eucommentary{
  \begin{compactitem}
  \item Describe the proposed communication measures for promoting the project
    and its findings during the period of the grant. Measures should be
    proportionate to the scale of the project, with clear objectives. They
    should be tailored to the needs of various audiences, including groups
    beyond the project's own community. Where relevant, include measures for
    public/societal engagement on issues related to the project.
  \end{compactitem}
}

Various members of the team have larger networks of communication both towards the general
public (public outreach) as towards industrial players and groups of interest.  The first
aim there is to transfer knowledge and tools, and to help in optimizing public interest. An
important initiative will be the organization of ``physics meets industry days'' where each
time during a week, specific problems of industry or economic activity will be presented.
They will be treated by students and experts towards helping to solve these problems, with
direct feedback toward the industry of company.  Such initiatives exist already in some
countries but will be started up in other European countries, and with an additional
selection and expertise platform related to complex and nonequilibrium phenomena.



%%% Local Variables:
%%% mode: latex
%%% TeX-master: "proposal"
%%% End:
