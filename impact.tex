\subsection{Expected impacts}

\eucommentary{
  Please be specific, and provide only information that applies to the proposal
  and its objectives. Wherever possible, use quantified indicators and targets.
%
  \begin{compactitem}
  \item Describe how your project will contribute to the expected impacts set out
    in the work programme under the relevant topic.
  \item Describe the importance of the technological outcome with regards to its
    transformational impact on science, technology and/or society.
  \item Describe the empowerment of new and high-potential actors towards future
    technological leadership.
  \end{compactitem}
}

Statistical mechanics by itself had an enormous impact on technological and economic
developments in the last century. Its framework not only inspired the logical framework
of modeling and analysing microscopics to reach mesoscopic and macroscopic scales, but it
has very specifically also enabled {\it ab initio} schemes and algorithms to reach and predict
macroscopic behavior.
%
The theoretical foundations of solid state physics and quantum field theory that have reshaped the
modern world and brought major changes in daily life invariably are consequences of the
fluctuation theories pioneered in statistical mechanics. Yet most, if not all, of that is
restricted to equilibrium statistical mechanics with its powerful and general Gibbs
machinery.
%
While that formalism and the consequent algorithms and implementations remain a superb tool
in technological research and innovation, the present project turns to nonequilibrium theory
and phenomenology. There we reach even much richer grounds, also connecting with biological
systems and hence also medical applications, or with new applications for material properties and nonlinear fluids. The expected impact here will therefore add to
the well-proven and established record of statistical mechanics in general, while opening
new avenues in the following directions:
\begin{compactitem}
\item new materials and material properties: we have in mind here extending the systematics
  and the solid theoretical base in the search of good and useful biomaterials and
  meta-materials.  Transport theory beyond linear response (close to equilibrium) and the
  emerging theory of statistical forces beyond equilibrium will allow systematic and
  algorithmic searches for new properties and materials.
\item Biomedical processes and transport: the cybernetics of active media and the steering
  of active particles in biological (living) environments is expected to revolutionize
  medical interventions and pharmacy. Localized and targeted transport of drugs and signals
  in biomaterials is essential for new microinvasive treatments and therapies. That requires
  the understanding and modeling of interacting particle systems in fluctuating
  nonequilibrium environments and the competence of realizing complex algorithms for
  transport.
\item Biomechanical structures and motility: Similar to the previous point, we need to
  understand much better the structural and architectural aspects of living materials such
  as the cytoskeleton of a cell or in general the mechanical behavior of tissue.  That is
  essential also for understanding the motility of cells and for the control of transduction
  of signals to cell interior.  In our vision, basic nonequilibrium physics insights are
  crucial here, and complement the more biochemical and experimental research there.
\item Quantum stability and coherence control: One of the major difficulties in upscaling
  the wonders of quantum mechanics is the loss of coherence when in contact with large
  thermal environments of when involving a great number of degrees of freedom.  We have obtained theoretical proof that
  when we bring the system in contact, not with an equilibrium thermal environment, but with a
  suitably driven nonequilibrium medium with well-defined and controlled properties, that in fact
  would control the obtained coherence of immersed (quantum) devices.
\item Transient and fast processes: A lot of industrial production relies on processes that
  occur far from equilibrium.  Parameters like stresses and temperatures are changing very
  rapidly, and the products are subject to transient or strongly time-dependent
  environments. That can only be controlled more systematically from the point of view of
  nonequilibrium physics where response theory also for transient and nonequilibrium
  processes are in full development.
\item Economy and society: Econophysics, financial mathematics and socio-physics all rely on
  models and insights of nonequilibrium physics. We expect that better insights in
  nonequilibrium processes can contribute to implement beneficial algorithms in economic or
  societal policy.  As one example, the idea of active and interacting agents that give rise
  to a collective macroscopic behavior, be it economic or political, is an interesting or
  even crucial addition to existing models if only for grasping nonequilibrium phenomena
  like clustering, flocking, heavy tails or non-additivity of forces.
\item Climate science and weather prediction: It goes almost without saying that climate
  science and meteorology are fundamentally based on nonequilibrium physics as realized more
  specifically in fluid mechanics and atmosphere dynamics. Statistical modeling and
  understanding rare events are more and more central and most relevant issues in these. The
  behavior of probes in turbulent flow, the modeling of a global weather process and the
  setting up of reliable predictive capacity are important research themes for some of
  today's world problems and nonequilibrium physics will obviously be a key-ingredient in
  any solution or advancement.
\item Eco-science and sustainable energy: New ways of protecting the planet, keeping life
  interesting and have sustainable resources will rely on understanding non-dissipative
  aspects of physical processes.  Dissipation governs most of equilibrium physics, but
  beyond there are possibilities of self-organization and stabilization of structures and
  phases that are unstable or even non-existing in equilibrium.
\end{compactitem}
In all of the above one recognizes the start of a major new ingredient in dealing with
certain processes and technologies. The importance of the nonequilibrium paradigm and its
transformational impact on science, technology and society are enormous and largely
unexploited. The present project wants to take that serious and start the far reaching way
of implementing far from equilibrium physics. The more direct and specific impacts are related to the objectives expressed in the work packages.

\subsection{Measures to maximise impact}

\paragraph{a) Dissemination and exploitation of results}

The major method of dissemination of our results is first the standard scientific practice
of open access publishing, and participation in international conferences and discussion
fora. The universities have important research and development centres where starting
spin-offs find resources and support.
%
There will be important contacts with these centres.

Dissemination of the results is done by the free web archive, in publications in the
standard specialized scientific journals (mostly non-commercial), by talks, by schooling and
in contacts in conferences. As members of the Editorial Boards of Journal of Physics A  (IOP), of the Journal of Statistical Physics (Springer), of Annales Henri Poincar\'e, of
Fundamental Theories of Physics (Springer) and of the liquid matter board of the European Physical Society, etc we can organize special issues and create
special volumes dedicated to aspects of the project. Indeed, as the project deals with some
very novel aspects that have not appeared in standard or even not so standard treatments of
irreversible thermodynamics, an important effort will be necessary and will be made to reach
also less specialized researchers as well as scientists involved in possible applications.
%
Obviously an important cost then also goes to traveling and participation in conferences and
international discussions. Again, since the project is not entirely mainstream and some
concepts/questions are quite new, the project needs to invest in visibility and in numerous
exchanges on international platforms.  All team members will be
expected to show great mobility.  During the years, the various members meet once at each
node with everybody, at a rate of roughly one meeting every 10 months. PIs stay a few days
for talks of everyone and discussions. Postdocs and students can stay
longer, extending their visits for a month. That will allow perfect exchange of ideas and
progress.  Similar considerations apply for inviting scientists where
costs include traveling, housing and subsistence.  The project will also invest (still under other
direct costs) in bringing together various points of view on the role of dynamical activity
and the construction of nonequilibrium statistical mechanics. For that reason we foresee to
organize two major conferences, one at around month 20, the other towards the end of the
project.  


\eucommentary{
  \begin{compactitem}
  \item Provide a plan for disseminating and exploiting the project results. The
    plan, which should be proportionate to the scale of the project, should
    contain measures to be implemented both during and after the project.
  \item Explain how the proposed measures will help to achieve the expected
    impact of the project.
  \item Where relevant, include information on how the participants will manage
    the research data generated and/or collected during the project, in
    particular addressing the following issues\footnote{For further guidance on
      research data management, please refer to the H2020 Online Manual on the
      Participant Portal.}:
    \begin{compactitem}
    \item What types of data will the project generate/collect?  o What
      standards will be used?
    \item How will this data be exploited and/or shared/made accessible for
      verification and re-use?  If data cannot be made available, explain why.
    \item How will this data be curated and preserved?
    \end{compactitem}
 %      
    You will need an appropriate consortium agreement to manage (among other
    things) the ownership and access to key knowledge (IPR, data etc.). Where
    relevant, these will allow you, collectively and individually, to pursue
    market opportunities arising from the project's results.\\
%
    The appropriate structure of the consortium to support exploitation is
    addressed in section~3.3.
%
  \item Outline the strategy for knowledge management and protection. Include
    measures to provide open access (free on-line access, such as the 'green' or
    'gold' model) to peer- reviewed scientific publications which might result
    from the project.%
    \footnote{Open access must be granted to all scientific publications
      resulting from Horizon 2020 actions. Further guidance on open access is
      available in the H2020 Online Manual on the Participant Portal.}\\
%
    Open access publishing (also called 'gold' open access) means that an
    article is immediately provided in open access mode by the scientific
    publisher. The associated costs are usually shifted away from readers, and
    instead (for example) to the university or research institute to which the
    researcher is affiliated, or to the funding agency supporting the research.\\
%
    Self-archiving (also called 'green' open access) means that the published
    article or the final peer-reviewed manuscript is archived by the researcher
    - or a representative - in an online repository before, after or alongside
    its publication.  Access to this article is often - but not necessarily -
    delayed ('embargo period'), as some scientific publishers may wish to recoup
    their investment by selling subscriptions and charging pay-per-download/view
    fees during an exclusivity period.
%
  \end{compactitem}
}

\paragraph{b) Communication activities}

\eucommentary{
  \begin{compactitem}
  \item Describe the proposed communication measures for promoting the project
    and its findings during the period of the grant. Measures should be
    proportionate to the scale of the project, with clear objectives. They
    should be tailored to the needs of various audiences, including groups
    beyond the project's own community. Where relevant, include measures for
    public/societal engagement on issues related to the project.
  \end{compactitem}
}

Various members of the team have larger networks of communication both towards the general
public (public outreach) as towards industrial players and groups of interest.  The first
aim there is to transfer knowledge and tools, and to help in optimizing public interest. An
important initiative will be the organization of ``physics meets industry days'' where each
time during a week, specific problems of industry or economic activity will be presented.
They will be treated by students and experts towards helping to solve these problems, with
direct feedback toward the industry of company.  Such initiatives exist already in some
countries but will be started up in other European countries, and with an additional
selection and expertise platform related to complex and nonequilibrium phenomena.



%%% Local Variables:
%%% mode: latex
%%% TeX-master: "proposal"
%%% End:
